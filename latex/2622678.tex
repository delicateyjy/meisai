% 美赛模板:正文部分

\documentclass[12pt]{article}  % 官方要求字号不小于 12 号,此处选择 12 号字体
\usepackage[table]{xcolor}
% \linespread{1.1}
% \bibliographystyle{plain}
% 本模板不需要填写年份,以当前电脑时间自动生成
% 请在以下的方括号中填写队伍控制号
\usepackage[2622678]{easymcm}  % 载入 EasyMCM 模板文件
\problem{C}  % 请在此处填写题号
% \usepackage{mathptmx}  % 这是 Times 字体,中规中矩 
\usepackage{palatino}  % mathpazo 这palatino是 COMAP 官方杂志采用的更好看的 Palatino 字体,可替代以上的 mathptmx 宏包
\usepackage{pdfpages}
\usepackage{longtable}
\usepackage{tabu}
\usepackage{threeparttable}
\usepackage{listings}
\usepackage{paralist}
\usepackage{setspace}
\usepackage{float} %设置图片浮动位置的宏包
\usepackage{graphicx} %插入图片的宏包
\usepackage{subfigure} %插入多图时用子图显示的宏包
\usepackage[normalem]{ulem}
\useunder{\uline}{\ul}{}



 \let\itemize\compactitem
 \let\enditemize\endcompactitem
% \let\enumerate\compactenum
% \let\endenumerate\endcompactenum
% \let\description\compactdesc
% \let\enddescription\endcompactdesc
% \usepackage{biblatex} 
% \usepackage{cite}
% \usepackage{natbib}
\newcommand{\upcite}[1]{\textsuperscript{\textsuperscript{\cite{#1}}}}
\title{One Space, One Future}  % 标题

% 如需要修改题头(默认为 MCM/ICM),请使用以下命令(此处修改为 MCM)
%\renewcommand{\contest}{MCM}

% 文档开始
\begin{document}
\begin{abstract}
	
	% (Need to be reviewed)
With the development of space technology, outer space resources are gradually being exploited, and the asteroid mining industry is expected to have a high return on investment in the future. The United Nations has developed a series of treaties to maintain the common interests of humanity and global equity in outer space. But with the development of the asteroid mining industry, these fair international conventions might be broken.

We clarify \textbf{global equity} as "Whether or not a country has the ability to participate in asteroid mining, it can gain some benefit from asteroid mining". Then, we evaluated 25 countries with better space science and technology through the \textbf{TOPSIS method} in four aspects: \textbf{\emph{Economic Level (EL), Energy Output (EO), Sense of Responsibility (SR) and Scientific and Technological Level (STL)}}, and the weights of subordinate indicators in each aspect is determined by \textbf{AHP-EWM method}. Using AHP to determine the weights of four superior indicators and calculate the weighted average of their scores, define the results as "the \textbf{Fair Possibility Coefficient (FPC)} of each country", the larger the coefficient is, the greater chance a country has to gain profit from asteroid mining. Finally, \textbf{\emph{the US, Russia and China}} ranked in the top 3. Referring to the \textbf{Gini Coefficient (GC)} in Economics, we plotted the \textbf{Lorenz curve} of the distribution of FPC by country, and get \textbf{\emph{GC$ = $0.2277}}, which is less than 0.3\upcite{6}. So asteroid mining is globally fair at present, we called this the \textbf{FPC-GC Model}.

After studying the current international situation, we believe that the asteroid mining industry will face the danger of a few countries monopolizing the entire industry in the future. To prove our idea, we used the \textbf{Hierarchical Cluster method} to classify countries into two types, and establish the \textbf{Future Impact Forecast Model (FIF) based on interspecies competition differential equations} to analyze the changes of the comprehensive competitiveness of these two types of countries in the next 50 years. The result shows that the competitiveness of the first class countries will increase continuously. While the second class countries will develop in the opposite trend, therefore, the asteroid mining industry will be \textbf{\emph{completely monopolized}} by the first class countries. Referring to their final competitiveness, we calculated that \textbf{\emph{GC$ = $0.4423}} after 50 years, when the global equity will be seriously damaged.

By \textbf{changing the parameters} of the FIF model, we explore how global equity is affected by parameters after influencing conditions changed. So we suggest powerful countries to sacrifice some of their interests to help other countries whichh will benefit the common development of the world and the global equity of asteroid mining.

Finally, we put forward suggestions on the implementation of the policy from the two aspects of "amending existing treaties" and "making new policies", then we draw a \textbf{\emph{target timeline}}, hoping to basically achieve global equity by around 2070.











	
	
	% Here is the abstract of your paper.
	
	% Firstly, that is ...
	
	% Secondly, that is ...
	
	% Finally, that is ...
	
	% 美赛论文中无需注明关键字。若您一定要使用,
	% 请将以下两行的注释号 '%' 去除,以使其生效
	\vspace{5pt}
	\textbf{Keywords}: Asteroid mining; Global equity; Lorenz curve; Gini Coefficient; Cluster analysis; Differential equations; Policy recommendations
	
\end{abstract}
    
\maketitle  % 生成 Summary Sheet

\tableofcontents


% 正文开始
% Chapter 1: Introduction
\section{Introduction}
\subsection{Background}
Reality television is a global phenomenon that blends competitive tensions with mass entertainment. As the American version of this international franchise, Dancing with the Stars (DWTS) has gained immense popularity by pairing celebrities with professional dancers to compete for the championship. Over 34 seasons, the competition has paired celebrities with professional dancers, using a scoring system that combines technical scores from expert judges with popularity votes from fans.

The method of combining these evaluations—alternating between "rank-based" and "percentage-based" systems—remains a subject of significant controversy regarding fairness. Producers face the ongoing challenge of balancing competitive integrity with viewer engagement. This study utilizes mathematical modeling to evaluate these scoring mechanisms and propose a more balanced framework for future seasons.

\subsection{Our Work}
In order to examine the impact of asteroid mining on global equity both now and in the future, and what factors and impacts are relevant. We need to address many questions: Including but not limited to how to define and measure global equity? What will asteroid mining look like in the future given the current world situation? How will these changes affect global equity? What factors will influence the changes of asteroid mining in the future? Once these factors changed, how will the asteroid mining situation be? What kind of treaty should we conclude to make asteroid mining beneficial to all of humanity?

Our process for solving these problems is shown below:
\begin{figure}[H]
	\centering
	\includegraphics[width=.7\textwidth]{obian.png}
	\caption{Our work}
	\label{img}
\end{figure}
            
% \subsection{Literature Review}
% A literatrue\cite{1} say something about this problem ...

% \subsection{Our work}
% We do such things ...

% \begin{enumerate}[\bfseries 1.]
%     \item We do ...
%     \item We do ...
%     \item We do ...
% \end{enumerate}

% Chapter 2:
\section{Assumptions and Notations}

\begin{figure}[H]
	\centering
	\includegraphics[width = 1.0\textwidth]{img/t1_uncertainty.png}
	\caption{Certainty Analysis: Vote Share vs. Model Uncertainty}
	\label{img}
\end{figure}
\vspace{-0.6cm}

\subsection{Assumptions and Justifications}
In order to achieve a global and equitable analysis of asteroid mining, we make the following assumptions in a comprehensive consideration to better simplify and understand the problem:
\begin{itemize}
	\item The growth of the comprehensive competitiveness of each country is continuous and slow, without any sudden changes.\\
	\textbf{Justification:} A country's overall competitiveness, which is influenced by many factors, is unlikely to change abruptly.
	\item The data we collect is accurate.\\
	\textbf{Justification:} The data we collect are from authoritative organizations or websites such as the World Bank, so we can assume that they are true and accurate.
	\item The country is the smallest unit of analysis.\\
	\textbf{Justification:} It is impossible to achieve equality for all persons when considering asteroid mining, which would make the problem more complex and impractical, so in order to convenient our analysis, we take the country as the smallest unit and consider global equity.
	\item All countries will respond to the treaties.\\
	\textbf{Justification:} In order to analyze the impact of changing realistic conditions on global equity, we must assume that all countries will respond to the treaties concluded so that the parameters will change and our study will be worthwhile.
\end{itemize} 
\subsection{Notations}
Here we clarify some of the symbols and their definitions, others will be explained in the paper.
\vspace{-0.4cm}
\begin{table}[H]
	\centering
	\caption{Notations}
	\begin{tabular}{cc}
		\toprule[1.5pt]
		\textbf{Symbol} & \textbf{Description} \\ 
		\midrule
		$ \omega _{Aj} $ & The weight of the $ j^{th} $ indicator determined by AHP \\
		$ \omega _{Ej} $ & The weight of the $ j^{th} $ indicator determined by EWM \\ 
		$ FPC_i $ & FPC value of the $ i^{th} $ country \\
		$ GC $ & Gini Coefficient \\
		$ c $ & Comprehensive competitiveness \\
		$ \alpha $ & Relative degree of dependence \\
		$ \beta $ & Comprehensive development factor \\
		\bottomrule[1.5pt]
	\end{tabular}
\end{table}
\vspace{-0.6cm}


% ----------------------------------------------------------------------------
% 第4章:Task 1 - 投票预估模型
% ----------------------------------------------------------------------------
%
% ============================================================================
% 模型名称: Feature-Enhanced Adaptive Bayesian Vote Estimation (FABVE)
%           特征增强自适应贝叶斯投票估计模型
% ============================================================================
%
% 核心挑战:观众投票数据从未公开,只能观察评委分数和淘汰结果
% 建模思路:贝叶斯推断 + 软约束 + 特征增强先验 + 自适应MCMC
%
% ============================================================================

\section{Task 1: Vote Estimation Model}

% ============================================================================
% 4.1 问题形式化
% ============================================================================

\subsection{Problem Formulation}

% 在《与星共舞》节目中,每周的淘汰由评委分数和观众投票共同决定。
In \textit{Dancing with the Stars}, weekly eliminations are determined by a combination
of judges' scores and audience votes.
% 然而,在节目34季的历史中,实际的观众投票数据从未被公开披露。
However, the show has never disclosed the actual
audience voting data throughout its 34-season history.
% 这构成了一个经典的逆问题:我们必须从可观测的淘汰结果反推未观测到的投票分布。
This constitutes a classic
\textbf{inverse problem}: we must infer the unobserved voting distribution from
the observable elimination outcomes.

% 形式化地,对于每季s的每周w,我们观察到:
Formally, for each week $w$ in season $s$, we observe:
\begin{itemize}
    % 评委分数向量S,其中每个分数在4到40分之间
    \item Judge scores $\mathbf{S} = (S_1, S_2, \ldots, S_n)$, where $S_i \in [4, 40]$
    % 淘汰结果E,即被淘汰选手的索引
    \item Elimination result $E \in \{1, 2, \ldots, n\}$ (the index of the eliminated contestant)
\end{itemize}

% 我们的目标是估计未观测到的投票份额分布π
Our goal is to estimate the unobserved vote share distribution
$\boldsymbol{\pi} = (\pi_1, \ldots, \pi_n)$,
% 其中π_i表示选手i获得的观众投票比例
where $\pi_i$ represents the fraction
of audience votes received by contestant $i$, satisfying:
% 约束条件:所有份额之和为1,且每个份额非负
\begin{equation}
\sum_{i=1}^{n} \pi_i = 1, \quad \pi_i \geq 0
\end{equation}

% 淘汰由节目的计分规则决定:
The elimination is governed by the show's scoring rules:

% 排名法(第1-2季,第28-34季):
\textbf{Ranking Method} (Seasons 1--2, 28--34):
% 综合排名 = 评委排名 + 投票排名,排名数最高者被淘汰
\begin{equation}
C_i^{\text{rank}} = R_i^{S} + R_i^{\pi}, \quad E = \arg\max_i C_i^{\text{rank}}
\end{equation}
% 其中R_i^S和R_i^π分别是选手i按评委分数和粉丝投票的排名
where $R_i^{S}$ and $R_i^{\pi}$ represent the contestant's ranks by judge scores and fan votes, respectively.

% 百分比法(第3-27季):
\textbf{Percentage Method} (Seasons 3--27):
% 综合得分 = 评委分数百分比 + 粉丝投票百分比,得分最低者被淘汰
\begin{equation}
C_i^{\text{pct}} = \frac{S_i}{\sum_j S_j} + \pi_i, \quad E = \arg\min_i C_i^{\text{pct}}
\end{equation}

% 从数学上看,给定淘汰结果E,存在无穷多个可能产生该结果的投票分布。

Mathematically, given an elimination outcome $E$, there exist infinitely many fan votes distributions that satisfy this result.
% 我们的任务是通过融入先验知识来识别最合理的分布,同时恰当地量化固有的不确定性。
Our task is to identify the most plausible distribution
by incorporating prior knowledge while properly quantifying the inherent uncertainty.

% ============================================================================
% 4.2 方法论
% ============================================================================

\subsection{Methodology: Bayesian Vote Estimation}

% 我们采用贝叶斯推断框架来解决这个逆问题。
We adopt a \textbf{Bayesian inference framework} to address this inverse problem.
% 该框架提供三个关键优势:
This framework offers three key advantages:
\begin{enumerate}
    % 1. 不确定性量化:提供完整的后验分布而非点估计
    \item \textbf{Uncertainty quantification}: Provides complete posterior distributions rather than point estimates
    % 2. 先验知识融合:融入关于投票行为的领域知识
    \item \textbf{Prior knowledge integration}: Incorporates domain knowledge about voting behavior
    % 3. 软约束:以概率方式使用淘汰信息,而非硬约束
    \item \textbf{Soft constraints}: Uses elimination information probabilistically rather than as hard constraints
\end{enumerate}

% 后验分布由贝叶斯定理给出:
The posterior distribution is given by Bayes' theorem:
% 后验 ∝ 似然 × 先验
\begin{equation}
P(\boldsymbol{\pi} | E, \mathbf{S}) \propto
  P(E | \boldsymbol{\pi}, \mathbf{S}) \cdot P(\boldsymbol{\pi} | \mathbf{S}, \mathbf{X})
\end{equation}
% 其中X表示选手特征(职业、年龄等)
where $\mathbf{X}$ denotes contestant features (profession, age, etc.).

% --- 先验分布 ---
\subsubsection{Prior Distribution}

% 我们使用Dirichlet分布作为先验:
We use a Dirichlet distribution as the prior:
\begin{equation}
\boldsymbol{\pi} \sim \text{Dirichlet}(\alpha \cdot \boldsymbol{\theta})
\end{equation}

% 先验权重向量θ设计为加权混合:
The prior weight vector $\boldsymbol{\theta}$ is designed as a weighted mixture:
\begin{equation}
\theta_i = (1 - \beta) \cdot f_i^{\text{feature}} + \beta \cdot \tilde{S}_i
\end{equation}
% 其中S̃_i是归一化的评委分数
where $\tilde{S}_i = S_i / \sum_j S_j$ is the normalized judge scores.

% 基于特征的权重:f_i^feature项融入职业人气权重和年龄效应
\textbf{Feature-based weights}: The term $f_i^{\text{feature}}$ incorporates profession
popularity weights (e.g., singers and actors typically have larger fan bases) and age effects.
% 这些权重基于领域知识:某些职业天然拥有更大的粉丝群体
These weights are based on domain knowledge that certain professions inherently command
larger followings.
% 重要的是,敏感性分析表明模型结果对这些权重设置的变化具有鲁棒性
Importantly, our sensitivity analysis (Section~\ref{sec:sensitivity})
demonstrates that model results are robust to variations in these weight settings,
% 因为主要信息来自淘汰约束而非先验
as the primary information comes from elimination constraints rather than priors.

% 年龄效应:对25-35岁年龄段有轻微偏好(社交媒体参与度更高)
\textbf{Age effect}:
\begin{equation}
\text{age\_factor}_i = 1.0 - 0.01 \times |a_i - 30|
\end{equation}
% 限制在[0.7, 1.2]范围内
capped within $[0.7, 1.2]$.

% --- 似然函数 ---
\subsubsection{Likelihood Function}

% 我们采用softmax软约束而非硬约束:
We employ a softmax soft constraint instead of hard constraints:

% 百分比法的似然:
\textbf{For Percentage Method}:
% 综合得分低的选手被淘汰概率更高(指数形式)
\begin{equation}
P(E = k | \boldsymbol{\pi}, \mathbf{S}) =
  \frac{\exp(-\tau \cdot C_k^{\text{pct}})}{\sum_{j} \exp(-\tau \cdot C_j^{\text{pct}})}
\end{equation}

% 排名法的似然:
\textbf{For Ranking Method}:
% 综合排名高的选手被淘汰概率更高
\begin{equation}
P(E = k | \boldsymbol{\pi}, \mathbf{S}) =
  \frac{\exp(\tau \cdot C_k^{\text{rank}})}{\sum_{j} \exp(\tau \cdot C_j^{\text{rank}})}
\end{equation}

% 温度参数τ控制约束的严格程度
The temperature parameter $\tau$ controls constraint strictness.
% 我们使用动态温度:
We use dynamic temperature:
\begin{equation}
\tau_{\text{eff}} = \tau_0 \cdot (1 + 2 \cdot m)
\end{equation}
% 其中m衡量淘汰边界的清晰程度
where $m$ measures the clarity of the elimination margin.

% --- 为什么使用软约束而非硬约束 ---
% 解释软约束的设计动机
\textbf{Why soft constraints?}
% 硬约束方法只接受满足淘汰条件的样本,会导致循环论证
A hard constraint approach---accepting only samples where the eliminated contestant
has the lowest combined score---leads to circular reasoning:
% 模型会达到100%的"预测准确率",但这只是因为我们强制满足了约束
the model would achieve 100\% ``prediction accuracy'' simply because we forced
the constraint to be satisfied.
% 这不提供真正的预测洞察
This provides no genuine predictive insight.

% 软约束通过概率方式使用淘汰信息,允许模型学习投票分布的结构
Our soft constraint formulation uses elimination information probabilistically,
% 而非机械地强制满足
allowing the model to learn the structure of vote distributions
rather than mechanically enforcing outcomes.
% 结果是更诚实的不确定性估计和更有意义的后验分布
The result is more honest uncertainty estimates and more meaningful posterior distributions.

% ============================================================================
% 4.3 模型实现
% ============================================================================

\subsection{Model Implementation}

% 由于后验分布没有解析解,我们使用MCMC采样来近似它
Since the posterior distribution lacks an analytical solution, we approximate it
using Markov Chain Monte Carlo (MCMC) sampling.

% --- 单纯形上的提议分布 ---
\subsubsection{Proposal Distribution on the Simplex}

% 我们使用Dirichlet提议分布,它自然满足单纯形约束:
We use a Dirichlet proposal that naturally respects the simplex constraint:
\begin{equation}
\boldsymbol{\pi}' \sim \text{Dirichlet}\left(\frac{\boldsymbol{\pi}}{\lambda}\right)
\end{equation}

% --- 自适应步长 ---
\subsubsection{Adaptive Step Size}

% 步长λ自适应调整以达到约30%的目标接受率:
The step size $\lambda$ is adaptively adjusted to achieve a target acceptance rate of 30\%:
\begin{equation}
\lambda^{(t+1)} = \begin{cases}
  0.95 \cdot \lambda^{(t)} & r_{\text{recent}} < 0.25 \\
  1.05 \cdot \lambda^{(t)} & r_{\text{recent}} > 0.35 \\
  \lambda^{(t)} & \text{otherwise}
\end{cases}
\end{equation}

% --- 超参数设置 ---
\subsubsection{Hyperparameter Settings}

% 默认超参数:τ_0=12, α=1.5, β=0.3, λ_0=0.06
The default hyperparameters are: $\tau_0=12$, $\alpha=1.5$, $\beta=0.3$, $\lambda_0=0.06$,
% 在3000次预热后采集10000个后验样本
with $N=10000$ posterior samples after $B=3000$ burn-in iterations.
% 这些值通过敏感性分析确定,以平衡模型性能和计算效率
These values were
determined through sensitivity analysis to balance model performance and computational
efficiency.
% 固定随机种子以确保可重复性
The random seed is fixed to ensure reproducibility.

% ============================================================================
% 4.4 结果验证
% ============================================================================

\subsection{Results and Validation}

% --- 一致性指标(Q1a)---
\subsubsection{Consistency Metrics (Q1a)}

% 淘汰预测准确率(EPA):预测正确的周数占总周数的比例
\textbf{Elimination Prediction Accuracy (EPA)}:
\begin{equation}
\text{EPA} = \frac{1}{W} \sum_{w=1}^{W} \mathbb{I}[\hat{E}_w = E_w]
\end{equation}

% 平均淘汰概率:模型对实际淘汰结果的平均置信度
\textbf{Mean Elimination Probability}:
\begin{equation}
\bar{P}_{\text{elim}} = \frac{1}{W} \sum_{w=1}^{W} P(E_w | \bar{\boldsymbol{\pi}}_w, \mathbf{S}_w)
\end{equation}

% 其他指标包括Top-2准确率和Kendall's τ秩相关系数
Additional metrics include Top-2 accuracy and Kendall's $\tau$ rank correlation.

% 表格总结了34季的一致性指标
Table~\ref{tab:consistency} summarizes the consistency metrics in all 34 seasons.

\begin{table}[htbp]
\centering
\caption{Consistency Metrics Summary (Q1a)}
% 一致性指标汇总表
\label{tab:consistency}
\begin{tabular}{lc}
\hline
\textbf{Metric} & \textbf{Value} \\
\hline
% 分析的总周数
Total weeks analyzed & 301 \\
% 淘汰预测准确率
Elimination Prediction Accuracy (EPA) & 93.7\% \\
% Top-2准确率(淘汰者在预测的后两名中)
Top-2 Accuracy & 97.7\% \\
% 平均淘汰概率
Mean Elimination Probability & 0.481 \\
% 平均Kendall's τ
Mean Kendall's $\tau$ & 0.717 \\
\hline
% 按计分方法分类:
\multicolumn{2}{l}{\textit{By Scoring Method:}} \\
% 排名法(74周)
\quad Ranking Method (n=74) & EPA = 95.9\% \\
% 百分比法(227周)
\quad Percentage Method (n=227) & EPA = 93.0\% \\
\hline
\end{tabular}
\end{table}

% --- 一致性结果分析 ---
% 对表格数据进行解读,说明模型的有效性
The results demonstrate that our Bayesian vote estimation model achieves strong consistency
with observed elimination outcomes.
% 93.7%的EPA表明估计的投票在绝大多数情况下能正确预测淘汰
The 93.7\% EPA indicates that our estimated vote distributions correctly predict
the actual elimination in the vast majority of weeks.
% Top-2准确率97.7%更具说服力:几乎所有情况下,实际被淘汰者都在模型预测的后两名中
The 97.7\% Top-2 accuracy is particularly compelling: in nearly all cases, the
eliminated contestant was among the bottom two predicted by our model.

% 值得注意的是,排名法(95.9%)的EPA高于百分比法(93.0%)
Notably, the Ranking Method achieves higher EPA (95.9\%) compared to the Percentage
Method (93.0\%).
% 这一差异可以从数学性质解释:排名法的离散性质使得约束更加明确
This difference can be explained by the mathematical properties of each method:
the discrete nature of the Ranking Method creates sharper elimination boundaries,
% 使MCMC更容易收敛到满足约束的投票分布
making it easier for MCMC to converge to vote distributions that satisfy the constraints.
% 相比之下,百分比法的连续性质导致更大的可行解空间
In contrast, the continuous nature of the Percentage Method leads to a larger
feasible solution space, introducing more uncertainty.

% 平均淘汰概率0.481表明模型对淘汰预测有中等置信度
The mean elimination probability of 0.481 indicates moderate confidence in predictions.
% 这个值低于1.0是符合预期的:软约束允许非确定性预测
This value being well below 1.0 is expected and desirable: our soft constraint
formulation deliberately avoids deterministic predictions,
% 诚实地反映了逆问题固有的不确定性
honestly reflecting the inherent uncertainty of this inverse problem.

% Kendall's τ为0.717表明预测排名与隐含投票排名高度相关
The Kendall's $\tau$ of 0.717 indicates strong rank correlation between predicted
elimination order and the implied vote rankings,
% 进一步验证了模型捕捉投票行为模式的能力
further validating the model's ability to capture voting behavior patterns.

% --- 确定性指标(Q1b)---
\subsubsection{Certainty Metrics (Q1b)}

% 变异系数(CV):标准差与均值的比值,衡量估计的相对不确定性
\textbf{Coefficient of Variation (CV)}:
\begin{equation}
\text{CV}_i = \frac{\sigma_{\pi_i}}{\mu_{\pi_i}}
\end{equation}

% 95%可信区间宽度:后验分布的2.5%和97.5%分位数之差
\textbf{95\% Credible Interval Width}:
\begin{equation}
\text{CI}_i = \pi_i^{97.5\%} - \pi_i^{2.5\%}
\end{equation}

% 我们还报告有效样本量(ESS)和MCMC接受率来评估采样质量
We also report effective sample size (ESS) and MCMC acceptance rates to assess
sampling quality.

% 表格总结了确定性指标
Table~\ref{tab:certainty} summarizes the certainty metrics.

\begin{table}[htbp]
\centering
\caption{Certainty Metrics Summary (Q1b)}
% 确定性指标汇总表
\label{tab:certainty}
\begin{tabular}{lc}
\hline
\textbf{Metric} & \textbf{Value} \\
\hline
% 所有选手的平均CV
Mean CV (all contestants) & 0.710 \\
% 平均95%可信区间宽度
Mean 95\% CI Width & 0.292 \\
\hline
% 按淘汰状态分类:
\multicolumn{2}{l}{\textit{By Elimination Status:}} \\
% 被淘汰选手的CV
\quad Eliminated contestants CV & 0.761 \\
% 未被淘汰选手的CV
\quad Non-eliminated contestants CV & 0.704 \\
% 被淘汰选手的CI宽度
\quad Eliminated contestants CI width & 0.167 \\
% 未被淘汰选手的CI宽度
\quad Non-eliminated contestants CI width & 0.308 \\
\hline
\end{tabular}
\end{table}

% --- 确定性结果分析 ---
% 对不确定性指标进行解读
The certainty metrics reveal important patterns about estimation uncertainty.
% 平均CV为0.710表明投票估计存在相当的不确定性,这是逆问题的固有特征
The mean CV of 0.710 indicates substantial uncertainty in vote estimates,
which is an inherent characteristic of this inverse problem rather than a model deficiency.

% 一个关键发现是被淘汰者与未淘汰者之间的不确定性差异
A key finding is the asymmetry between eliminated and non-eliminated contestants.
% 被淘汰者的CI宽度(0.167)显著小于未淘汰者(0.308)
Eliminated contestants have narrower CI widths (0.167) compared to non-eliminated
contestants (0.308).
% 这是符合直觉的:淘汰事件为被淘汰者的投票份额提供了强约束(必须是最低的)
This is intuitive: the elimination event provides a strong constraint on the
eliminated contestant's vote share---it must be low enough to result in elimination.
% 而未淘汰者只需满足"不是最低"的弱约束,允许更大的变化范围
Non-eliminated contestants only need to satisfy the weaker constraint of ``not being
the lowest,'' allowing for a wider range of plausible values.

% 尽管被淘汰者的CV(0.761)略高于未淘汰者(0.704)
Despite eliminated contestants having slightly higher CV (0.761 vs. 0.704),
% 这是因为被淘汰者通常投票份额较低,导致相同的绝对不确定性产生更大的相对不确定性
this is because eliminated contestants typically have lower vote shares,
causing the same absolute uncertainty to translate into larger relative uncertainty.

% --- 不确定性解读 ---
\subsubsection{Interpretation of Uncertainty}

% 投票估计中的不确定性反映了该逆问题固有的不确定性
The uncertainty in vote estimates reflects the inherent indeterminacy of this inverse
problem.
% 我们的模型提供完整的后验分布而非点估计
Rather than point estimates, our model provides complete posterior distributions.
% 例如,后验均值15%、95%CI为[12%,18%]表示:
For example, a posterior mean of 15\% with 95\% CI [12\%, 18\%] indicates that
% "我们有95%的置信度认为该选手获得了12%到18%的观众投票"
``we are 95\% confident that the contestant received between 12\% and 18\% of audience votes.''


% ----------------------------------------------------------------------------
% 第5章:Task 2 - 计分方法比较
% ----------------------------------------------------------------------------
\section{Task 2: Scoring Method Comparison}

\subsection{Ranking Method vs. Percentage Method}

% 回顾3.3节中定义的两种计分方法,这里聚焦于它们的关键差异。
Recall the two scoring methods defined in Section~\ref{sec:scoring-evolution}.
% 我们不重复数学公式,而是关注影响淘汰决策的核心差异。
Rather than repeating the mathematical formulations, we focus on the key differences
that affect elimination decisions.

% 排名法将评委分数和观众投票分别转换为名次后相加。
The Ranking Method converts both judges' scores and audience votes to ordinal ranks
before combining them.
% 这种转换消除了原始分数的数值信息,只保留相对顺序。
This conversion discards the numerical magnitude, preserving only relative ordering.
% 因此,30分和25分的差距在排名法中与25分和24分的差距等价——都只差一个名次。
Consequently, a gap between 30 and 25 points carries the same weight as a gap between
25 and 24 points---both represent a single rank difference.

% 相比之下,百分比法保留分数的数值信息。
In contrast, the Percentage Method preserves numerical information.
% 分数差距大的选手在百分比贡献上也会有更大差距。
Contestants with larger score gaps will have correspondingly larger gaps in their
percentage contributions.
% 这使得百分比法对评委分数的离散程度更加敏感。
This makes the Percentage Method more sensitive to the dispersion of judges' scores.

% 两种方法的核心差异可以从三个维度理解:
The fundamental differences can be understood along three dimensions:
\begin{itemize}
    % 权重结构:固定 vs 动态
    \item \textbf{Weight structure}: The Ranking Method assigns fixed 50\%--50\% weights
    to judges and audience (in rank contribution), while the Percentage Method has
    dynamic effective weights depending on score distributions.
    % 敏感性:离散 vs 连续
    \item \textbf{Sensitivity}: The Ranking Method exhibits step-function sensitivity
    (only responds when ranks change), while the Percentage Method responds
    continuously to any score change.
    % 极端值抗性
    \item \textbf{Robustness to extremes}: The Ranking Method is robust to outliers
    since it only uses ordinal information, while the Percentage Method can be
    dominated by extreme scores.
\end{itemize}

% 这些差异意味着两种方法可能产生不同的淘汰决定。
These differences imply that the two methods may produce different elimination
decisions for the same underlying data.
% 我们接下来量化这些差异的频率和模式。
We next quantify the frequency and patterns of such discrepancies.

\subsection{Comparative Metrics}

% 为了系统地比较两种计分方法,我们定义以下评估指标。
To systematically compare the two scoring methods, we define evaluation metrics
targeting the core question: does one method favor fan votes more?

% === 淘汰翻转率 ===
\subsubsection{Elimination Flip Rate (EFR)}

% 淘汰翻转率衡量当切换计分方法时,淘汰决定改变的频率。
The Elimination Flip Rate measures how often the elimination decision changes
when switching between scoring methods.
% 形式化定义:
Formally:
\begin{equation}
    \text{EFR} = \frac{1}{W}\sum_{w=1}^{W} \mathbf{1}[E^{\text{actual}}_w \neq E^{\text{cf}}_w]
\end{equation}
% 其中W是分析的总周数,E表示被淘汰选手。
where $W$ is the total number of weeks, and $E^{\text{actual}}_w$, $E^{\text{cf}}_w$
denote the eliminated contestants under actual and counterfactual methods.

% === 有效权重 ===
\subsubsection{Effective Weight}

% 核心问题是:在每种方法下,评委和观众的实际贡献比例是多少?
The key question is: what is the actual contribution of judges versus audience
under each method?
% 我们通过有效权重来回答。
We answer this through \textit{effective weight}, defined as the variance
contribution to the combined score:
\begin{equation}
    \alpha^{\text{pct}} = \frac{\text{Var}(P^{(S)})}{\text{Var}(P^{(S)}) + \text{Var}(P^{(V)})}
\end{equation}
% 其中P^(S)是评委分数百分比,P^(V)是投票百分比。
where $P^{(S)}$ and $P^{(V)}$ are the judges' score and vote percentages.
% 观众有效权重为1-α。
The effective audience weight is $1 - \alpha$.
% 对于排名法,理论权重固定为0.5。
For the Ranking Method, the theoretical weight is fixed at 0.5.

\subsection{Results}

% 我们使用Task 1的投票估计结果,对34季301周数据进行反事实模拟分析。
Using the vote estimates from Task~1, we conduct counterfactual simulation on
301 weeks across all 34 seasons.

% 表格总结了反事实分析的整体结果。
Table~\ref{tab:counterfactual-summary} summarizes the counterfactual analysis results.

\begin{table}[htbp]
\centering
\caption{Counterfactual Analysis Summary}
% 反事实分析汇总表
\label{tab:counterfactual-summary}
\begin{tabular}{lccc}
\hline
\textbf{Metric} & \textbf{Overall} & \textbf{Ranking} & \textbf{Percentage} \\
\hline
% 分析周数
Weeks analyzed & 301 & 74 & 227 \\
% 淘汰翻转次数
Elimination flips & 66 & 0 & 66 \\
% 淘汰翻转率
Elimination Flip Rate & 21.9\% & 0.0\% & 29.1\% \\
% 平均Kendall's τ
Mean Kendall's $\tau$ & 0.828 & 0.703 & 0.868 \\
% 平均Spearman's ρ
Mean Spearman's $\rho$ & 0.897 & 0.785 & 0.934 \\
\hline
\end{tabular}
\end{table}

% 关键发现:排名法时期翻转率为0,百分比法时期约29%的周会产生不同淘汰结果。
A striking finding is that all 66 elimination flips occurred during the
Percentage Method era (Seasons 3--27).
% 排名法时期(第1-2季和第28-34季)没有发生任何翻转。
No flips occurred during the Ranking Method periods (Seasons 1--2 and 28--34).

% 表格展示了两种方法下的有效权重分析结果。
Table~\ref{tab:weight-decomposition} presents the effective weight analysis,
which directly addresses the question of method bias.

\begin{table}[htbp]
\centering
\caption{Effective Weight Analysis}
% 有效权重分析表
\label{tab:weight-decomposition}
\begin{tabular}{lcc}
\hline
\textbf{Metric} & \textbf{Ranking Method} & \textbf{Percentage Method} \\
\hline
% 评委有效权重
Judge effective weight & 48.2\% & 22.9\% \\
% 观众有效权重
Audience effective weight & 51.8\% & 77.1\% \\
% 观众权重>50%的周数占比
Weeks with audience weight $>$50\% & $\sim$50\% & 93.4\% \\
\hline
\end{tabular}
\end{table}

% 权重分解回答了核心问题:哪种方法更偏向观众?
The weight decomposition directly answers the core question: \textbf{which method
favors fan votes more?}
% 百分比法下观众有效权重平均达到77.1%,远高于排名法的51.8%。
Under the Percentage Method, the effective audience weight averages 77.1\%,
substantially exceeding the Ranking Method's 51.8\%.
% 93.4%的周次中,百分比法给予观众超过50%的权重。
In 93.4\% of weeks, the Percentage Method assigns more than half the effective
weight to audience votes.

% 这个结论可能与直觉相反。
This finding may seem counterintuitive.
% 解释:百分比法的有效权重取决于分数分布的方差。
The explanation lies in variance: the Percentage Method's effective weights
depend on the dispersion of scores.
% 当投票分布比评委分数更分散时,投票对综合得分的影响更大。
When vote distributions are more dispersed than judge scores---as is typically
the case with our estimates---votes dominate the combined score.

\subsection{Conclusion: The Percentage Method Favors Fan Votes}

% 基于以上分析,我们得出明确结论:
Based on our analysis, we reach a clear conclusion:

% 百分比法更偏向观众投票。
\textbf{The Percentage Method favors fan votes more than the Ranking Method.}

% 证据如下:
The evidence is threefold:
\begin{enumerate}
    % 证据1:有效权重
    \item \textbf{Effective weights}: The Percentage Method assigns 77.1\% effective
    weight to audience votes, compared to 51.8\% under the Ranking Method.
    % 证据2:翻转模式
    \item \textbf{Flip patterns}: All 66 elimination flips occurred during the
    Percentage Method era, suggesting this method is more susceptible to vote
    fluctuations overriding judge preferences.
    % 证据3:争议案例
    \item \textbf{Controversy cases}: Notable controversies (Bobby Bones in Season 27,
    Bristol Palin in Season 11) all occurred during the Percentage Method era,
    where high fan support could overcome low judge scores.
\end{enumerate}

% 这解释了为什么节目在第28季后回归排名法。
This explains why the show returned to the Ranking Method starting in Season 28:
% 制作方希望恢复评委与观众之间的平衡。
producers sought to restore balance between judges and audience influence.

% 这些发现将为Task 3的争议案例分析提供分析框架。
These findings provide the analytical framework for examining specific
controversy cases in Task~3.


\end{document} 
