% 美赛模板:正文部分

\documentclass[12pt]{article}  % 官方要求字号不小于 12 号,此处选择 12 号字体
\usepackage[table]{xcolor}
% \linespread{1.1}
% \bibliographystyle{plain}
% 本模板不需要填写年份,以当前电脑时间自动生成
% 请在以下的方括号中填写队伍控制号
\usepackage[2622678]{easymcm}  % 载入 EasyMCM 模板文件
\problem{C}  % 请在此处填写题号
% \usepackage{mathptmx}  % 这是 Times 字体,中规中矩 
\usepackage{palatino}  % mathpazo 这palatino是 COMAP 官方杂志采用的更好看的 Palatino 字体,可替代以上的 mathptmx 宏包
\usepackage{pdfpages}
\usepackage{longtable}
\usepackage{tabu}
\usepackage{threeparttable}
\usepackage{listings}
\usepackage{paralist}
\usepackage{setspace}
\usepackage{float} %设置图片浮动位置的宏包
\usepackage{graphicx} %插入图片的宏包
\usepackage{subfigure} %插入多图时用子图显示的宏包
\usepackage[normalem]{ulem}
\useunder{\uline}{\ul}{}



 \let\itemize\compactitem
 \let\enditemize\endcompactitem
% \let\enumerate\compactenum
% \let\endenumerate\endcompactenum
% \let\description\compactdesc
% \let\enddescription\endcompactdesc
% \usepackage{biblatex} 
% \usepackage{cite}
% \usepackage{natbib}
\newcommand{\upcite}[1]{\textsuperscript{\textsuperscript{\cite{#1}}}}
\title{One Space, One Future}  % 标题

% 如需要修改题头(默认为 MCM/ICM),请使用以下命令(此处修改为 MCM)
%\renewcommand{\contest}{MCM}

% 文档开始
\begin{document}
\begin{abstract}
	
	% (Need to be reviewed)
With the development of space technology, outer space resources are gradually being exploited, and the asteroid mining industry is expected to have a high return on investment in the future. The United Nations has developed a series of treaties to maintain the common interests of humanity and global equity in outer space. But with the development of the asteroid mining industry, these fair international conventions might be broken.

We clarify \textbf{global equity} as "Whether or not a country has the ability to participate in asteroid mining, it can gain some benefit from asteroid mining". Then, we evaluated 25 countries with better space science and technology through the \textbf{TOPSIS method} in four aspects: \textbf{\emph{Economic Level (EL), Energy Output (EO), Sense of Responsibility (SR) and Scientific and Technological Level (STL)}}, and the weights of subordinate indicators in each aspect is determined by \textbf{AHP-EWM method}. Using AHP to determine the weights of four superior indicators and calculate the weighted average of their scores, define the results as "the \textbf{Fair Possibility Coefficient (FPC)} of each country", the larger the coefficient is, the greater chance a country has to gain profit from asteroid mining. Finally, \textbf{\emph{the US, Russia and China}} ranked in the top 3. Referring to the \textbf{Gini Coefficient (GC)} in Economics, we plotted the \textbf{Lorenz curve} of the distribution of FPC by country, and get \textbf{\emph{GC$ = $0.2277}}, which is less than 0.3\upcite{6}. So asteroid mining is globally fair at present, we called this the \textbf{FPC-GC Model}.

After studying the current international situation, we believe that the asteroid mining industry will face the danger of a few countries monopolizing the entire industry in the future. To prove our idea, we used the \textbf{Hierarchical Cluster method} to classify countries into two types, and establish the \textbf{Future Impact Forecast Model (FIF) based on interspecies competition differential equations} to analyze the changes of the comprehensive competitiveness of these two types of countries in the next 50 years. The result shows that the competitiveness of the first class countries will increase continuously. While the second class countries will develop in the opposite trend, therefore, the asteroid mining industry will be \textbf{\emph{completely monopolized}} by the first class countries. Referring to their final competitiveness, we calculated that \textbf{\emph{GC$ = $0.4423}} after 50 years, when the global equity will be seriously damaged.

By \textbf{changing the parameters} of the FIF model, we explore how global equity is affected by parameters after influencing conditions changed. So we suggest powerful countries to sacrifice some of their interests to help other countries whichh will benefit the common development of the world and the global equity of asteroid mining.

Finally, we put forward suggestions on the implementation of the policy from the two aspects of "amending existing treaties" and "making new policies", then we draw a \textbf{\emph{target timeline}}, hoping to basically achieve global equity by around 2070.











	
	
	% Here is the abstract of your paper.
	
	% Firstly, that is ...
	
	% Secondly, that is ...
	
	% Finally, that is ...
	
	% 美赛论文中无需注明关键字。若您一定要使用,
	% 请将以下两行的注释号 '%' 去除,以使其生效
	\vspace{5pt}
	\textbf{Keywords}: Asteroid mining; Global equity; Lorenz curve; Gini Coefficient; Cluster analysis; Differential equations; Policy recommendations
	
\end{abstract}
    
\maketitle  % 生成 Summary Sheet

\tableofcontents


% 正文开始
% Chapter 1: Introduction
\section{Introduction}
\subsection{Background}
Reality television is a global phenomenon that blends competitive tensions with mass entertainment. As the American version of this international franchise, Dancing with the Stars (DWTS) has gained immense popularity by pairing celebrities with professional dancers to compete for the championship. Over 34 seasons, the competition has paired celebrities with professional dancers, using a scoring system that combines technical scores from expert judges with popularity votes from fans.

The method of combining these evaluations—alternating between "rank-based" and "percentage-based" systems—remains a subject of significant controversy regarding fairness. Producers face the ongoing challenge of balancing competitive integrity with viewer engagement. This study utilizes mathematical modeling to evaluate these scoring mechanisms and propose a more balanced framework for future seasons.

\subsection{Our Work}
In order to examine the impact of asteroid mining on global equity both now and in the future, and what factors and impacts are relevant. We need to address many questions: Including but not limited to how to define and measure global equity? What will asteroid mining look like in the future given the current world situation? How will these changes affect global equity? What factors will influence the changes of asteroid mining in the future? Once these factors changed, how will the asteroid mining situation be? What kind of treaty should we conclude to make asteroid mining beneficial to all of humanity?

Our process for solving these problems is shown below:
\begin{figure}[H]
	\centering
	\includegraphics[width=.7\textwidth]{obian.png}
	\caption{Our work}
	\label{img}
\end{figure}
            
% \subsection{Literature Review}
% A literatrue\cite{1} say something about this problem ...

% \subsection{Our work}
% We do such things ...

% \begin{enumerate}[\bfseries 1.]
%     \item We do ...
%     \item We do ...
%     \item We do ...
% \end{enumerate}

% Chapter 2:
\section{Assumptions and Notations}

\begin{figure}[H]
	\centering
	\includegraphics[width = 1.0\textwidth]{img/t1_uncertainty.png}
	\caption{Certainty Analysis: Vote Share vs. Model Uncertainty}
	\label{img}
\end{figure}
\vspace{-0.6cm}

\subsection{Assumptions and Justifications}
In order to achieve a global and equitable analysis of asteroid mining, we make the following assumptions in a comprehensive consideration to better simplify and understand the problem:
\begin{itemize}
	\item The growth of the comprehensive competitiveness of each country is continuous and slow, without any sudden changes.\\
	\textbf{Justification:} A country's overall competitiveness, which is influenced by many factors, is unlikely to change abruptly.
	\item The data we collect is accurate.\\
	\textbf{Justification:} The data we collect are from authoritative organizations or websites such as the World Bank, so we can assume that they are true and accurate.
	\item The country is the smallest unit of analysis.\\
	\textbf{Justification:} It is impossible to achieve equality for all persons when considering asteroid mining, which would make the problem more complex and impractical, so in order to convenient our analysis, we take the country as the smallest unit and consider global equity.
	\item All countries will respond to the treaties.\\
	\textbf{Justification:} In order to analyze the impact of changing realistic conditions on global equity, we must assume that all countries will respond to the treaties concluded so that the parameters will change and our study will be worthwhile.
\end{itemize} 
\subsection{Notations}
Here we clarify some of the symbols and their definitions, others will be explained in the paper.
\vspace{-0.4cm}
\begin{table}[H]
	\centering
	\caption{Notations}
	\begin{tabular}{cc}
		\toprule[1.5pt]
		\textbf{Symbol} & \textbf{Description} \\ 
		\midrule
		$ \omega _{Aj} $ & The weight of the $ j^{th} $ indicator determined by AHP \\
		$ \omega _{Ej} $ & The weight of the $ j^{th} $ indicator determined by EWM \\ 
		$ FPC_i $ & FPC value of the $ i^{th} $ country \\
		$ GC $ & Gini Coefficient \\
		$ c $ & Comprehensive competitiveness \\
		$ \alpha $ & Relative degree of dependence \\
		$ \beta $ & Comprehensive development factor \\
		\bottomrule[1.5pt]
	\end{tabular}
\end{table}
\vspace{-0.6cm}


% ----------------------------------------------------------------------------
% 第4章:Task 1 - 投票预估模型
% ----------------------------------------------------------------------------
%
% ============================================================================
% 模型名称: Feature-Enhanced Adaptive Bayesian Vote Estimation (FABVE)
%           特征增强自适应贝叶斯投票估计模型
% ============================================================================
%
% 核心挑战:观众投票数据从未公开,只能观察评委分数和淘汰结果
% 建模思路:贝叶斯推断 + 软约束 + 特征增强先验 + 自适应MCMC
%
% ============================================================================

\section{Task 1: Vote Estimation Model}

% ============================================================================
% 4.1 问题形式化
% ============================================================================

\subsection{Problem Formulation}

% 在《与星共舞》节目中,每周的淘汰由评委分数和观众投票共同决定。
In \textit{Dancing with the Stars}, weekly eliminations are determined by a combination
of judges' scores and audience votes.
% 然而,在节目34季的历史中,实际的观众投票数据从未被公开披露。
However, the show has never disclosed the actual
audience voting data throughout its 34-season history.
% 这构成了一个经典的逆问题:我们必须从可观测的淘汰结果反推未观测到的投票分布。
This constitutes a classic
\textbf{inverse problem}: we must infer the unobserved voting distribution from
the observable elimination outcomes.

% 形式化地,对于每季s的每周w,我们观察到:
Formally, for each week $w$ in season $s$, we observe:
\begin{itemize}
    % 评委分数向量S,其中每个分数在4到40分之间
    \item Judge scores $\mathbf{S} = (S_1, S_2, \ldots, S_n)$, where $S_i \in [4, 40]$
    % 淘汰结果E,即被淘汰选手的索引
    \item Elimination result $E \in \{1, 2, \ldots, n\}$ (the index of the eliminated contestant)
\end{itemize}

% 我们的目标是估计未观测到的投票份额分布π
Our goal is to estimate the unobserved vote share distribution
$\boldsymbol{\pi} = (\pi_1, \ldots, \pi_n)$,
% 其中π_i表示选手i获得的观众投票比例
where $\pi_i$ represents the fraction
of audience votes received by contestant $i$, satisfying:
% 约束条件:所有份额之和为1,且每个份额非负
\begin{equation}
\sum_{i=1}^{n} \pi_i = 1, \quad \pi_i \geq 0
\end{equation}

% 淘汰由节目的计分规则决定:
The elimination is governed by the show's scoring rules:

% 排名法(第1-2季,第28-34季):
\textbf{Ranking Method} (Seasons 1--2, 28--34):
% 综合排名 = 评委排名 + 投票排名,排名数最高者被淘汰
\begin{equation}
C_i^{\text{rank}} = R_i^{S} + R_i^{\pi}, \quad E = \arg\max_i C_i^{\text{rank}}
\end{equation}
% 其中R_i^S和R_i^π分别是选手i按评委分数和粉丝投票的排名
where $R_i^{S}$ and $R_i^{\pi}$ represent the contestant's ranks by judge scores and fan votes, respectively.

% 百分比法(第3-27季):
\textbf{Percentage Method} (Seasons 3--27):
% 综合得分 = 评委分数百分比 + 粉丝投票百分比,得分最低者被淘汰
\begin{equation}
C_i^{\text{pct}} = \frac{S_i}{\sum_j S_j} + \pi_i, \quad E = \arg\min_i C_i^{\text{pct}}
\end{equation}

% 从数学上看,给定淘汰结果E,存在无穷多个可能产生该结果的投票分布。

Mathematically, given an elimination outcome $E$, there exist infinitely many fan votes distributions that satisfy this result.
% 我们的任务是通过融入先验知识来识别最合理的分布,同时恰当地量化固有的不确定性。
Our task is to identify the most plausible distribution
by incorporating prior knowledge while properly quantifying the inherent uncertainty.

% ============================================================================
% 4.2 方法论
% ============================================================================

\subsection{Methodology: Bayesian Vote Estimation}

% 我们采用贝叶斯推断框架来解决这个逆问题。
We adopt a \textbf{Bayesian inference framework} to address this inverse problem.
% 该框架提供三个关键优势:
This framework offers three key advantages:
\begin{enumerate}
    % 1. 不确定性量化:提供完整的后验分布而非点估计
    \item \textbf{Uncertainty quantification}: Provides complete posterior distributions rather than point estimates
    % 2. 先验知识融合:融入关于投票行为的领域知识
    \item \textbf{Prior knowledge integration}: Incorporates domain knowledge about voting behavior
    % 3. 软约束:以概率方式使用淘汰信息,而非硬约束
    \item \textbf{Soft constraints}: Uses elimination information probabilistically rather than as hard constraints
\end{enumerate}

% 后验分布由贝叶斯定理给出:
The posterior distribution is given by Bayes' theorem:
% 后验 ∝ 似然 × 先验
\begin{equation}
P(\boldsymbol{\pi} | E, \mathbf{S}) \propto
  P(E | \boldsymbol{\pi}, \mathbf{S}) \cdot P(\boldsymbol{\pi} | \mathbf{S}, \mathbf{X})
\end{equation}
% 其中X表示选手特征(职业、年龄等)
where $\mathbf{X}$ denotes contestant features (profession, age, etc.).

% --- 先验分布 ---
\subsubsection{Prior Distribution}

% 我们使用Dirichlet分布作为先验:
We use a Dirichlet distribution as the prior:
\begin{equation}
\boldsymbol{\pi} \sim \text{Dirichlet}(\alpha \cdot \boldsymbol{\theta})
\end{equation}
% 其中α是控制集中度的先验强度参数,θ是先验权重向量
where $\alpha$ is the prior strength parameter controlling concentration,
and $\boldsymbol{\theta} = (\theta_1, \ldots, \theta_n)$ is the prior weight vector.

% 先验权重向量θ设计为加权混合:
The prior weight vector $\boldsymbol{\theta}$ is designed as a weighted mixture:
\begin{equation}
\theta_i = (1 - \beta) \cdot f_i^{\text{feature}} + \beta \cdot \tilde{S}_i
\end{equation}
% 其中β控制先验与评委分数的关联程度,f_i^feature是选手i的特征权重,S̃_i是归一化评委分数
where $\beta \in [0,1]$ controls the correlation between prior and judges' scores,
$f_i^{\text{feature}}$ is the feature-based weight for contestant $i$,
and $\tilde{S}_i = S_i / \sum_j S_j$ is the normalized judges' score.

% 基于特征的权重:f_i^feature项融入职业人气权重和年龄效应
\textbf{Feature-based weights}: The term $f_i^{\text{feature}}$ incorporates profession
popularity weights (e.g., singers and actors typically have larger fan bases) and age effects.
% 这些权重基于领域知识:某些职业天然拥有更大的粉丝群体
These weights are based on domain knowledge that certain professions inherently command
larger followings.
% 重要的是,敏感性分析表明模型结果对这些权重设置的变化具有鲁棒性
Importantly, our sensitivity analysis (Section~\ref{sec:sensitivity})
demonstrates that model results are robust to variations in these weight settings,
% 因为主要信息来自淘汰约束而非先验
as the primary information comes from elimination constraints rather than priors.

% 年龄效应:对25-35岁年龄段有轻微偏好(社交媒体参与度更高)
\textbf{Age effect}:
\begin{equation}
\text{age\_factor}_i = 1.0 - 0.01 \times |a_i - 30|
\end{equation}
% 其中a_i是选手i的年龄,结果限制在[0.7, 1.2]范围内
where $a_i$ is the age of contestant $i$, capped within $[0.7, 1.2]$.

% --- 似然函数 ---
\subsubsection{Likelihood Function}

% 我们采用softmax软约束而非硬约束:
We employ a softmax soft constraint instead of hard constraints:

% 百分比法的似然:
\textbf{For Percentage Method}:
% 综合得分低的选手被淘汰概率更高(指数形式)
\begin{equation}
P(E = k | \boldsymbol{\pi}, \mathbf{S}) =
  \frac{\exp(-\tau \cdot C_k^{\text{pct}})}{\sum_{j} \exp(-\tau \cdot C_j^{\text{pct}})}
\end{equation}

% 排名法的似然:
\textbf{For Ranking Method}:
% 综合排名高的选手被淘汰概率更高
\begin{equation}
P(E = k | \boldsymbol{\pi}, \mathbf{S}) =
  \frac{\exp(\tau \cdot C_k^{\text{rank}})}{\sum_{j} \exp(\tau \cdot C_j^{\text{rank}})}
\end{equation}

% 温度参数τ控制约束的严格程度:值越高,对淘汰结果的约束越严格
The temperature parameter $\tau$ controls constraint strictness:
higher values enforce stricter adherence to elimination outcomes.
% 我们使用动态温度:
We use dynamic temperature:
\begin{equation}
\tau_{\text{eff}} = \tau_0 \cdot (1 + 2 \cdot m)
\end{equation}
% 其中τ_0是基础温度,m衡量淘汰边界的清晰程度
where $\tau_0$ is the base temperature and $m$ measures the clarity of the elimination margin.


% ============================================================================
% 4.3 模型实现
% ============================================================================

\subsection{Model Implementation}

% 由于后验分布没有解析解,我们使用MCMC采样来近似它
Since the posterior distribution lacks an analytical solution, we approximate it
using Markov Chain Monte Carlo (MCMC) sampling.

% --- 单纯形上的提议分布 ---
\subsubsection{Proposal Distribution on the Simplex}

% 我们使用Dirichlet提议分布,它自然满足单纯形约束:
We use a Dirichlet proposal that naturally respects the simplex constraint:
\begin{equation}
\boldsymbol{\pi}' \sim \text{Dirichlet}\left(\frac{\boldsymbol{\pi}}{\lambda}\right)
\end{equation}
% 其中λ是控制探索幅度的提议步长
where $\lambda$ is the proposal step size controlling exploration magnitude.

% --- 自适应步长 ---
\subsubsection{Adaptive Step Size}

% 步长λ自适应调整以达到约30%的目标接受率
The step size $\lambda$ is adaptively adjusted to maintain an acceptance rate around 30\%.

% --- 超参数设置 ---
\subsubsection{Hyperparameter Settings}

% 默认超参数:τ_0=12, α=1.5, β=0.3, λ_0=0.06
The default hyperparameters are: $\tau_0=12$, $\alpha=1.5$, $\beta=0.3$, $\lambda_0=0.06$,
% 在3000次预热后采集10000个后验样本
with $N=10000$ posterior samples after $B=3000$ burn-in iterations.
% 这些值通过敏感性分析确定,以平衡模型性能和计算效率
These values were
determined through sensitivity analysis to balance model performance and computational
efficiency.
% 固定随机种子以确保可重复性
The random seed is fixed to ensure reproducibility.

% ============================================================================
% 4.4 结果验证
% ============================================================================

\subsection{Results and Validation}

% --- 一致性指标(Q1a)---
\subsubsection{Consistency Metrics (Q1a)}

% 淘汰预测准确率(EPA):预测正确的周数占总周数的比例
\textbf{Elimination Prediction Accuracy (EPA)}:
\begin{equation}
\text{EPA} = \frac{1}{W} \sum_{w=1}^{W} \mathbb{I}[\hat{E}_w = E_w]
\end{equation}

% 平均淘汰概率:模型对实际淘汰结果的平均置信度
\textbf{Mean Elimination Probability}:
\begin{equation}
\bar{P}_{\text{elim}} = \frac{1}{W} \sum_{w=1}^{W} P(E_w | \bar{\boldsymbol{\pi}}_w, \mathbf{S}_w)
\end{equation}

% 其他指标包括Top-2准确率和Kendall's τ秩相关系数
Additional metrics include Top-2 accuracy and Kendall's $\tau$ rank correlation.

% 表格总结了34季的一致性指标
Table~\ref{tab:consistency} summarizes the consistency metrics in all 34 seasons.

\begin{table}[htbp]
\centering
\caption{Consistency Metrics Summary (Q1a)}
% 一致性指标汇总表
\label{tab:consistency}
\begin{tabular}{lc}
\hline
\textbf{Metric} & \textbf{Value} \\
\hline
% 分析的总周数
Total weeks analyzed & 301 \\
% 淘汰预测准确率
Elimination Prediction Accuracy (EPA) & 93.7\% \\
% Top-2准确率(淘汰者在预测的后两名中)
Top-2 Accuracy & 97.7\% \\
% 平均淘汰概率
Mean Elimination Probability & 0.481 \\
% 平均Kendall's τ
Mean Kendall's $\tau$ & 0.717 \\
\hline
% 按计分方法分类:
\multicolumn{2}{l}{\textit{By Scoring Method:}} \\
% 排名法(74周)
\quad Ranking Method (n=74) & EPA = 95.9\% \\
% 百分比法(227周)
\quad Percentage Method (n=227) & EPA = 93.0\% \\
\hline
\end{tabular}
\end{table}

% --- 一致性结果分析 ---
% 结果表明模型与观测淘汰结果一致
The 93.7\% EPA indicates that our estimated vote distributions correctly predict
the actual elimination in most weeks.
% Top-2准确率97.7%:几乎所有情况下,实际被淘汰者都在模型预测的后两名中
The 97.7\% Top-2 accuracy shows that in nearly all cases, the eliminated contestant
was among the bottom two predicted by our model.
% 排名法EPA略高于百分比法,这与排名法的离散约束更明确有关
The Ranking Method achieves slightly higher EPA (95.9\%) than the Percentage Method (93.0\%),
likely due to its sharper discrete constraints.
% Kendall's τ为0.717表明预测排名与投票排名相关性较好
The Kendall's $\tau$ of 0.717 indicates reasonable rank correlation between predicted
and implied vote rankings.

% --- 确定性指标(Q1b)---
\subsubsection{Certainty Metrics (Q1b)}

% 变异系数(CV):标准差与均值的比值,衡量估计的相对不确定性
\textbf{Coefficient of Variation (CV)}:
\begin{equation}
\text{CV}_i = \frac{\sigma_{\pi_i}}{\mu_{\pi_i}}
\end{equation}

% 95%可信区间宽度:后验分布的2.5%和97.5%分位数之差
\textbf{95\% Credible Interval Width}:
\begin{equation}
\text{CI}_i = \pi_i^{97.5\%} - \pi_i^{2.5\%}
\end{equation}

% 我们还报告有效样本量(ESS)和MCMC接受率来评估采样质量
We also report effective sample size (ESS) and MCMC acceptance rates to assess
sampling quality.

% 表格总结了确定性指标
Table~\ref{tab:certainty} summarizes the certainty metrics.

\begin{table}[htbp]
\centering
\caption{Certainty Metrics Summary (Q1b)}
% 确定性指标汇总表
\label{tab:certainty}
\begin{tabular}{lc}
\hline
\textbf{Metric} & \textbf{Value} \\
\hline
% 所有选手的平均CV
Mean CV (all contestants) & 0.710 \\
% 平均95%可信区间宽度
Mean 95\% CI Width & 0.292 \\
\hline
% 按淘汰状态分类:
\multicolumn{2}{l}{\textit{By Elimination Status:}} \\
% 被淘汰选手的CV
\quad Eliminated contestants CV & 0.761 \\
% 未被淘汰选手的CV
\quad Non-eliminated contestants CV & 0.704 \\
% 被淘汰选手的CI宽度
\quad Eliminated contestants CI width & 0.167 \\
% 未被淘汰选手的CI宽度
\quad Non-eliminated contestants CI width & 0.308 \\
\hline
\end{tabular}
\end{table}

% --- 确定性结果分析 ---
% 平均CV为0.710,反映逆问题固有的不确定性
The mean CV of 0.710 reflects the inherent uncertainty of this inverse problem.
% 被淘汰者CI更窄,因为淘汰事件提供了强约束
Eliminated contestants have narrower CI widths (0.167 vs. 0.308) because
the elimination event constrains their vote share more tightly.


% ----------------------------------------------------------------------------
% 第5章:Task 2 - 计分方法比较
% ----------------------------------------------------------------------------
\section{Task 2: Scoring Method Comparison}

\subsection{Method Characteristics}

% 两种计分方法在如何结合评委分数和观众投票方面存在根本差异:
The two scoring methods differ fundamentally in how they combine judges' scores
and audience votes:

\begin{itemize}
    % 排名法:将分数和投票转换为名次后相加,消除数值大小信息
    \item \textbf{Ranking Method}: Converts scores and votes to ordinal ranks before
    combining, discarding numerical magnitude.
    % 百分比法:保留数值信息,直接按百分比相加
    \item \textbf{Percentage Method}: Preserves numerical information by combining
    score percentages directly.
\end{itemize}

% 这导致三个关键差异:
% (1) 排名法权重固定50%-50%,百分比法权重动态变化
% (2) 排名法仅在名次变化时响应,百分比法连续响应
% (3) 排名法对极端值具有鲁棒性
These lead to three key differences:
(1) The Ranking Method has fixed 50\%--50\% weight structure, while the Percentage
Method has dynamic effective weights;
(2) The Ranking Method responds only when ranks change, while the Percentage
Method responds continuously;
(3) The Ranking Method is robust to outliers.

\subsection{Metrics}

% 淘汰翻转率(EFR):切换计分方法时淘汰决定改变的周数比例
\textbf{Elimination Flip Rate (EFR)}: The fraction of weeks where switching methods
would change the elimination decision:
\begin{equation}
    \text{EFR} = \frac{1}{W}\sum_{w=1}^{W} \mathbf{1}[E^{\text{actual}}_w \neq E^{\text{cf}}_w]
\end{equation}

where $E_w^{actual}$ is the historically observed elimination and $E_w^{cf}$ is the counterfactual outcome under the alternative aggregation scheme.

% 有效权重:各成分对综合得分的方差贡献比例
\textbf{Effective Weight}: The variance contribution to the combined score:
\begin{equation}
    \alpha^{\text{pct}} = \frac{\text{Var}(P^{(S)})}{\text{Var}(P^{(S)}) + \text{Var}(P^{(V)})}
\end{equation}
% 其中P^(S)和P^(V)分别是评委分数百分比和投票百分比

where $P^{(S)}$ and $P^{(V)}$ are the judge scores percentages and fan votes percentages.

\subsection{Results}

% 使用Task 1的投票估计,对34季301周数据进行反事实分析
Using the vote estimates from Task~1, we conduct counterfactual analysis on
301 weeks across 34 seasons.

% 反事实分析汇总表
\begin{table}[H]
\centering
\caption{Counterfactual Analysis Summary}
\label{tab:counterfactual-summary}
\begin{tabular}{lccc}
\hline
\textbf{Metric} & \textbf{Overall} & \textbf{Ranking Era} & \textbf{Percentage Era} \\
\hline
% 分析周数
Weeks analyzed & 301 & 74 & 227 \\
% 淘汰翻转次数
Elimination flips & 66 & 0 & 66 \\
% 翻转率
EFR & 21.9\% & 0.0\% & 29.1\% \\
\hline
\end{tabular}
\end{table}

% 有效权重分析表
\begin{table}[htbp]
\centering
\caption{Effective Weight Analysis}
\label{tab:weight-decomposition}
\begin{tabular}{lcc}
\hline
\textbf{Metric} & \textbf{Ranking Method} & \textbf{Percentage Method} \\
\hline
% 评委有效权重
Judge effective weight & 48.2\% & 22.9\% \\
% 观众有效权重
Audience effective weight & 51.8\% & 77.1\% \\
\hline
\end{tabular}
\end{table}

% 有效权重分析表明:百分比法给予观众77.1%的有效权重,而排名法为51.8%
The effective weight analysis reveals that the Percentage Method assigns 77.1\%
effective weight to fan votes, compared to 51.8\% under the Ranking Method.
% 原因:投票分布通常比评委分数更分散,导致投票主导百分比综合得分
This occurs because fan votes distributions are typically more dispersed than judge
scores, causing fan votes to dominate the combined percentage score.

% 结论:百分比法比排名法更偏向粉丝投票
% 这解释了第3-27季的争议:低评委分数但高粉丝支持的选手能够晋级
% 节目在第28季回归排名法,旨在恢复评委与观众之间的平衡

Synthesizing the evidence from our counterfactual analysis and effective weight metrics, we concluded that the Percentage Method favors fan votes more than the Ranking
Method. This explains the controversies during Seasons 3--27, where contestants
with low judge scores but strong fan support could advance.

% The show's return to the Ranking Method in Season 28 aimed to restore balance. 
   
% ----------------------------------------------------------------------------
% 第6章:Task 3 - 争议案例分析与方法推荐
% ----------------------------------------------------------------------------
\section{Task 3: Controversy Analysis and Method Recommendation}


% ============================================================================
% 6.1 争议案例概述
% ============================================================================

\subsection{Overview of Controversial Cases}

% 在《与星共舞》34季历史中,某些选手因评委评分与粉丝投票之间的巨大差异而引发重大争议。
% 我们分析四个典型案例,它们体现了技术评估与大众支持之间的张力。
Throughout the 34-season history of \textit{Dancing with the Stars}, certain contestants
have sparked significant controversy due to stark discrepancies between judges' assessments
and fan voting patterns. We examine four prominent cases that exemplify the tension between
technical evaluation and popular support.

% 争议案例汇总表
\begin{table}[htbp]
\centering
\caption{Summary of Controversial Cases}
\label{tab:controversy-cases}
\begin{tabular}{lccccc}
\hline
\textbf{Contestant} & \textbf{Season} & \textbf{Method} & \textbf{Placement} & \textbf{Judge Last} & \textbf{Saved by Fans} \\
\hline
Jerry Rice & 2 & Ranking & 2nd & 1 & 1 \\
Billy Ray Cyrus & 4 & Percentage & 5th & 3 & 2 \\
Bristol Palin & 11 & Percentage & 3rd & 4 & 4 \\
Bobby Bones & 27 & Percentage & \textbf{1st} & 1 & 1 \\
\hline
\end{tabular}
\end{table}

% 我们使用 Task 1-2 中建立的符号定义以下争议度指标:
We define the following controversy metrics using the notation established in Tasks 1--2:
\begin{itemize}
    % 评委最低次数:被评委评为最低分的周数
    \item \textbf{Times Judge Last}: Number of weeks where $R_i^S = n$ (ranked last by judges)
    % 被粉丝救回次数:评委最低但未被淘汰的周数
    \item \textbf{Times Saved by Fans}: Weeks where $R_i^S = n$ but $E \neq i$
    % 争议强度:平均评委排名与平均投票排名之差(正值表示粉丝比评委更喜欢该选手)
    \item \textbf{Controversy Intensity}: $\overline{R_i^S} - \overline{R_i^\pi}$
    (positive values indicate fans favor the contestant more than judges)
\end{itemize}

% 争议度指标详细表
\begin{table}[htbp]
\centering
\caption{Controversy Metrics by Contestant}
\label{tab:controversy-metrics}
\begin{tabular}{lcccccc}
\hline
\textbf{Contestant} & \textbf{Weeks} & \textbf{Avg $R^S$} & \textbf{Avg $R^\pi$} & \textbf{Intensity} & \textbf{Best $R^S$} & \textbf{Worst $R^S$} \\
\hline
Jerry Rice & 7 & 4.93 & 3.71 & 1.21 & 4 & 7 \\
Billy Ray Cyrus & 8 & 6.88 & 4.00 & 2.88 & 5 & 11 \\
Bristol Palin & 9 & 6.33 & 5.89 & 0.44 & 3 & 9 \\
Bobby Bones & 8 & 7.62 & 7.25 & 0.38 & 6 & 10 \\
\hline
\end{tabular}
\end{table}

% Billy Ray Cyrus 的争议强度最高(2.88),表明评委与粉丝排名差距最大。
% 值得注意的是,尽管 Bobby Bones 争议强度较低(0.38),但他以持续低评委分夺冠引发了最多公众讨论。
The controversy intensity is highest for Billy Ray Cyrus (2.88), indicating the largest gap
between judge and fan rankings. Notably, while Bobby Bones has a modest controversy intensity
(0.38), his \textbf{winning the championship} despite consistently low judge scores generated
the most public discourse.

% ============================================================================
% 6.2 逐案分析
% ============================================================================

\subsection{Case-by-Case Analysis}

% --- Jerry Rice (Season 2) ---
\subsubsection{Jerry Rice (Season 2)}

% Jerry Rice,NFL传奇球星,在排名法下参赛并获得亚军,尽管平均评委排名 $R^S = 4.93$(位于后半段)。
% 在百分比法下,他的淘汰顺序将保持不变(0周改变)。
% 他两次进入底部两人;评委机制会淘汰他一次,但那周他本来就被淘汰了。
Jerry Rice, the NFL legend, competed under the Ranking Method and finished as runner-up
despite averaging $R^S = 4.93$ (bottom half). Under the Percentage Method, his elimination
sequence would remain unchanged (0 weeks altered). He appeared in the bottom-two twice;
the judge mechanism would have eliminated him once, but only in the week he was actually
eliminated anyway.

% --- Billy Ray Cyrus (Season 4) ---
\subsubsection{Billy Ray Cyrus (Season 4)}

% Billy Ray Cyrus,乡村音乐巨星,在百分比法下参赛并获得第5名,争议强度(2.88)在四个案例中最高。
% 在排名法下,8周中有4周会产生不同的淘汰者,但他仍会在第8周被淘汰。
% 他3次进入底部两人;评委机制会在所有3次中淘汰他,大幅缩短他的比赛历程。
Billy Ray Cyrus, the country music star, competed under the Percentage Method and finished 5th
with the highest controversy intensity (2.88) among the four cases. Under the Ranking Method,
4 of 8 weeks would have different eliminees, but he would still exit in Week 8. He appeared in
the bottom-two 3 times; the judge mechanism would have eliminated him in all 3 instances,
significantly shortening his run.

% --- Bristol Palin (Season 11) ---
\subsubsection{Bristol Palin (Season 11)}

% Bristol Palin,前州长 Sarah Palin 之女,在百分比法下参赛,尽管4次被评委评为最低分仍获得第3名。
% 在排名法下,她会在第5周被淘汰(而非 Florence Henderson),最终第8名而非第3名。
% 她在9周中有7周进入底部两人;评委机制会在其中4次淘汰她——这是所有案例中影响最严重的。
Bristol Palin, daughter of former Governor Sarah Palin, competed under the Percentage Method
and finished 3rd despite ranking last among judges 4 times. Under the Ranking Method, she would
have been eliminated in Week 5 instead of Florence Henderson, finishing 8th rather than 3rd.
She appeared in the bottom-two 7 out of 9 weeks; the judge mechanism would have eliminated her
in 4 of those instances---the most severe impact among all cases.

% --- Bobby Bones (Season 27) ---
\subsubsection{Bobby Bones (Season 27)}

% Bobby Bones,乡村音乐电台主持人,在百分比法下赢得第27季冠军,尽管平均评委排名 $R^S = 7.62$(四个案例中最差)。
% 在排名法下,他会在第5周或第8周被淘汰,无法夺冠。
% 他3次进入底部两人;评委机制会淘汰他一次(第8周)。节目在第28季恢复使用排名法。
Bobby Bones, a country music radio host, won Season 27 under the Percentage Method despite
averaging $R^S = 7.62$ (the worst among our cases). Under the Ranking Method, he would have
been eliminated in Week 5 or Week 8, making his championship impossible. He appeared in the
bottom-two 3 times; the judge mechanism would have eliminated him once (Week 8). The show
returned to the Ranking Method in Season 28.

% ============================================================================
% 6.3 反事实分析汇总
% ============================================================================

\subsection{Counterfactual Summary}

% 反事实分析汇总表:各选手在替代方法下的结果变化
\begin{table}[htbp]
\centering
\small
\caption{Counterfactual Analysis: Would Switching Methods Change Outcomes?}
\label{tab:counterfactual-controversy}
\begin{tabular}{lcccc}
\hline
\textbf{Contestant} & \textbf{Actual Method} & \textbf{CF Elim.} & \textbf{Weeks Changed} & \textbf{Impact} \\
\hline
Jerry Rice & Ranking & 0 & 0 & None \\
Billy Ray Cyrus & Percentage & 1 (wk 8) & 4 & Same elim. week \\
Bristol Palin & Percentage & 1 (wk 5) & 2 & Elim. 4 wks earlier \\
Bobby Bones & Percentage & 2 (wk 5, 8) & 3 & \textbf{Would not win} \\
\hline
\end{tabular}
\end{table}

% 回答核心问题:选择不同的计分方法是否会导致相同结果?
\textbf{Would the choice of method have led to the same result?}
\begin{itemize}
    % Jerry Rice:是的,两种方法会产生相同的淘汰顺序。
    \item \textbf{Jerry Rice}: Yes. Both methods would yield the same elimination sequence.
    % Billy Ray Cyrus:部分相同。他仍会在第8周被淘汰,但过程不同。
    \item \textbf{Billy Ray Cyrus}: Partially. He would still be eliminated in Week 8, but
    the path there would differ.
    % Bristol Palin:不同。在排名法下她会在第5周被淘汰,最终第8名而非第3名。
    \item \textbf{Bristol Palin}: No. She would have been eliminated in Week 5 under the
    Ranking Method, finishing 8th instead of 3rd.
    % Bobby Bones:不同。在排名法下他会在第5周或第8周被淘汰,无法夺冠。
    \item \textbf{Bobby Bones}: No. He would have been eliminated in Week 5 or Week 8 under
    the Ranking Method, making his championship victory impossible.
\end{itemize}

% ============================================================================
% 6.4 评委淘汰机制的影响
% ============================================================================

\subsection{Impact of Judge Elimination Mechanism}

% 从第28季开始,DWTS 引入评委淘汰机制:在综合分数和投票后,评委从底部两人中选择淘汰一人。
% 我们模拟该机制对争议案例的影响。
Starting from Season 28, DWTS introduced a judge elimination mechanism where, after
combining scores and votes, judges choose which of the bottom-two contestants to eliminate.
We simulate this mechanism's impact on our controversial cases.

% 假设:当面对两位选手时,评委会淘汰当周评委分数较低的那位。
\textbf{Assumption}: When presented with two contestants, judges would eliminate the one
with the lower judge score that week.

% 评委淘汰机制影响汇总表
\begin{table}[htbp]
\centering
\caption{Judge Elimination Mechanism Impact}
\label{tab:judge-mechanism}
\begin{tabular}{lccc}
\hline
\textbf{Contestant} & \textbf{In Bottom-Two} & \textbf{Would Be Eliminated} & \textbf{Impact Assessment} \\
\hline
Jerry Rice & 2 times & 1 time & Moderate \\
Billy Ray Cyrus & 3 times & 3 times & \textbf{Severe} \\
Bristol Palin & 7 times & 4 times & \textbf{Severe} \\
Bobby Bones & 3 times & 1 time & Moderate \\
\hline
\end{tabular}
\end{table}

% Billy Ray Cyrus 和 Bristol Palin 会受到严重影响:Cyrus 在3次底部两人中都会被淘汰,Palin 在7次中会被淘汰4次。
% Bobby Bones 和 Jerry Rice 受到中等影响(各被淘汰一次)。
% 该机制作为有效的制衡措施,在不压制粉丝参与的情况下恢复专家影响力。
Billy Ray Cyrus and Bristol Palin would have been substantially affected: Cyrus would have been
eliminated in all 3 bottom-two appearances, while Palin would have been eliminated in 4 of 7.
Bobby Bones and Jerry Rice face moderate impact (eliminated once each). The mechanism serves
as an effective balancing measure that restores expert influence without overriding fan engagement.

% ============================================================================
% 6.5 综合建议
% ============================================================================

\subsection{Recommendations and Synthesis}

% 基于争议分析,我们推荐排名法结合评委淘汰机制。
Based on the controversy analysis, we recommend the \textbf{Ranking Method} combined with
the \textbf{Judge Elimination Mechanism}.

% 计分方法公平性对比表
\begin{table}[htbp]
\centering
\caption{Comparison of Scoring Methods for Fairness}
\label{tab:method-fairness}
\begin{tabular}{lcc}
\hline
\textbf{Criterion} & \textbf{Ranking Method} & \textbf{Percentage Method} \\
\hline
Weight Stability & Fixed 50\%--50\% & Dynamic (22.9\%--77.1\%) \\
Predictability & High & Low \\
Expert Influence & Protected & Can be overridden \\
Major Controversies & 0 (S1--2, S28--34) & 3+ (S3--27) \\
Transparency & Simple rules & Complex score-dependent \\
\hline
\end{tabular}
\end{table}

% 排名法保证评委与粉丝之间的平等影响力(50%-50%),不受分数分布影响。
% 使用该方法的季度(1-2和28-34)没有发生重大争议,而百分比法时代(3-27)产生了多个争议结果,包括 Bobby Bones 的夺冠。
The Ranking Method guarantees equal influence (50\%--50\%) between judges and fans,
regardless of score distributions. Seasons using this method (1--2 and 28--34) experienced
no major controversies, while the Percentage Method era (3--27) produced multiple disputed
outcomes including Bobby Bones's championship victory.

% 第28季引入的评委淘汰机制提供额外保障:当选手进入底部两人时,由评委做最终决定。
% 这会对4个争议案例中的2个(Billy Ray Cyrus 和 Bristol Palin)产生重大影响,同时保留粉丝对大多数淘汰的影响力。
% 这一组合(已在第28-34季采用)解决了导致争议结果的结构性问题,同时保留粉丝参与。
The Judge Elimination Mechanism, introduced in Season 28, provides an additional safeguard:
when contestants are placed in the bottom two, judges make the final decision. This would
have substantially impacted 2 of 4 controversy cases (Billy Ray Cyrus and Bristol Palin)
while maintaining fan influence for the majority of eliminations. This combination,
already adopted in Seasons 28--34, addresses the structural issues that enabled
controversial outcomes while preserving fan participation.

\end{document}
