% 美赛模板:正文部分

\documentclass[12pt]{article}  % 官方要求字号不小于 12 号,此处选择 12 号字体
\usepackage[table]{xcolor}
% \linespread{1.1}
% \bibliographystyle{plain}
% 本模板不需要填写年份,以当前电脑时间自动生成
% 请在以下的方括号中填写队伍控制号
\usepackage[2622678]{easymcm}  % 载入 EasyMCM 模板文件
\problem{C}  % 请在此处填写题号
% \usepackage{mathptmx}  % 这是 Times 字体,中规中矩 
\usepackage{palatino}  % mathpazo 这palatino是 COMAP 官方杂志采用的更好看的 Palatino 字体,可替代以上的 mathptmx 宏包
\usepackage{pdfpages}
\usepackage{longtable}
\usepackage{tabu}
\usepackage{threeparttable}
\usepackage{listings}
\usepackage{paralist}
\usepackage{setspace}
\usepackage{float} %设置图片浮动位置的宏包
\usepackage{graphicx} %插入图片的宏包
\usepackage{subfigure} %插入多图时用子图显示的宏包
\usepackage[normalem]{ulem}
\useunder{\uline}{\ul}{}



 \let\itemize\compactitem
 \let\enditemize\endcompactitem
% \let\enumerate\compactenum
% \let\endenumerate\endcompactenum
% \let\description\compactdesc
% \let\enddescription\endcompactdesc
% \usepackage{biblatex} 
% \usepackage{cite}
% \usepackage{natbib}
\newcommand{\upcite}[1]{\textsuperscript{\textsuperscript{\cite{#1}}}}
\title{One Space, One Future}  % 标题

% 如需要修改题头(默认为 MCM/ICM),请使用以下命令(此处修改为 MCM)
%\renewcommand{\contest}{MCM}

% 文档开始
\begin{document}
\begin{abstract}
	
	% (Need to be reviewed)
With the development of space technology, outer space resources are gradually being exploited, and the asteroid mining industry is expected to have a high return on investment in the future. The United Nations has developed a series of treaties to maintain the common interests of humanity and global equity in outer space. But with the development of the asteroid mining industry, these fair international conventions might be broken.

We clarify \textbf{global equity} as "Whether or not a country has the ability to participate in asteroid mining, it can gain some benefit from asteroid mining". Then, we evaluated 25 countries with better space science and technology through the \textbf{TOPSIS method} in four aspects: \textbf{\emph{Economic Level (EL), Energy Output (EO), Sense of Responsibility (SR) and Scientific and Technological Level (STL)}}, and the weights of subordinate indicators in each aspect is determined by \textbf{AHP-EWM method}. Using AHP to determine the weights of four superior indicators and calculate the weighted average of their scores, define the results as "the \textbf{Fair Possibility Coefficient (FPC)} of each country", the larger the coefficient is, the greater chance a country has to gain profit from asteroid mining. Finally, \textbf{\emph{the US, Russia and China}} ranked in the top 3. Referring to the \textbf{Gini Coefficient (GC)} in Economics, we plotted the \textbf{Lorenz curve} of the distribution of FPC by country, and get \textbf{\emph{GC$ = $0.2277}}, which is less than 0.3\upcite{6}. So asteroid mining is globally fair at present, we called this the \textbf{FPC-GC Model}.

After studying the current international situation, we believe that the asteroid mining industry will face the danger of a few countries monopolizing the entire industry in the future. To prove our idea, we used the \textbf{Hierarchical Cluster method} to classify countries into two types, and establish the \textbf{Future Impact Forecast Model (FIF) based on interspecies competition differential equations} to analyze the changes of the comprehensive competitiveness of these two types of countries in the next 50 years. The result shows that the competitiveness of the first class countries will increase continuously. While the second class countries will develop in the opposite trend, therefore, the asteroid mining industry will be \textbf{\emph{completely monopolized}} by the first class countries. Referring to their final competitiveness, we calculated that \textbf{\emph{GC$ = $0.4423}} after 50 years, when the global equity will be seriously damaged.

By \textbf{changing the parameters} of the FIF model, we explore how global equity is affected by parameters after influencing conditions changed. So we suggest powerful countries to sacrifice some of their interests to help other countries whichh will benefit the common development of the world and the global equity of asteroid mining.

Finally, we put forward suggestions on the implementation of the policy from the two aspects of "amending existing treaties" and "making new policies", then we draw a \textbf{\emph{target timeline}}, hoping to basically achieve global equity by around 2070.











	
	
	% Here is the abstract of your paper.
	
	% Firstly, that is ...
	
	% Secondly, that is ...
	
	% Finally, that is ...
	
	% 美赛论文中无需注明关键字。若您一定要使用,
	% 请将以下两行的注释号 '%' 去除,以使其生效
	\vspace{5pt}
	\textbf{Keywords}: Asteroid mining; Global equity; Lorenz curve; Gini Coefficient; Cluster analysis; Differential equations; Policy recommendations
	
\end{abstract}
    
\maketitle  % 生成 Summary Sheet

\tableofcontents


% 正文开始
% Chapter 1: Introduction
\section{Introduction}
\subsection{Background}
Reality television is a global phenomenon that blends competitive tensions with mass entertainment. As the American version of this international franchise, Dancing with the Stars (DWTS) has gained immense popularity by pairing celebrities with professional dancers to compete for the championship. Over 34 seasons, the competition has paired celebrities with professional dancers, using a scoring system that combines technical scores from expert judges with popularity votes from fans.

The method of combining these evaluations—alternating between "rank-based" and "percentage-based" systems—remains a subject of significant controversy regarding fairness. Producers face the ongoing challenge of balancing competitive integrity with viewer engagement. This study utilizes mathematical modeling to evaluate these scoring mechanisms and propose a more balanced framework for future seasons.

\subsection{Our Work}
In order to examine the impact of asteroid mining on global equity both now and in the future, and what factors and impacts are relevant. We need to address many questions: Including but not limited to how to define and measure global equity? What will asteroid mining look like in the future given the current world situation? How will these changes affect global equity? What factors will influence the changes of asteroid mining in the future? Once these factors changed, how will the asteroid mining situation be? What kind of treaty should we conclude to make asteroid mining beneficial to all of humanity?

Our process for solving these problems is shown below:
\begin{figure}[H]
	\centering
	\includegraphics[width=.7\textwidth]{obian.png}
	\caption{Our work}
	\label{img}
\end{figure}
            
% \subsection{Literature Review}
% A literatrue\cite{1} say something about this problem ...

% \subsection{Our work}
% We do such things ...

% \begin{enumerate}[\bfseries 1.]
%     \item We do ...
%     \item We do ...
%     \item We do ...
% \end{enumerate}

% Chapter 2:
\section{Assumptions and Notations}

\begin{figure}[H]
	\centering
	\includegraphics[width = 1.0\textwidth]{img/t1_uncertainty.png}
	\caption{Certainty Analysis: Vote Share vs. Model Uncertainty}
	\label{img}
\end{figure}
\vspace{-0.6cm}

\subsection{Assumptions and Justifications}
In order to achieve a global and equitable analysis of asteroid mining, we make the following assumptions in a comprehensive consideration to better simplify and understand the problem:
\begin{itemize}
	\item The growth of the comprehensive competitiveness of each country is continuous and slow, without any sudden changes.\\
	\textbf{Justification:} A country's overall competitiveness, which is influenced by many factors, is unlikely to change abruptly.
	\item The data we collect is accurate.\\
	\textbf{Justification:} The data we collect are from authoritative organizations or websites such as the World Bank, so we can assume that they are true and accurate.
	\item The country is the smallest unit of analysis.\\
	\textbf{Justification:} It is impossible to achieve equality for all persons when considering asteroid mining, which would make the problem more complex and impractical, so in order to convenient our analysis, we take the country as the smallest unit and consider global equity.
	\item All countries will respond to the treaties.\\
	\textbf{Justification:} In order to analyze the impact of changing realistic conditions on global equity, we must assume that all countries will respond to the treaties concluded so that the parameters will change and our study will be worthwhile.
\end{itemize} 
\subsection{Notations}
Here we clarify some of the symbols and their definitions, others will be explained in the paper.
\vspace{-0.4cm}
\begin{table}[H]
	\centering
	\caption{Notations}
	\begin{tabular}{cc}
		\toprule[1.5pt]
		\textbf{Symbol} & \textbf{Description} \\ 
		\midrule
		$ \omega _{Aj} $ & The weight of the $ j^{th} $ indicator determined by AHP \\
		$ \omega _{Ej} $ & The weight of the $ j^{th} $ indicator determined by EWM \\ 
		$ FPC_i $ & FPC value of the $ i^{th} $ country \\
		$ GC $ & Gini Coefficient \\
		$ c $ & Comprehensive competitiveness \\
		$ \alpha $ & Relative degree of dependence \\
		$ \beta $ & Comprehensive development factor \\
		\bottomrule[1.5pt]
	\end{tabular}
\end{table}
\vspace{-0.6cm}


% ----------------------------------------------------------------------------
% 第4章:Task 1 - 投票预估模型
% ----------------------------------------------------------------------------
%
% ============================================================================
% 模型名称: Feature-Enhanced Adaptive Bayesian Vote Estimation (FABVE)
%           特征增强自适应贝叶斯投票估计模型
% ============================================================================
%
% 核心挑战:观众投票数据从未公开,只能观察评委分数和淘汰结果
% 建模思路:贝叶斯推断 + 软约束 + 特征增强先验 + 自适应MCMC
%
% ============================================================================

\section{Task 1: Vote Estimation Model}

% ============================================================================
% 4.1 问题形式化
% ============================================================================

\subsection{Problem Formulation}

% 在《与星共舞》节目中,每周的淘汰由评委分数和观众投票共同决定。
In \textit{Dancing with the Stars}, weekly eliminations are determined by a combination
of judges' scores and audience votes.
% 然而,在节目34季的历史中,实际的观众投票数据从未被公开披露。
However, the show has never disclosed the actual
audience voting data throughout its 34-season history.
% 这构成了一个经典的逆问题:我们必须从可观测的淘汰结果反推未观测到的投票分布。
This constitutes a classic
\textbf{inverse problem}: we must infer the unobserved voting distribution from
the observable elimination outcomes.

% 形式化地,对于每季s的每周w,我们观察到:
Formally, for each week $w$ in season $s$, we observe:
\begin{itemize}
    % 评委分数向量S,其中每个分数在4到40分之间
    \item Judge scores $\mathbf{S} = (S_1, S_2, \ldots, S_n)$, where $S_i \in [4, 40]$
    % 淘汰结果E,即被淘汰选手的索引
    \item Elimination result $E \in \{1, 2, \ldots, n\}$ (the index of the eliminated contestant)
\end{itemize}

% 我们的目标是估计未观测到的投票份额分布π
Our goal is to estimate the unobserved vote share distribution
$\boldsymbol{\pi} = (\pi_1, \ldots, \pi_n)$,
% 其中π_i表示选手i获得的观众投票比例
where $\pi_i$ represents the fraction
of audience votes received by contestant $i$, satisfying:
% 约束条件:所有份额之和为1,且每个份额非负
\begin{equation}
\sum_{i=1}^{n} \pi_i = 1, \quad \pi_i \geq 0
\end{equation}

% 淘汰由节目的计分规则决定:
The elimination is governed by the show's scoring rules:

% 排名法(第1-2季,第28-34季):
\textbf{Ranking Method} (Seasons 1--2, 28--34):
% 综合排名 = 评委排名 + 投票排名,排名数最高者被淘汰
\begin{equation}
C_i^{\text{rank}} = R_i^{S} + R_i^{\pi}, \quad E = \arg\max_i C_i^{\text{rank}}
\end{equation}
% 其中R_i^S和R_i^π分别是选手i按评委分数和粉丝投票的排名
where $R_i^{S}$ and $R_i^{\pi}$ represent the contestant's ranks by judge scores and fan votes, respectively.

% 百分比法(第3-27季):
\textbf{Percentage Method} (Seasons 3--27):
% 综合得分 = 评委分数百分比 + 粉丝投票百分比,得分最低者被淘汰
\begin{equation}
C_i^{\text{pct}} = \frac{S_i}{\sum_j S_j} + \pi_i, \quad E = \arg\min_i C_i^{\text{pct}}
\end{equation}

% 从数学上看,给定淘汰结果E,存在无穷多个可能产生该结果的投票分布。

Mathematically, given an elimination outcome $E$, there exist infinitely many fan votes distributions that satisfy this result.
% 我们的任务是通过融入先验知识来识别最合理的分布,同时恰当地量化固有的不确定性。
Our task is to identify the most plausible distribution
by incorporating prior knowledge while properly quantifying the inherent uncertainty.

% ============================================================================
% 4.2 方法论
% ============================================================================

\subsection{Methodology: Bayesian Vote Estimation}

% 我们采用贝叶斯推断框架来解决这个逆问题。
We adopt a \textbf{Bayesian inference framework} to address this inverse problem.
% 该框架提供三个关键优势:
This framework offers three key advantages:
\begin{enumerate}
    % 1. 不确定性量化:提供完整的后验分布而非点估计
    \item \textbf{Uncertainty quantification}: Provides complete posterior distributions rather than point estimates
    % 2. 先验知识融合:融入关于投票行为的领域知识
    \item \textbf{Prior knowledge integration}: Incorporates domain knowledge about voting behavior
    % 3. 软约束:以概率方式使用淘汰信息,而非硬约束
    \item \textbf{Soft constraints}: Uses elimination information probabilistically rather than as hard constraints
\end{enumerate}

% 后验分布由贝叶斯定理给出:
The posterior distribution is given by Bayes' theorem:
% 后验 ∝ 似然 × 先验
\begin{equation}
P(\boldsymbol{\pi} | E, \mathbf{S}) \propto
  P(E | \boldsymbol{\pi}, \mathbf{S}) \cdot P(\boldsymbol{\pi} | \mathbf{S}, \mathbf{X})
\end{equation}
% 其中X表示选手特征(职业、年龄等)
where $\mathbf{X}$ denotes contestant features (profession, age, etc.).

% --- 先验分布 ---
\subsubsection{Prior Distribution}

% 我们使用Dirichlet分布作为先验:
We use a Dirichlet distribution as the prior:
\begin{equation}
\boldsymbol{\pi} \sim \text{Dirichlet}(\alpha \cdot \boldsymbol{\theta})
\end{equation}

% 先验权重向量θ设计为加权混合:
The prior weight vector $\boldsymbol{\theta}$ is designed as a weighted mixture:
\begin{equation}
\theta_i = (1 - \beta) \cdot f_i^{\text{feature}} + \beta \cdot \tilde{S}_i
\end{equation}
% 其中S̃_i是归一化的评委分数
where $\tilde{S}_i = S_i / \sum_j S_j$ is the normalized judge scores.

% 基于特征的权重:f_i^feature项融入职业人气权重和年龄效应
\textbf{Feature-based weights}: The term $f_i^{\text{feature}}$ incorporates profession
popularity weights (e.g., singers and actors typically have larger fan bases) and age effects.
% 这些权重基于领域知识:某些职业天然拥有更大的粉丝群体
These weights are based on domain knowledge that certain professions inherently command
larger followings.
% 重要的是,敏感性分析表明模型结果对这些权重设置的变化具有鲁棒性
Importantly, our sensitivity analysis (Section~\ref{sec:sensitivity})
demonstrates that model results are robust to variations in these weight settings,
% 因为主要信息来自淘汰约束而非先验
as the primary information comes from elimination constraints rather than priors.

% 年龄效应:对25-35岁年龄段有轻微偏好(社交媒体参与度更高)
\textbf{Age effect}:
\begin{equation}
\text{age\_factor}_i = 1.0 - 0.01 \times |a_i - 30|
\end{equation}
% 限制在[0.7, 1.2]范围内
capped within $[0.7, 1.2]$.

% --- 似然函数 ---
\subsubsection{Likelihood Function}

% 我们采用softmax软约束而非硬约束:
We employ a softmax soft constraint instead of hard constraints:

% 百分比法的似然:
\textbf{For Percentage Method}:
% 综合得分低的选手被淘汰概率更高(指数形式)
\begin{equation}
P(E = k | \boldsymbol{\pi}, \mathbf{S}) =
  \frac{\exp(-\tau \cdot C_k^{\text{pct}})}{\sum_{j} \exp(-\tau \cdot C_j^{\text{pct}})}
\end{equation}

% 排名法的似然:
\textbf{For Ranking Method}:
% 综合排名高的选手被淘汰概率更高
\begin{equation}
P(E = k | \boldsymbol{\pi}, \mathbf{S}) =
  \frac{\exp(\tau \cdot C_k^{\text{rank}})}{\sum_{j} \exp(\tau \cdot C_j^{\text{rank}})}
\end{equation}

% 温度参数τ控制约束的严格程度
The temperature parameter $\tau$ controls constraint strictness.
% 我们使用动态温度:
We use dynamic temperature:
\begin{equation}
\tau_{\text{eff}} = \tau_0 \cdot (1 + 2 \cdot m)
\end{equation}
% 其中m衡量淘汰边界的清晰程度
where $m$ measures the clarity of the elimination margin.

% --- 为什么使用软约束而非硬约束 ---
% 解释软约束的设计动机
\textbf{Why soft constraints?}
% 硬约束方法只接受满足淘汰条件的样本,会导致循环论证
A hard constraint approach---accepting only samples where the eliminated contestant
has the lowest combined score---leads to circular reasoning:
% 模型会达到100%的"预测准确率",但这只是因为我们强制满足了约束
the model would achieve 100\% ``prediction accuracy'' simply because we forced
the constraint to be satisfied.
% 这不提供真正的预测洞察
This provides no genuine predictive insight.

% 软约束通过概率方式使用淘汰信息,允许模型学习投票分布的结构
Our soft constraint formulation uses elimination information probabilistically,
% 而非机械地强制满足
allowing the model to learn the structure of vote distributions
rather than mechanically enforcing outcomes.
% 结果是更诚实的不确定性估计和更有意义的后验分布
The result is more honest uncertainty estimates and more meaningful posterior distributions.

% ============================================================================
% 4.3 模型实现
% ============================================================================

\subsection{Model Implementation}

% 由于后验分布没有解析解,我们使用MCMC采样来近似它
Since the posterior distribution lacks an analytical solution, we approximate it
using Markov Chain Monte Carlo (MCMC) sampling.

% --- 单纯形上的提议分布 ---
\subsubsection{Proposal Distribution on the Simplex}

% 我们使用Dirichlet提议分布,它自然满足单纯形约束:
We use a Dirichlet proposal that naturally respects the simplex constraint:
\begin{equation}
\boldsymbol{\pi}' \sim \text{Dirichlet}\left(\frac{\boldsymbol{\pi}}{\lambda}\right)
\end{equation}

% --- 自适应步长 ---
\subsubsection{Adaptive Step Size}

% 步长λ自适应调整以达到约30%的目标接受率:
The step size $\lambda$ is adaptively adjusted to achieve a target acceptance rate of 30\%:
\begin{equation}
\lambda^{(t+1)} = \begin{cases}
  0.95 \cdot \lambda^{(t)} & r_{\text{recent}} < 0.25 \\
  1.05 \cdot \lambda^{(t)} & r_{\text{recent}} > 0.35 \\
  \lambda^{(t)} & \text{otherwise}
\end{cases}
\end{equation}

% --- 超参数设置 ---
\subsubsection{Hyperparameter Settings}

% 默认超参数:τ_0=12, α=1.5, β=0.3, λ_0=0.06
The default hyperparameters are: $\tau_0=12$, $\alpha=1.5$, $\beta=0.3$, $\lambda_0=0.06$,
% 在3000次预热后采集10000个后验样本
with $N=10000$ posterior samples after $B=3000$ burn-in iterations.
% 这些值通过敏感性分析确定,以平衡模型性能和计算效率
These values were
determined through sensitivity analysis to balance model performance and computational
efficiency.
% 固定随机种子以确保可重复性
The random seed is fixed to ensure reproducibility.

% ============================================================================
% 4.4 结果验证
% ============================================================================

\subsection{Results and Validation}

% --- 一致性指标(Q1a)---
\subsubsection{Consistency Metrics (Q1a)}

% 淘汰预测准确率(EPA):预测正确的周数占总周数的比例
\textbf{Elimination Prediction Accuracy (EPA)}:
\begin{equation}
\text{EPA} = \frac{1}{W} \sum_{w=1}^{W} \mathbb{I}[\hat{E}_w = E_w]
\end{equation}

% 平均淘汰概率:模型对实际淘汰结果的平均置信度
\textbf{Mean Elimination Probability}:
\begin{equation}
\bar{P}_{\text{elim}} = \frac{1}{W} \sum_{w=1}^{W} P(E_w | \bar{\boldsymbol{\pi}}_w, \mathbf{S}_w)
\end{equation}

% 其他指标包括Top-2准确率和Kendall's τ秩相关系数
Additional metrics include Top-2 accuracy and Kendall's $\tau$ rank correlation.

% 表格总结了34季的一致性指标
Table~\ref{tab:consistency} summarizes the consistency metrics across all 34 seasons.

\begin{table}[htbp]
\centering
\caption{Consistency Metrics Summary (Q1a)}
% 一致性指标汇总表
\label{tab:consistency}
\begin{tabular}{lc}
\hline
\textbf{Metric} & \textbf{Value} \\
\hline
% 分析的总周数
Total weeks analyzed & 301 \\
% 淘汰预测准确率
Elimination Prediction Accuracy (EPA) & 93.7\% \\
% Top-2准确率(淘汰者在预测的后两名中)
Top-2 Accuracy & 97.7\% \\
% 平均淘汰概率
Mean Elimination Probability & 0.481 \\
% 平均Kendall's τ
Mean Kendall's $\tau$ & 0.717 \\
\hline
% 按计分方法分类:
\multicolumn{2}{l}{\textit{By Scoring Method:}} \\
% 排名法(74周)
\quad Ranking Method (n=74) & EPA = 95.9\% \\
% 百分比法(227周)
\quad Percentage Method (n=227) & EPA = 93.0\% \\
\hline
\end{tabular}
\end{table}

% --- 一致性结果分析 ---
% 对表格数据进行解读,说明模型的有效性
The results demonstrate that our Bayesian vote estimation model achieves strong consistency
with observed elimination outcomes.
% 93.7%的EPA表明估计的投票在绝大多数情况下能正确预测淘汰
The 93.7\% EPA indicates that our estimated vote distributions correctly predict
the actual elimination in the vast majority of weeks.
% Top-2准确率97.7%更具说服力:几乎所有情况下,实际被淘汰者都在模型预测的后两名中
The 97.7\% Top-2 accuracy is particularly compelling: in nearly all cases, the
eliminated contestant was among the bottom two predicted by our model.

% 值得注意的是,排名法(95.9%)的EPA高于百分比法(93.0%)
Notably, the Ranking Method achieves higher EPA (95.9\%) compared to the Percentage
Method (93.0\%).
% 这一差异可以从数学性质解释:排名法的离散性质使得约束更加明确
This difference can be explained by the mathematical properties of each method:
the discrete nature of the Ranking Method creates sharper elimination boundaries,
% 使MCMC更容易收敛到满足约束的投票分布
making it easier for MCMC to converge to vote distributions that satisfy the constraints.
% 相比之下,百分比法的连续性质导致更大的可行解空间
In contrast, the continuous nature of the Percentage Method leads to a larger
feasible solution space, introducing more uncertainty.

% 平均淘汰概率0.481表明模型对淘汰预测有中等置信度
The mean elimination probability of 0.481 indicates moderate confidence in predictions.
% 这个值低于1.0是符合预期的:软约束允许非确定性预测
This value being well below 1.0 is expected and desirable: our soft constraint
formulation deliberately avoids deterministic predictions,
% 诚实地反映了逆问题固有的不确定性
honestly reflecting the inherent uncertainty of this inverse problem.

% Kendall's τ为0.717表明预测排名与隐含投票排名高度相关
The Kendall's $\tau$ of 0.717 indicates strong rank correlation between predicted
elimination order and the implied vote rankings,
% 进一步验证了模型捕捉投票行为模式的能力
further validating the model's ability to capture voting behavior patterns.

% --- 确定性指标(Q1b)---
\subsubsection{Certainty Metrics (Q1b)}

% 变异系数(CV):标准差与均值的比值,衡量估计的相对不确定性
\textbf{Coefficient of Variation (CV)}:
\begin{equation}
\text{CV}_i = \frac{\sigma_{\pi_i}}{\mu_{\pi_i}}
\end{equation}

% 95%可信区间宽度:后验分布的2.5%和97.5%分位数之差
\textbf{95\% Credible Interval Width}:
\begin{equation}
\text{CI}_i = \pi_i^{97.5\%} - \pi_i^{2.5\%}
\end{equation}

% 我们还报告有效样本量(ESS)和MCMC接受率来评估采样质量
We also report effective sample size (ESS) and MCMC acceptance rates to assess
sampling quality.

% 表格总结了确定性指标
Table~\ref{tab:certainty} summarizes the certainty metrics.

\begin{table}[htbp]
\centering
\caption{Certainty Metrics Summary (Q1b)}
% 确定性指标汇总表
\label{tab:certainty}
\begin{tabular}{lc}
\hline
\textbf{Metric} & \textbf{Value} \\
\hline
% 所有选手的平均CV
Mean CV (all contestants) & 0.710 \\
% 平均95%可信区间宽度
Mean 95\% CI Width & 0.292 \\
\hline
% 按淘汰状态分类:
\multicolumn{2}{l}{\textit{By Elimination Status:}} \\
% 被淘汰选手的CV
\quad Eliminated contestants CV & 0.761 \\
% 未被淘汰选手的CV
\quad Non-eliminated contestants CV & 0.704 \\
% 被淘汰选手的CI宽度
\quad Eliminated contestants CI width & 0.167 \\
% 未被淘汰选手的CI宽度
\quad Non-eliminated contestants CI width & 0.308 \\
\hline
\end{tabular}
\end{table}

% --- 确定性结果分析 ---
% 对不确定性指标进行解读
The certainty metrics reveal important patterns about estimation uncertainty.
% 平均CV为0.710表明投票估计存在相当的不确定性,这是逆问题的固有特征
The mean CV of 0.710 indicates substantial uncertainty in vote estimates,
which is an inherent characteristic of this inverse problem rather than a model deficiency.

% 一个关键发现是被淘汰者与未淘汰者之间的不确定性差异
A key finding is the asymmetry between eliminated and non-eliminated contestants.
% 被淘汰者的CI宽度(0.167)显著小于未淘汰者(0.308)
Eliminated contestants have narrower CI widths (0.167) compared to non-eliminated
contestants (0.308).
% 这是符合直觉的:淘汰事件为被淘汰者的投票份额提供了强约束(必须是最低的)
This is intuitive: the elimination event provides a strong constraint on the
eliminated contestant's vote share---it must be low enough to result in elimination.
% 而未淘汰者只需满足"不是最低"的弱约束,允许更大的变化范围
Non-eliminated contestants only need to satisfy the weaker constraint of ``not being
the lowest,'' allowing for a wider range of plausible values.

% 尽管被淘汰者的CV(0.761)略高于未淘汰者(0.704)
Despite eliminated contestants having slightly higher CV (0.761 vs. 0.704),
% 这是因为被淘汰者通常投票份额较低,导致相同的绝对不确定性产生更大的相对不确定性
this is because eliminated contestants typically have lower vote shares,
causing the same absolute uncertainty to translate into larger relative uncertainty.

% --- 不确定性解读 ---
\subsubsection{Interpretation of Uncertainty}

% 投票估计中的不确定性反映了该逆问题固有的不确定性
The uncertainty in vote estimates reflects the inherent indeterminacy of this inverse
problem.
% 我们的模型提供完整的后验分布而非点估计
Rather than point estimates, our model provides complete posterior distributions.
% 例如,后验均值15%、95%CI为[12%,18%]表示:
For example, a posterior mean of 15\% with 95\% CI [12\%, 18\%] indicates that
% "我们有95%的置信度认为该选手获得了12%到18%的观众投票"
``we are 95\% confident the contestant received between 12\% and 18\% of audience votes.''

% 这种诚实的不确定性量化对于后续任务至关重要
This honest uncertainty quantification is crucial for subsequent tasks.
% 在Task 2-3的比较分析中,我们将使用完整的后验分布而非点估计
In Tasks 2--3, we will propagate this uncertainty through our comparative analyses,
% 确保结论对投票估计中的不确定性具有鲁棒性
ensuring that our conclusions are robust to the uncertainty in vote estimates.
% 任何声称"观众投票比例恰好是X%"的模型都是过度自信的
Any model claiming ``the audience vote share was exactly X\%'' would be overconfident;
% 我们的贝叶斯方法避免了这种陷阱
our Bayesian approach avoids this pitfall.



% Chapter 3:
\section{Fair Possibility Coefficient-Gini Coefficient (FPC-GC) Model}
\vspace{-0.4cm}
First of all, we need to clarify the definition of global equity: “exploration and use of outer space, including the moon and other celestial bodies, shall be carried out for the benefit and in the interests of all countries, irrespective of their degree of economic or scientific development, and shall be the province of all mankind”, as we all know from the 1967 UN Outer Space Treaty. Therefore, we think that global equity means that "a country can profit from asteroid mining regardless of its ability to participate in asteroid mining". 

We are going to develop a model that measures global equity, and it should satisfy the following conditions:
\begin{itemize}
	\item The model can measure the current global equity in the issue of "asteroid mining" well. 
	\item The model should take more comprehensive range of factors into account, including history, region, and national power.
	\item The model is of universal use. When the world's situation changes, our model can measure global equity by giving the appropriate indicators.
\end{itemize}
\vspace{-0.4cm}
\subsection{Determination of Indicators}

It is important to note that only few countries are currently able to conduct asteroid mining, and most countries in the world still do not have the ability to launch spacecraft into outer space for mining. Therefore, if these countries want to benefit from asteroid mining, they need help and support from countries that are able to participate in asteroid mining, such as technological support and benefit sharing. Therefore, the countries we are considering need to meet the following two conditions:
\begin{itemize}
	\item The country is capable of launching spacecraft for asteroid mining.
	\item The country is capable and willing to comply with policies that are globally equitable and beneficial. Also, they are willing to provide scientific and technological support and resource sharing to those countries that are lagging behind.
\end{itemize}

Based on the above two principles, we define a new variable: the \textbf{Fair Probability Coefficient (FPC)}. The larger this factor, the greater a country's asteroid mining capability and the higher the likelihood of compliance with international treaties.

For the purpose of asteroid mining, the country's technology and economy must be strong. Moreover, a country's willingness to share some of its benefits with other countries is predicated on that country having sufficient energy production and a desire to achieve global equity. We use the four indicators of "Energy Output (EO), Scientific and Technological Level (STL), Economic Level (EL), and Sense of Responsibility (SR)" as our superior indicators to determine the fairness likelihood factor, and the corresponding subordinate indicators are shown in the table below:
\vspace{-0.2cm}
\begin{table}[H]
	\centering
	\caption{The descriptions of indicators}
	\setlength{\tabcolsep}{4pt}
	\renewcommand{\arraystretch}{0.8}
	\begin{tabular}{|l|l|c|}
		\hline
		\rowcolor[HTML]{96A6D4} 
		\textbf{Superior Indicator}                                                             & \multicolumn{1}{c|}{\cellcolor[HTML]{96A6D4}\textbf{Subordinate Indicator}}                                                                                                                                                            & \textbf{Effect}                                       \\ \hline
		\rowcolor[HTML]{DAE8FC} 
		\begin{tabular}[c]{@{}l@{}}Energy Output \\ (EO)\end{tabular}                          & \begin{tabular}[c]{@{}l@{}}Mineral resources (MR)\\ Liquid fuel (LF)\end{tabular}                                                                                                                                                      & \begin{tabular}[c]{@{}c@{}}$ + $\\ $ + $\end{tabular}         \\ \hline
		\rowcolor[HTML]{C7DFFF} 
		\begin{tabular}[c]{@{}l@{}}Scientific and \\ Technological \\ Level (STL)\end{tabular} & \begin{tabular}[c]{@{}l@{}}Education index (EI)\\ Research and development expense (RDP) \\ Number of articles in scientific journals (NASJ)\\ Number of spacecraft launched (NSL)\end{tabular}                                        & \begin{tabular}[c]{@{}c@{}}$ + $\\ $ * $\\ $ * $\\ $ + $\end{tabular} \\ \hline
		\rowcolor[HTML]{DAE8FC} 
		\begin{tabular}[c]{@{}l@{}}Economic Level \\ (EL)\end{tabular}                         & \begin{tabular}[c]{@{}l@{}}GDP per capita (GDPP)\\ GDP\\ GDP growth (GDPG)\end{tabular}                                                                                                                                                 & \begin{tabular}[c]{@{}c@{}}$ + $\\ $ + $\\ $ + $\end{tabular}     \\ \hline
		\rowcolor[HTML]{C7DFFF} 
		\begin{tabular}[c]{@{}l@{}}Sense of \\ Responsibility \\ (SR)\end{tabular}             & \begin{tabular}[c]{@{}l@{}}Amount of foreign aid in the last ten years (AFA)\\ Number of breaches of international treaties (NBIT)\\ Military strength score (MSS)\\ Number of countries with diplomatic relations (NCDR)\end{tabular} & \begin{tabular}[c]{@{}c@{}}$ + $\\ $ - $\\ $ * $\\ $ + $\end{tabular} \\ \hline
	\end{tabular}
\end{table}
\vspace{-0.4cm}
\newpage
In this table, effects of "$ + $" are benefit attributes (the larger the better), effects of "$ - $" are cost attributes (the smaller the better), and effects of "$ * $" are interval attributes (the closer to a certain interval [$ a $, $ b $] the better).

\begin{enumerate}
	\item \textbf{Energy Output (EO)}
	\begin{itemize}
		\setlength{\parsep}{0ex} %段落间距
		\setlength{\topsep}{2ex} %列表到上下文的垂直距离
		\setlength{\itemsep}{1ex} %条目间距
		\item \textbf{\emph{MR}}: MR means a country's mineral resources.
		\item \textbf{\emph{LF}}: LF means Liquid fuel which represents a country's production of liquid fuels.
	\end{itemize}
	\item \textbf{Scientific and Technological Level (STL)}
	\begin{itemize}
		\setlength{\parsep}{0ex} %段落间距
		\setlength{\topsep}{2ex} %列表到上下文的垂直距离
		\setlength{\itemsep}{1ex} %条目间距
		\item \textbf{\emph{EI}}: EI means education index, which represent the educational performance of a country.
		\item \textbf{\emph{RDP}}: RDP represents a country's R\&D spending as a percentage of GDP. Considering that the RDP of a country like South Korea, which is making great efforts to develop science and technology, is higher than other countries, but its level of science and technology is not very outstanding. So, we this is an interval attribute indicator, and according to the characteristics of the indicator, we set the optimal interval as \textbf{[2.5, 2.8]}.
		\item \textbf{\emph{NASJ}}: NASJ represents the number of scientific articles published by a country in international journals in the past year. Considering that excessive focus on the number of journals will hinder the development of manufacturing, so we set the optimal interval to \textbf{[100,120]}.
		\item \textbf{\emph{NSL}}: NSL means the number of spacecrafts launched. 
	\end{itemize}
	\item \textbf{Economic Level (EL)}
	\begin{itemize}
		\setlength{\parsep}{0ex} %段落间距
		\setlength{\topsep}{2ex} %列表到上下文的垂直距离
		\setlength{\itemsep}{1ex} %条目间距
		\item \textbf{\emph{GDPP}}: GDPP means the GDP per capita. 
		\item \textbf{\emph{GDP}}: GDP means gross domestic product in a certain period of time. 
		\item \textbf{\emph{GDPG}}: GDPG means GDP growth in the past year.
	\end{itemize}
	\item \textbf{Sense of Responsibility (SR)}
	\begin{itemize}
		\setlength{\parsep}{0ex} %段落间距
		\setlength{\topsep}{2ex} %列表到上下文的垂直距离
		\setlength{\itemsep}{1ex} %条目间距
		\item \textbf{\emph{AFA}}: AFA represents the total amount of foreign aid a country has provided over the past decade. The higher the AFA, the more friendly the country is as well as the more likely it tends to be responsible on issues of global equity.
		\item \textbf{\emph{NBIT}}: NBIT represents the number of times a country has violated international treaties in its history. The higher the NBIT, the less credible and less accountable the country is, so this is a cost attribute indicator.
		\item \textbf{\emph{MSS}}: MSS means military strength score. We believe that excessive military power in a country may facilitate a dictatorship by that country, and no international conventions will be able to restrain their behavior, so we set the best interval as \textbf{[0.15, 0.17]}.
		\item \textbf{\emph{NCDR}}: NCDR means the number of countries with diplomatic relations. 
	\end{itemize}
\end{enumerate}

\subsection{Data Pre-processing}
\subsubsection{Data Collection}
Adequate data are the foundation for designing the evaluation system. Considering that there are only a limited number of countries that have the capability to launch spacecraft for asteroid mining, we selected 25 countries that are currently able to launch spacecraft or will launch spacecraft in the next few years for data collection. The countries participating in the evaluation including Asia's China and Japan, North America's United States and Canada, and Europe's France and United Kingdom. In addition, our data were obtained from credible official websites such as the World Bank, the U.S. Bureau of Statistics, UNESCO, etc.
\subsubsection{Data Filling}
Due to the limited access to international data and the fact that certain countries do not count the indicators we have selected or are not included by the United Nations and other international organizations, so the data we have collected are a little bit missing, and in order to fill these missing data, we have done the followings:
\begin{enumerate}
	\item For the Energy output data, we considering that a certain natural resource is concentrated in a particular landscape. Therefore, we look for a country that is geographically similar to it and use the ratio of the territory of the two countries to approximate the ratio of natural resource reserves to obtain the natural resource reserves of the country.
	\item For some indicators with small variance and few missing values, we chose to use the mean of that indicator as an alternative.
	\item For some indicators with few missing values and strong time correlation, we use regression interpolation to fill them.
	\item In the case where a country lacks data for all years for a statistical indicator, by considering the specific meaning of the indicator, we use the mean value of all countries to fill it.
\end{enumerate}

By processing the missing values, we obtained complete data for all 25 countries for the last 20 years.

\subsubsection{Handling Outliers}

We analyze each indicator through descriptive statistics and box line plots. We found highly anomalous outliers that deviated from the mean by more than twice the standard deviation in all countries.
\begin{itemize}
	\setlength{\itemsep}{1ex} %条目间距
	\item For data with $\alpha <$ 0.01, we discarded them and then filled them with the method mentioned above.
	\item In addition, in order not to create a large noise in the results of the latter method, we also discarded some outliers that differed significantly from other indicators and filled in the missing values using the above method.
\end{itemize}

\subsubsection{Data Normalization}
Since we have obtained complete and reliable data after the above steps, it is important to take into account that the different metrics are not the same type. Therefore, in order to use the same metric, we also need to standardize all the metrics. For the different types of metrics, we use the following different ways of standardization:
\begin{itemize}
	\item Benefit Attributes: the bigger the better.
	\begin{equation}
		\tilde{x}_{i j}=\frac{x_{i j}-\min \left\{x_{i}\right\}}{\max \left\{x_{i}\right\}-\min \left\{x_{i}\right\}}
	\end{equation}
	\item Cost Attributes: the smaller the better.
	\begin{equation}
		\tilde{x}_{i j}=\frac{\max \left\{x_{i}\right\}-x_{i j}}{\max \left\{x_{i}\right\}-\min \left\{x_{i}\right\}}
	\end{equation}
	\item Interval Attributes: its optimal value lies within a certain interval [$ a $, $ b $].
	\begin{equation}
		M=\max \left\{a-\min \left\{x_{i}\right\}, \max \left\{x_{i}\right\}-b\right\}
	\end{equation}
    \begin{equation}
	    \tilde{x}_{i j}=\left\{\begin{array}{l}
	    	1-\frac{a-x_{i j}}{M}, x_{i j}<a \\
	    	\\
	    	1 \quad, a \leqslant x_{i j} \ll b \\
	    	\\
	    	1-\frac{x_{i j}-b}{M}, x_{i j}>b
	    \end{array}\right.
    \end{equation}
\end{itemize}

\subsection{Determination of the Weights for Indicators}
There are various ways to determine the weights of each indicator. Considering that the \textbf{Analytic Hierarchy Process (AHP)} is more subjective in determining the weights and is vulnerable to the subjective perception of decision makers. While the \textbf{Entropy Weighting Method (EWM)} is more sensitive to the data, even though we have preprocessed the data, we cannot avoid the situation that some data do not match the actual situation, so we use the \textbf{EWM-AHP} combined weighting method to reduce our errors and improve the accuracy of our model.

\subsubsection{Determination of Weights by EWM}

In the EWM, the greater the dispersion of an indicator, the higher its corresponding weight is. The data we collect is real, so it can reflect the information in reality very well. We take each superior indicator and its corresponding subordinate indicator as an evaluation system, and let there be $ n $ objects to be evaluated in our evaluation system and $ m $ evaluation indicators that have been standardized to form a standardized matrix\upcite{2}, which is showed below:
	\begin{equation}
	X=\left[\begin{array}{cccc}
		x_{11} & x_{12} & \cdots & x_{1 m} \\
		x_{21} & x_{22} & \cdots & x_{2 m} \\
		\vdots & \vdots & \ddots & \vdots \\
		x_{n 1} & x_{n 2} & \cdots & x_{n m}
	\end{array}\right]
\end{equation}

We define the normalized matrix as $ Z $, and each element in $ Z $ as:
\begin{equation}
	z_{ij}=\frac{x_{ij}}{\sqrt{\sum\limits_{i=1}^n{{x_{ij}}^2}}}
\end{equation}

%Determine whether there are negative numbers in the $ z $ matrix, and if so, normalize the matrix $ x $ using another method to obtain the $ \widetilde{z} $ matrix, which is normalized by the formula:
%\begin{equation}
%	\tilde{z}_{i j}=\frac{x_{i j}-\min \left\{x_{1 j}, x_{2 j}, \cdots, x_{n j}\right\}}{\max \left\{x_{1 j}, x_{2 j}, \cdots, x_{n j}\right\}-\min \left\{x_{1 j}, x_{2 j}, \cdots, x_{n j}\right\}}
%\end{equation}

After processing, we get the normalized matrix:
\begin{equation}
Z=\left[ \begin{matrix}
	z_{11}&		z_{12}&		\cdots&		z_{1m}\\
	z_{21}&		z_{22}&		\cdots&		z_{2m}\\
	\vdots&		\vdots&		\ddots&		\vdots\\
	z_{n1}&		z_{n2}&		\cdots&		z_{nm}\\
\end{matrix} \right] 
\end{equation}

Calculating the probability matrix $ p $. The formula for each element $ p_{ij} $ in $ p $ is as follows:
\begin{equation}
	p_{ij}=\frac{\widetilde{z}_{ij}}{\sum\limits_{i=1}^n{\widetilde{z}_{ij}}}
\end{equation}

We can easily verify that the probability sum corresponding to an indicator is 1.

For the $j^{th}$ indicator, the formula for calculating its information entropy is:
\begin{equation}
	e_{j}=-\frac{1}{\ln n} \sum_{i=1}^{n} p_{i j} \ln \left(p_{i j}\right)(j=1,2, \cdots, m)
\end{equation}

We define the information utility value as:
\begin{equation}
	d_j=1-e_j
\end{equation}

After normalizing the information utility values, we are able to obtain the entropy weight of each indicator:
\begin{equation}
	W_{Ej}=\frac{d_j}{\sum\limits_{j=1}^m{d_j}}(j=1,2,\cdots ,m)
\end{equation}

We applied the EWM to our data and obtained the weights of each subordinate indicator in its corresponding evaluation system, and the results are shown in the following table:

\begin{minipage}{\textwidth}
	\begin{minipage}[t]{0.45\textwidth}
		\centering
		\makeatletter\def\@captype{table}\makeatother\caption{The EWM's weights}
		\renewcommand{\arraystretch}{0.9}
		\begin{tabular}{|c|cc|}
			\hline
			\rowcolor[HTML]{B9C5E8} 
			\multicolumn{1}{|l|}{\textbf{}} & \textbf{Indicator}                                              & \textbf{Weight}                                                           \\ \hline
			\rowcolor[HTML]{F6F6F9} 
			EO                             & \begin{tabular}[c]{@{}c@{}}MR\\ LF\end{tabular}                 & \begin{tabular}[c]{@{}c@{}}0.6239\\ 0.3761\end{tabular}                   \\ \hline
			\rowcolor[HTML]{E4E6F0} 
			STL                            & \begin{tabular}[c]{@{}c@{}}EI\\ RDP\\ NASJ\\ NSL\end{tabular}   & \begin{tabular}[c]{@{}c@{}}0.0110\\ 0.1622\\ 0.1124\\ 0.7144\end{tabular} \\ \hline
			\rowcolor[HTML]{F6F6F9}
			EL                             & \begin{tabular}[c]{@{}c@{}}GDPP\\ GDP\\ GDPG\end{tabular}        & \begin{tabular}[c]{@{}c@{}}0.0742\\ 0.2508\\ 0.6750\end{tabular}          \\ \hline
			\rowcolor[HTML]{E4E6F0}
			SR                             & \begin{tabular}[c]{@{}c@{}}AFA\\ NBIT\\ MSS\\ NCDR\end{tabular} & \begin{tabular}[c]{@{}c@{}}0.3484\\ 0.1410\\ 0.2644\\ 0.2462\end{tabular} \\ \hline
		\end{tabular}
	\end{minipage}
	\begin{minipage}[t]{0.45\textwidth}
		\centering
		\makeatletter\def\@captype{table}\makeatother\caption{The AHP's weights}
		\renewcommand{\arraystretch}{0.9}
		\begin{tabular}{|c|cc|}
			\hline
			\rowcolor[HTML]{B9C5E8} 
			\multicolumn{1}{|l|}{\textbf{}} & \textbf{Indicator}                                              & \textbf{Weight}                                                           \\ \hline
			\rowcolor[HTML]{F6F6F9} 
			EO                             & \begin{tabular}[c]{@{}c@{}}MR\\ LF\end{tabular}                 & \begin{tabular}[c]{@{}c@{}}0.5876\\ 0.4124\end{tabular}                   \\ \hline
			\rowcolor[HTML]{E4E6F0} 
			STL                            & \begin{tabular}[c]{@{}c@{}}EI\\ RDP\\ NASJ\\ NSL\end{tabular}   & \begin{tabular}[c]{@{}c@{}}0.1538\\ 0.3829\\ 0.2346\\ 0.2287\end{tabular} \\ \hline
			\rowcolor[HTML]{F6F6F9}
			EL                             & \begin{tabular}[c]{@{}c@{}}GDPP\\ GDP\\ GDPG\end{tabular}        & \begin{tabular}[c]{@{}c@{}}0.4632\\ 0.3825\\ 0.1543\end{tabular}          \\ \hline
			\rowcolor[HTML]{E4E6F0}
			SR                             & \begin{tabular}[c]{@{}c@{}}AFA\\ NBIT\\ MSS\\ NCDR\end{tabular} & \begin{tabular}[c]{@{}c@{}}0.4232\\ 0.2613\\ 0.2079\\ 0.1076\end{tabular} \\ \hline
		\end{tabular}
	\end{minipage}
\end{minipage}
\vspace{-0.4cm}



\subsubsection{Determination of Weights by AHP}
Similarly, we use each superior indicator and its subordinate secondary indicator as an evaluation system, then we use \textbf{AHP} to determine the weight $\omega _{Aj}$ of the $j^{th}$ indicator in each evaluation system\upcite{3}. After passing the consistency test, our results are shown in the table of the last page.

In order to determine the \textbf{Fair possibility coefficient (FPC)}, we also use \textbf{AHP} to determine the weights of the four superior indicators. By combining those four factors by using AHP, we give the following judgment matrix:
\begin{table}[H]
	\centering
	\setlength{\tabcolsep}{2pt}
	\renewcommand{\arraystretch}{1.3}
	\begin{tabular}{ll}
		& \enspace SR STL EO EL \\
		\begin{tabular}[c]{@{}l@{}} SR \\ STL\\ EO\\ EL\end{tabular} & 	$ \left(\begin{array}{cccc}
			1 & 2 & 3 & 3 \\
			\frac{1}{2} & 1 & \frac{3}{2} & 2 \\
			\frac{1}{3} & \frac{2}{3} & 1 & 1 \\
			\frac{1}{3} & \frac{1}{2} & 1 & 1
		\end{array}\right) $          
	\end{tabular}
\end{table}

The results we obtained have passed the consistency test, and the weights vector of the superior indicators we obtained is:
\begin{equation}
	\omega _S=\left( 0.4582,0.2467,0.1527,0.1424 \right) 
\end{equation}
\subsubsection{Determination of Integrated Weights}
We define that the weight of the $ j^{th} $ indicator in the evaluation system formed by each superior indicator and its subordinate secondary indicator is:
\begin{equation}
	\omega _j=\frac{\omega _{Ej}+\omega _{Aj}}{2}
\end{equation}

Ultimately, we derived the weight magnitude of each subordinate and superior indicator by calculation, which are shown in the following figure:
\begin{figure}[H]
	\centering
	\includegraphics[width=.6\textwidth]{quan2.png}
	\caption{Pie chart of weights}
	\label{img}
\end{figure}
\vspace{-0.4cm}
\subsection{Establishment of FPC-GC Model}
\subsubsection{Determination of Scores of Superior Indicators}
We use \textbf{TOPSIS method} to determine the score of each superior indicator. Taking each superior indicator as an evaluation system, we suppose that there are $ n $ evaluation countries and $ m $ subordinate indicators in the evaluation system with matrix "$ Z $".

We define the maximum value is:
\begin{equation}
	\begin{aligned}
		Z^+ &= \left( Z_{1}^{+},Z_{2}^{+},\cdots ,Z_{m}^{+} \right) \\
		& =\left( max\left\{ z_{11},z_{21},\cdots ,\left. z_{n1} \right\} \right. ,max\left\{ z_{12},z_{22},\cdots ,\left. z_{n2} \right\} \right. ,\cdots ,max\left\{ z_{1m},z_{2m},\cdots ,\left. z_{nm} \right\} \right. \right)
	\end{aligned}
\end{equation}

We define the minimum value is:
\begin{equation}
	\begin{aligned}
		Z^- &= \left( Z_{1}^{-},Z_{2}^{-},\cdots ,Z_{m}^{-} \right) \\
		& =\left( min\left\{ z_{11},z_{21},\cdots ,\left. z_{n1} \right\} \right. ,min\left\{ z_{12},z_{22},\cdots ,\left. z_{n2} \right\} \right. ,\cdots ,min\left\{ z_{1m},z_{2m},\cdots ,\left. z_{nm} \right\} \right. \right)
	\end{aligned}
\end{equation}

Define the distance of the $ i^{th} $ evaluation object from the maximum and minimum value are:
\begin{equation}
	D_{i}^{+}=\sqrt{\sum_{j=1}^m{\omega _j\left( Z_{j}^{+}-z_{ij} \right) ^2}}\quad , \quad D_{i}^{-}=\sqrt{\sum_{j=1}^m{\omega _j\left( Z_{j}^{-}-z_{ij} \right) ^2}}
\end{equation}

Then, we can calculate the unnormalized score of the $ i^{th} $ ($ i $=1,2,$ \cdots $,$ n $) evaluation object:
\begin{equation}
	S_i=\frac{D_{i}^{-}}{D_{i}^{+}+D_{i}^{-}}
\end{equation}

We can normalize the score:

\begin{equation}
	\widetilde{S}_i=\frac{S_i}{\sum\limits_{i=1}^n{S_i}} 
\end{equation}

In this case, \quad $ \sum\limits_{i=1}^n{\widetilde{S}_i}=1 $.
We further define the score vector for the four superior indicators for country $ i $ as:
\begin{equation}
	S=\left( S_{SRi},S_{STLi},S_{EOi},S_{ELi} \right) 
\end{equation}

We can then calculate the value of the final FPC for country $ i $:
\begin{equation}
	FPC=\left( S_{SRi},S_{STLi},S_{EOi},S_{ELi} \right) \cdot {\omega _S}^T
\end{equation}

Then we rank the countries according to the size of the FPC and plot the Lorenz curve of FPC distribution by country. By using the curve and the corresponding formula, we can calculate the corresponding \textbf{Gini Coefficient (GC)}. Finally by analyzing the Gini Coefficient and the Lorenz curve, we can measure global equity.

\subsection{Application of FPC-GC Model}
We applied the \textbf{FPC-GC model} to our data. Through calculations, we obtained the corresponding FPC values for the 25 countries involved in the evaluation. Then we plot the corresponding heat map:

\begin{figure}[H]
	\centering
	\includegraphics[width=11cm, height=5cm]{3cai.png}
	\caption{Heat map}
	\label{img}
\end{figure}
\vspace{-0.6cm}
Through the information reflected in the picture, the U.S. has the highest FPC, meanwhile, Russia and China are closely following. At the same time, European countries, Australia and Japan as well as Korea are in the middle. Brazil and Argentina in South America are lagging behind relatively.

"Lorenz curve is a curve used in economics to analyze the fairness of social income distribution or property distribution, and Gini Coefficient is a statistical indicator reflecting the fairness of social income distribution based on the calculated Lorenz curve\upcite{4}". In short, by rationalizing the Lorenz curve and Gini Coefficient slightly and applying them to our model, we can measure the global equity of the current allocation of the world on the asteroid mining problem.
We are going to plot the Lorenz curve of FPC distribution by country and calculate its corresponding Gini Coefficient.

We have $ n $ countries participating in the evaluation, and after ascending order, the FPC of the $ i^{th} $ country is 23, then the coordinates of the $ k^{th} $ point on the Lorenz curve are\upcite{5}:
\begin{equation}
	\begin{cases}
		x=\frac{k}{n}\\
		\\
		y=\frac{\sum\limits_{i=1}^k{FPC_i}}{\sum\limits_{i=1}^n{FPC_i}}\\
	\end{cases}
\end{equation}

By the method described above, we plotted the Lorenz curve as shown below:
\vspace{-0.2cm}
\begin{figure}[H]     %固定,容易留白,但可能串段
	\centering
	\begin{minipage}[t]{0.48\textwidth}
		\centering
		\includegraphics[width=7cm]{lan1.png}
		\caption{Lorenz curve}
	\end{minipage}
	\begin{minipage}[t]{0.48\textwidth}
		\centering
		\includegraphics[width=7cm]{lan2.png}
		\caption{Gini Coefficient diagram}
	\end{minipage}
\end{figure}
\vspace{-0.4cm}
In the left figure, the green dashed line represents the theoretical fairness curve. When the Lorenz curve overlaps completely with the theoretical fairness curve, it means that the evaluation object is absolutely fair. The farther the Lorenz curve is from the theoretical fairness curve, the higher its curvature is, and the corresponding evaluation object is less global equity.

According to the Lorenz curve, we can calculate the corresponding Gini Coefficient of the system, as shown in the figure on the right, A and B represent the area of two closed graphs in the plane of the curve, and we define the \textbf{Gini Coefficient}:
\begin{equation}
	GC=\frac{A}{A+B}
\end{equation}
\vspace{-0.2cm}
By reviewing the literature\upcite{6}, it is clear that:  
\begin{itemize}
	\setlength{\parsep}{0ex} %段落间距
	\setlength{\topsep}{2ex} %列表到上下文的垂直距离
	\setlength{\itemsep}{1ex} %条目间距
	\item $ GC< $ 0.3: best fair state
	\item 0.3 $ <GC< $  0.4: normal fair state
	\item $ GC> $ 0.4: alert fair state
	\item $ GC> $ 0.6: highly dangerous fair state
\end{itemize}
Based on the geometric definition of the Gini Coefficient, we can express the $ GC $ as:
\begin{equation}
	GC=1-\frac{1}{n}\sum\limits_{k=0}^{n-1}{\frac{\text{(}2\sum\limits_{i=1}^k{FPC_i}\text{)}+FPC_{k+1}}{\sum\limits_{i=1}^n{FPC_i}}}
\end{equation}

Importing our data into Matlab, after calculation we obtain: \textbf{$ \textbf{GC=} $0.2277$ \textbf{<} $0.3}.

Therefore, we can consider that in the current world situation, the issue of asteroid mining is currently in a state of global equity.

\section{Models to Predict Future Impacts}
Asteroid mining is globally equitable at present. But in the near future, such equities are likely to be broken.
\subsection{Obstacles to Asteroid Mining}
The asteroid mining industry is not yet fully developed mainly because of the following two constraints:
\begin{itemize}
	\setlength{\itemsep}{1ex} %条目间距
	\item \textbf{Constraints of technical and economic}:
	A country that wants to conduct asteroid mining needs to overcome the technological constraints of spacecraft launch, space transportation, and resource capture, as well as space station establishment, the realization of thess would be difficult for most countries\upcite{7}.
	\item \textbf{Constrains of regulations}:
    The Outer Space Treaty sets out the basic principles for the development and use of outer space, including that "the exploration and use of outer space resources must be aimed at the benefit of all mankind" as well as that "the development and use of outer space should be open to all countries”\upcite{9}. Also the 1979 Moon Agreement, states that "The Moon and its natural resources are the common heritage of all mankind"\upcite{8}.
    These legal treaties, which are based on the common good of mankind, restrict the mining activities of some major powers.
\end{itemize}
\subsection{Strategic Significance of Planetary Mining}
Planetary mining is of strategic importance on three dimensions:
\begin{enumerate}
	\item \textbf{Strategic economic needs}:
	Outer space is rich in resources, and if it can be exploited, it will bring enormous economic and social value.
	\item \textbf{Strategic technical needs}:
	The development of outer space resources can be used for both civilian and military purposes, and if base construction, deep space exploration and celestial defense activities are carried out on other planets, then this issue will affect the security interests of the country. 
	\item \textbf{Strategic political needs}:
    After all, outer space resources are finite, and countries are in competition with each other, as the countries with mature technology first will restrict the participation of later countries in asteroid mining.
\end{enumerate}

\subsection{Analysis of Future Vision}
With the development of space technology by major powers, the current equitable situation may be broken. Under the drive of strategic significance and economic value, some countries will ignore international rules and grab resources, which will undermine global equity.

\begin{itemize}
	\setlength{\itemsep}{1ex} %条目间距
	\item \textbf{Resource monopoly issues}:
    Due to the difference in technological power among countries, the major powers will have an absolute advantage in asteroid mining. Even some countries have provided the legal basis for private entities to conduct asteroid mining, giving them various powers over asteroid mining\upcite{10}. The commercialization and privatization of asteroid mining is an inevitable trend. However, the great powers will monopolize the economic benefits of asteroid mining by means of their great power\upcite{11}.The countries with weaker space technology can only compromise with the major powers if they want to enjoy the benefits of outer space resource development.
	\item \textbf{Military activities issues}:
    Because of the strategic technological importance of asteroid mining, every country will take military applications into consideration when conducts outer space activities\upcite{12}. 
    But the military applications of outer space are inevitable and necessary in order to enhance the defense capabilities of the earth. \upcite{13}. This will likely result in the great powers being able to greatly enhance their military power in resource exploitation. In turn, this would further increase the deterrent power of the major powers in the world, while other countries would be powerless to restrain the behavior of these major powers, thus they completely losing the right to decide on asteroid mining.
\end{itemize}

In summary, both the commercialization and privatization of asteroid mining as well as the ensuing issue of military activities will result in a monopoly of the asteroid mining industry by the major powers. This will gravely harm global equity.

\subsection{Establish Future Impact Forecast (FIF) Model}
In order to study the impact of the above issues on global equity, we need to develop a model to describe a possible vision of how asteroid mining might look like in the future. And the model should satisfy the following conditions:
\begin{itemize}
	\item The model needs to consider the impact of various current factors, such as the difference in power between countries and the impact of currently existing policies on the future vision of asteroid mining.
	\item The model needs to be highly adaptable and universal, and it can respond quickly to unknown impacts that may occur in the future, like new treaties for global equity, etc.
	\item The model's predicted results for the future vision need to be similar to the results of our previous analysis.
\end{itemize}

We define this model as the \textbf{FIF model}.
\subsubsection{Classification of Countries}
In the future, countries whose space technology is advanced will have a great advantage in asteroid mining activities. In order to define which countries are advanced in space technology and which countries' space technology is not mature or even lagging behind, we need to conduct a \textbf{Cluster Analysis} of countries, because FPC integrates a country's technological leadership and international responsibility, so we believe that FPC can be used as a measure for a country's maturity of aerospace technology. We use FPC as a reference to \textbf{Hierarchically cluster} countries, and substitute our data into SPSS for cluster analysis to obtain the elbow diagram shown below:
\begin{figure}[H]
	\centering
	\includegraphics[width=.6\textwidth]{zhou2.png}
	\caption{Elbow diagram}
	\label{img}
\end{figure}
\vspace{-0.4cm}
By observing the elbow diagram, we can know that the degree of distortion decreases significantly when the number of clusters reaches 2. Therefore, we determine the number of clusters to be 2, whereby the final clustering scheme obtained is shown in the following heat and spectral diagrams:
\begin{figure}[H]     %固定,容易留白,但可能串段
	\centering
	\begin{minipage}[t]{0.48\textwidth}
		\centering
		\includegraphics[width=8.5cm,height=6cm]{3lia.png}
		\caption{Classification heat map}
	\end{minipage}
	\begin{minipage}[t]{0.48\textwidth}
		\centering
		\includegraphics[width=6cm,height=6cm]{4pu.png}
		\caption{Classification spectral chart}
	\end{minipage}
\end{figure}
\vspace{-0.4cm}
As we can see from the figure, we can classify the three superpowers - the United States, China, and Russia - into one category, while the other countries are in the second category. In the future, these three countries are likely to have a monopoly in the field of asteroid mining.\\

\begin{figure}[H]
	\centering
	\includegraphics[width=.6\textwidth]{cartoon.png}
	\caption{First class VS Second class}
	\label{img}
\end{figure}
\vspace{-0.4cm}
Thus, we can define countries into two classes:
\begin{itemize}
	\item \textbf{First class}: countries that are the prominent leaders in the field of asteroid mining, they are the United States, China and Russia.
	\item \textbf{Second class}: countries that are at a disadvantage in the field of asteroid mining, i.e. countries other than the above three countries.
\end{itemize}
\subsubsection{Future Impact Prediction Model Based on Differential Equations}
There is not only competition but also interdependence between the two classes of countries on the issue of asteroid mining. In the second class countries the labor is cheaper and the first class countries like the US will set up their factories in these countries and let the second class countries produce as agents, which is beneficial for both countries. Thus, the strong countries can lower their production costs and gain more profits, while the people of the agent countries will gain more jobs and achieve economic development of their countries because of the existence of these factories and enterprises. 

In the competition of energy, the first class countries will go to trade with other countries for the sake of protecting the local energy and environment. By importing energy from other countries, they can protect their environmental, but such a policy will increase the cost of energy. In constrain, for the trafficking countries, although they gain certain economic benefits, they also make the environmental resources suffer certain damage. And both economic and environmental resources can affect a country's FPC.

Referring to the population competition model, we define the comprehensive competitiveness of the first class countries in the asteroid mining problem as $ c_1 $ and the comprehensive competitiveness of the second class countries in the asteroid mining problem as $ c_2 $, $ c_1 $ and $ c_2 $ satisfying the following differential equations:
\begin{equation}
	\begin{cases}
		\frac{dc_1}{dt}=v_1c_1\left( 1-\frac{c_1}{L_1}-\alpha _1\frac{c_2}{L_2}+\beta _1\frac{c_2}{L_2} \right)\\
\\
		\frac{dc_2}{dt}=v_2c_2\left( 1-\frac{c_2}{L_2}-\alpha _2\frac{c_1}{L_1}+\beta _2\frac{c_1}{L_1} \right)\\
	\end{cases}
\end{equation}

In the above equations, $ v_1 $ and $ v_2 $ denote the natural growth rate of the comprehensive competitiveness of the two classes of countries respectively. $ L_1 $ and $ L_2 $ denote the maximum competitiveness of the two classes of countries under current conditions. $ \alpha _1 $ and $ \alpha _2 $ denote the relative dependence of the other country on its own environment and resources. And $ \beta _1 $ and $ \beta _2 $ denote a combined factor of the other country's dependence on this country in the development process, for example, the economic benefits generated by the environment and resources of this country, the scientific and technological sharing of space technology provided by this country and the dependence on the benefit sharing in asteroid mining through this country. Hence the variables $ v $ and $ \beta $ contribute to the comprehensive competitiveness of the country, while $ \alpha $ does not contribute to the comprehensive competitiveness of the country. The value of $ L $ needs to be determined on the country's own conditions, moreover we assume that both $ \alpha $ and $ \beta $ are measured independently.

\subsection{Application of FIF Model}
We assign values to the parameters of the differential equations with references to the present realistic conditions.

By calculating the average FPC of the two classes of countries, we can determine the initial competitiveness of the two classes of countries, the natural growth rate of the first class is smaller because they are at a high level in all aspects. The second class of countries has a higher development potential, so their natural growth rate is faster. By referring to the average of GDP growth rates of the two types of countries, we can determine the average size of $ v $.

With the same principle, we can determine that $ L_1 $ is equal to 1.7 times the initial competitiveness, while $ L_2 $ is equal to 3.4 times the initial competitiveness, then the parameters we were able to determine are shown in the table below:
\begin{table}[H]
	\centering
	\caption{Determined parameter}
	\setlength{\tabcolsep}{10pt}
	\renewcommand{\arraystretch}{1.1}
	\begin{tabular}{|c|c|c|c|c|c|c|}
		\hline
		parameter & $ v_1 $  & $ v_2 $  & $ L_1 $  & $ L_2 $  & $ c_2 $(0)     & $ c_2 $(0)     \\ \hline
		value     & 0.060 & 0.080 & 1.462 & 1.088 & 0.860 & 0.320 \\ \hline
	\end{tabular}
\end{table}
Since the first class countries attach great importance to the protection of environmental resources, so we set $ \alpha _1= $ 0.2. In contrast, the first class countries are able to obtain high benefits through cheap labor, and the benefits they get from cheap labor are very high, considering the rising trend of labor costs,  we set $ \beta _1 $ to be 2 and 1.5 respectively. Some second class countries are still willing to sacrifice the environment for their benefits, considering the importance of the environment, we set $ \alpha _2 $ to be 1.5 and 2 respectively. In order to consolidate their position, the first class countries will curb the development of the second class countries, so we set $ \beta _2=$ 0.6, In order to consolidate their position, the first class countries will curb the development of the second class countries.
\begin{figure}[H]
	\centering
	\includegraphics[width=.5\textwidth]{jzl.png}
	\caption{Changes of competitiveness in the current situation}
	\label{img}
\end{figure}
It can be seen from the figure that in either case, the competitiveness of the first class countries will far exceed that of the second class countries in the next 50 years.

According to the prediction results of the FIF model, we take the final average \textbf{FPC$ = $0.158} for the first class countries, and the final average \textbf{FPC$ = $0.045} for the second class countries, since the future development of a country is difficult to predict, we randomly generate data that meets the above conditions and substitute the data into the FPC-GC model:
\begin{figure}[H]     %固定,容易留白,但可能串段
	\centering
	\begin{minipage}[t]{0.48\textwidth}
		\centering
		\includegraphics[width=7cm]{lan3.png}
		\caption{Variation of Lorenz curve}
	\end{minipage}
	\begin{minipage}[t]{0.48\textwidth}
		\centering
		\includegraphics[width=7cm]{lan4.png}
		\caption{Area map and GC}
	\end{minipage}
\end{figure}
It can be seen from the figure that the bending degree of the future Lorentz curve will become larger, \textbf{GC will increase to 0.4423}, the global equity will be on alert, The first class countries are likely to monopolize the asteroid mining industry, and their economic power and military power will be significantly improved; Although other countries can obtain some benefits and the sharing of space technology through monopoly countries, they will always be at a disadvantage, and it will be difficult for them to develop further, the global equity will be greatly damaged.
\section{Analysis of the Impact at Global Equity After Conditions Changed by Using the FIF Mode}
In the FIF model, various realistic factors influence the parameters of our differential equations. Therefore, if we want to study the effect of changing conditions on global equity, we only need to change the parameters of the differential equations. Considering that parameters such as $ L $ and $ v $ are determined by the country's own conditions and attributes, we study this issue by changing $ \alpha $ and $ \beta $, and the practical significance of parameter changes is the impact of international treaties and pressures on the country. We will start our exploration with the different attitudes of the first class countries towards future treaties.
\begin{itemize}
	\setlength{\itemsep}{1ex} %条目间距
	\item \textbf {Willing to share and provide technical support and policy benefits}\\
	When the first class countries are willing to share their resources and benefits to other countries, the dependence of other countries on their resources will increase, but considering the principle of protecting the environment and resources and the effect of implementing the policy, we take $ \alpha _1 $ as 0.5 and 0.8 respectively, so as to analyze the change of comprehensive competitiveness when this country has different willingness to open up. Since the first class countries has implemented economic preferential policies, then it has to sacrifice some of its economic benefits, thus there will be some decline in economic efficiency. Therefore, we take $ \beta _1= $1.5. 
	In this case, both $ \alpha _2 $ and $ \beta _2 $ increase for the second class countries receive resource sharing and economic preferences from the first class countries. We also consider the development constraints of the first class countries on the second class countries
	by respectively taking $ \alpha _2= $0.7 and $ \beta _2 $ as 1 and 0.8. After substituting the data into the differential equation of the FIF, we get the following figure:
\vspace{-0.4cm}
	\begin{figure}[H]
		\centering
		\includegraphics[width=.5\textwidth]{c1.png}
		\caption{Change in competitiveness of the first case}
		\label{img}
	\end{figure}
\vspace{-0.4cm}
From the figure, we can see that the comprehensive competitiveness of both two classes of countries will increase. Although the gap between their competitiveness will still widen in the end, it has slowed down the situation of monopoly in the issue of asteroid mining. And we can see that when the $ \beta _2 $ decreases due to a higher degree of resource sharing by major powers who want to limit the development of weaker countries, the final comprehensive competitiveness of both classes of countries is less than that of the former. In this case, although the final competitiveness gap between the two classes of countries still becomes larger, the change is no longer so big, which will conduct the global equity, and the comprehensive competitiveness of each country will be improved. Therefore, from the perspective of global development and maintenance of global equity, we do not recommend that the first class countries take restrictions or deliberately increase the degree of resource sharing.
	\item \textbf {Provide economic and scientific support without opening up resource sharing}\\
	In this case, the first class countries pay more attention to the protection of their own resources and environment, so they are not willing to open up the environmental resources sharing, we don't change the value of $ \alpha _1 $ (which is 0.2). We take $ \beta _1 $ as 1.5 and 1.3 to simulate the comprehensive competitiveness of the countries with different implementation of technology and economic support policies. Because the first class countries provide technological and economic support, the second class countries will sacrifice some of their own benefits. Therefore, the comprehensive development factor $ \beta _1 $ will decline.
	Based on the principles of sustainable development and environmental protection, the first class of countries may be less dependent on the resources from other countries. Meanwhile, considering that the human environment varies from one major power to another, we take $ \alpha _2 $ as 1 and 0.8 to simulate different realistic situations respectively. And the economic policy dividend and social development will increase $ \beta _2 $, so in this case we take $ \beta _2 $ as 1.2.
	By substituting the data into the differential equation of the FIF, we get the following results:
	\begin{figure}[H]
		\centering
		\includegraphics[width=.5\textwidth]{c2.png}
		\caption{Change in competitiveness of the second case}
		\label{img}
	\end{figure}
\vspace{-0.4cm}
    As can be seen from the figure, in this case, the comprehensive competitiveness of both types of countries eventually improved, which likewise avoids the creation of a resource monopoly situation in the field of asteroid mining. What surprises us is that when the first class countries are willing to sacrifice more economic benefits to support the space technology and economic development of the second class countries, which means that the first class countries have a higher human environment score and is less dependent on the resources of the second class countries, both classes of countries eventually have a higher comprehensive competitiveness than the original one, which is a mutually beneficial result. Therefore, we propose that if each country of the first class transfer part of its development benefits to the second class, both of them can obtain better development as well as the global equity on asteroid mining can be guaranteed.
\end{itemize}
\section{Policy and Schedule}
%\subsection{Correlation Analysis of Influencing Factors}
%The fact that only the major powers currently have the technology to go into outer space for exploration gives these powers the ability to operate in outer space with impunity and compete for asteroid mining resources.\upcite{14}. 
%The results of our model show that outer space has the tendency to be monopolized by these few major powers. 

%Out of the intention to promote global equity, we need to increase the FPC value of some countries so that the gap in comprehensive competitiveness among individual countries is not too big. At the same time, in order to ensure that all countries can benefit from asteroid mining, we need to maintain the Gini index at a stable value over time. Furthermore, since there is a large correlation between the indicators we selected, we considered altering some of them to achieve our purpose. 

%We can see that most of the indicators are mostly strongly correlated with each other, and with this as a reference, we can make more targeted suggestions for policies.
\subsection{Amendments of Existing Treaties}
As early as 1963, the United Nations General Assembly adopted the Outer Space Treaty, and in the following decade, various treaties on outer space were concluded in order to guarantee the equity of the interests of all mankind in outer space, but as mankind began to explore resources in outer space, various problems in these treaties began to emerge. If we want to maintain global equity, we need to update and improve these original treaties, and we will provide improved views on some of the treaties of outer space law below:
\begin{enumerate}
	\item The Outer Space Treaty, as a document of principle, does not have a high degree of enforceability. So we suggest that the UN could work to turn it into an enforceable treaty, calling on member states to impose severe sanctions on countries who violate it.
	\item The Outer Space Treaty requires all States parties to use the Moon and other celestial bodies for peaceful purposes by prohibiting military activities on them, but on the other hand it does not restrict access to space by military personnel. Therefore, the definition of the term "peaceful" is highly controversial among countries. So we suggest that the United Nations should further clarify the definition of peace and the extent of restrictions on military activities.
	\item The main content of the Registration Convention is to require countries to register the launched outer space objects with the United Nations. So that the United Nations and the rest of the world are aware of the launch of the space objects of each country.Therefore, the United Nations needs to strengthen its efforts to scrutinize the compliance of countries with the Registration Convention and avoid the selfish launch of spacecraft by certain countries who want to occupy space resources. 
	\item The Moon is the closest celestial body to Earth, which is rich in mineral resources. The Moon Agreement defines the Moon as the "common property of all mankind". However, this treaty did not enlist as much support as other treaties because it conflicted with the interests of certain countries. In order to make this treaty more widely supported and to balance the interests of the various Allied Powers for the benefit of all nations. We recommend that the UN revise the Moon Agreement from the perspective of outer space commercialization and lunar environmental protection.
\end{enumerate}
\subsection{Conclusion of New Treaties}
As mankind further accelerates its march into space. If only to amend on the basis of the original treaty, it will still leave a lot of legal gaps. So, it is important to allocate the finite resources efficiently and maintain the order of competition among countries.  In this case, the UN should conclude a new treaty to deepen international practice and international cooperation in the asteroid mining industry.
\begin{enumerate}
	\item In view of the current level of spaceflight, only some countries have access to outer space for a relatively long time to come. Therefore, we suggest that the UN should urge the first class countries to lift part of their technology blockade and transfer some technology to the second class countries, so as to enhance global equity.
	\item Some countries give up mining ores from their own countries and import them from other countries in order to secure their energy reserves. This move has significantly worsened the environment in exporting countries. Therefore, we suggest that the UN urges the importing countries to open up their mineral resources appropriately, besides allocating the resources available to each country in the asteroid mining.
	\item With the development of asteroid mining industry, some junk will be left behind in space. This will greatly harm the space environment, so we propose that the UN promote the environmental protection treaty for outer space, which requires every country who goes into outer space to take responsibility for cleaning up space junk.
	\item At present, the commercialization of the asteroid mining industry is already an inevitable trend. In order to prevent some countries from taking advantage of reaching space first to make undue profits and make global equity suffered. We suggest that the UN promotes the conclusion of a treaty on the regulation of space commerce, limiting the maximum amount of mining by certain countries and regulating the relevant market behavior.
\end{enumerate}
\subsection{Target Schedule}
Based on our various views on asteroid mining and recommendations for policy, we propose the following \textbf{target timeline} for the international community:
\begin{figure}[H]
	\centering
	\includegraphics[width=14cm, height=5cm]{nian.png}
	\caption{Target schedule}
	\label{img}
\end{figure}
\vspace{-0.4cm}
Earth's environment has been severely damaged at present, and we hope that we can improve Earth's enviroment greatly by developing the asteroid mining industry around 2050. At the same time, we hope that we can maintain the stability of the Gini Coefficient and basically reach global equity around 2070.
\newpage
\section{Sensitivity Analysis of the Model}
In measuring global equity, we determined the country's FPC by assigning weights to the four superior indicators to plot the Lorenz curve and calculate the GC. In order to test the reliability of our model's results, we partially shifted the weight of the indicator with the largest weight to the indicator with the smallest weight and observe the change in the Lorenz curve:
\begin{figure}[H]     %固定,容易留白,但可能串段
	\centering
	\begin{minipage}[t]{0.48\textwidth}
		\centering
		\includegraphics[width=8cm]{ling1.png}
	\end{minipage}
	\begin{minipage}[t]{0.48\textwidth}
		\centering
		\includegraphics[width=8cm]{ling2.png}
	\end{minipage}
    \caption{Sensitivity analysis}
\end{figure}
We can see that the Lorenz curve hardly changes after changing the weights, and only when we zoom in on the picture that the changes can be seen clearly. So the changes are very subtle. While the Lorenz curve gradually moves away from the theoretical fairness curve, the GC becomes larger, but its change is small, so we believe the results are reliable and the model is robust.
%8
\section{Model Evaluation and Further Discussion}
\subsection{Strengths}
Our research has following strengths:
\begin{enumerate}
	\item We used both AHP and EWM methods to determine the weights, which minimize the errors caused by subjectivity and the problems of the data itself.
	\item The data used in the model are from official websites and official journals to ensure the reliability of the results, and the findings are highly informative to the United Nations for policy formulation.
	\item The model is highly scalable. In measuring global equity and studying its changes, we considered a variety of factors and we can make a good assessment and prediction of global equity and its changes by simply providing relevant data.
	\item We use the Gini Coefficient to measure global equity, which makes our model highly scalable and can also be used to measure other equity issues between countries.
\end{enumerate}
\subsection{Weaknesses}
\begin{enumerate}
	\item The indicators we considered are not particularly complete. Because the issue of global equity in asteroid mining is complex, it involves political, economic, and technological aspects. It would be extremely difficult to examine all of them. If time permits, we can consider more indicators.
	\item It takes a long time for a policy to be implemented effectively, and it may be interfered by many factors. Therefore, it is difficult to apply the theoretical model to realistic situations.
	\item When processing the data, we introduced some errors into the model when rounding the number of decimal places, and such errors may cause a relatively large perturbation in the final results.
\end{enumerate}
\subsection{Further Discussion}
The premise of changing the parameters of our differential equation is that countries have strong enforcement power for new international treaties, but in reality, this condition is difficult to achieve.In the future, we can refer to the opinions of more experts and add more indicators and parameters to make our model's result be more accurate.


The global equity of asteroid mining is closely related to the future development of humanity, and our model has told us that powerful countries must take more responsibility, fulfill more obligations, and promote common global development, only then can we have a more peaceful future and asteroid mining can truly benefit humanity.

%\begin{table}[H]
%	\caption{Change makers in every genre}
%	\setlength{\tabcolsep}{11mm}{
%		\begin{tabular}{|c|c|c|c|c|}
%			\cline{1-2} \cline{4-5}
%			& p value &  &     & p value \\ \cline{1-2} \cline{4-5} 
%			GDP  & 0.849   &  & ED  & 0.693   \\ \cline{1-2} \cline{4-5} 
%			GDPG & 0.245   &  & NoE & 0.252   \\ \cline{1-2} \cline{4-5} 
%			Pop  & 0.056   &  & RC  & 0.636   \\ \cline{1-2} \cline{4-5} 
%			PCDI & 0.971   &  & FA  & 0.979   \\ \cline{1-2} \cline{4-5} 
%	\end{tabular}}
%\end{table}


%\begin{minipage}{\textwidth}
%	\begin{minipage}[t]{0.45\textwidth}
%		\centering
%		\makeatletter\def\@captype{table}\makeatother\caption{$ p $-values corresponding \\ \centerline{to the first 4 factors}}
%		\begin{tabular}{|c|c|}
%				\hline
%				%\rowcolor[HTML]{E2EFDA} 
%				   & $ p $-value \\ \hline
%				GDP  & 0.849   \\ \hline
%				GDPG & 0.245   \\ \hline
%				Pop  & 0.056   \\ \hline
%				PCDI & 0.971   \\ \hline
%		\end{tabular}
%	\end{minipage}
%	\begin{minipage}[t]{0.45\textwidth}
%		\centering
%		\makeatletter\def\@captype{table}\makeatother\caption{$ p $-values corresponding \\ \centerline{to the last 4 factors}}    
%		\begin{tabular}{|c|c|}
%				\hline
%				%\rowcolor[HTML]{E2EFDA} 
%				& $ p $-value \\ \hline
%				ED  & 0.693   \\ \hline
%				NoE & 0.252   \\ \hline
%				RC  & 0.636   \\ \hline
%				FA  & 0.979   \\ \hline
%		\end{tabular}
%	\end{minipage}
%\end{minipage}






%The influence of nodes in the network has been studied for a long time. Many scholars such as singla and Goyal\upcite{1} define the influence as the similarity and correlation between nodes. However, based on the data given in influence date, we can not understand the similarity between artists. The only similarity is whether these artists belong to the same school, that is, whether these nodes belong to the same community. Therefore, we introduce the twitterrank model proposed by Jianshu Weng:



%\begin{table}[H]
%	\centering
%	\caption{Characteristics of each genre}
%	\begin{tabular}{|c|c|c|}
%		\hline
%		Genre           & \multicolumn{2}{c|}{Characteristic} \\ \hline
%		Pop/Rock        & energy           & valence          \\ \hline
%		Vocal           & danceability     & acousticness     \\ \hline
%		International   & valence          & acousticness     \\ \hline
%		R\&B            & valence          & key              \\ \hline
%		Jazz            & acousticness     & instrumentalness \\ \hline
%		Comedy/Spoken   & speechiness      & explicit         \\ \hline
%		Latin           & energy           & valence          \\ \hline
%		Country         & valence          & mode             \\ \hline
%		Blues           & energy           & valence          \\ \hline
%		Children's      & danceability     & instrumentalness \\ \hline
%		Electronic      & instrumentalness & duration         \\ \hline
%		Classical       & instrumentalness & explicit         \\ \hline
%		Stage \& Screen & acousticness     & instrumentalness \\ \hline
%		Folk            & key              & acousticness     \\ \hline
%		Avant-Garde     & loudness         & instrumentalness \\ \hline
%		Religious       & energy           & tempo            \\ \hline
%		Reggae          & mode             & instrumentalness \\ \hline
%		New Age         & instrumentalness & duration         \\ \hline
%		Easy Listening  & loudness         & instrumentalness \\ \hline
%	\end{tabular}
%\end{table}
%
%
%\begin{table}[H]
%    \centering
%    \caption{the table of influence and simarity}
%    \begin{tabular}{lcc}
%    \hline
%    \hline
%                                          & \textbf{influence} & \textbf{simarity}           \\ \hline
%    \textbf{The Beatles}                  & 0.7864             & 0.8429                      \\
%    \textbf{John McLaughlin}              & 0.6794             & 0.7627                      \\
%    \textbf{Miles Davis}                  & 0.4376            & -0.3557 \\
%    \textbf{Jimi Hendrix}                 & 0.3467            & 0.2738                      \\
%    \textbf{Jim Hall}                     & 0.2467             & -0.1849 \\
%    \textbf{Julian Bream}                 & 0.1349            & 0.0705                      \\ \hline
%    \hline
%    \end{tabular}
%\end{table}
%
%
%
%
%
%
%\begin{table}[H]
%    \centering 
%    \caption{Similarity and influence of some Folk genre artists}
%    \begin{tabular}{ccc}
%    \hline
%    \hline
%    \multicolumn{1}{c|}{Name}                       & \multicolumn{1}{l}{simarity} & influence \\ \hline
%    \multicolumn{1}{c|}{The Kingston Trio}     & 0.948                        & 8.96      \\
%    \multicolumn{1}{c|}{Ramblin' Jack Elliott} & 0.903                        & 10.2      \\
%    \multicolumn{1}{c|}{John Fahey}            & 0.877                        & 9.31      \\
%    \multicolumn{1}{c|}{Joan Baez}             & 0.702                        & 15.5      \\
%    \multicolumn{1}{c|}{Dave Van Ronk}         & 0.852                        & 9.74      \\ \hline
%    \hline
%    \end{tabular}
%\end{table}



%The history of the development of the Folk genre\upcite{2}.
% shows that in the 1960s, Joan Baez combined the Folk genre with the Rock genre, introducing rock instruments such as the guitar to the Folk genre, thus driving a significant increase in the Folk genre's instrumentalness characteristics, further supporting our speculation.

%We used the same method to filter out other genre changers and the results are shown in the following table:
%
%\begin{table}[H]
%	\centering
%	\caption{Change makers in every genre}
%	\begin{tabular}{|l|l|l|l|l|}
%		\hline
%		Genre & Avant-Garde          & Blues           & Children's      & Classical      \\ \hline
%		Name  & Terry Riley          & Muddy Waters    & Alvin           & John Cage      \\ \hline
%		Genre & Latin                & New Age         & Pop/Rock        & R\&B;          \\ \hline
%		Name  & Antnio Carlos Jobim  & Mike Oldfield   & The Beatles     & Marvin Gaye    \\ \hline
%		Genre & Comedy/Spoken        & Country         & Easy Listening  & Electronic     \\ \hline
%		Name  & Spike Jones          & Johnny Cash     & Henry Mancini   & Kraftwerk      \\ \hline
%		Genre & Reggae               & Religious       & Stage \& Screen & Vocal          \\ \hline
%		Name  & Toots \& the Maytals & James Cleveland & Ennio Morricone & Billie Holiday \\ \hline
%		Genre & International        & Easy Listening  & Jazz            &                \\ \hline
%		Name  & Ravi Shankar         & loudness        & Miles Davis     &                \\ \hline
%	\end{tabular}
%	\label{tab:my-table}
%\end{table}
%
%
%\begin{figure}[htbp]
%	\centering
%	\begin{minipage}[t]{0.48\textwidth}
%		\centering
%		\includegraphics[width=8cm]{shijian.png}
%		\caption{vocal-organization}
%	\end{minipage}
%	\begin{minipage}[t]{0.48\textwidth}
%		\centering
%		\includegraphics[width=7cm]{f of vocal.png}
%		\caption{$\psi(t)$ of every year}
%	\end{minipage}
%\end{figure}







\clearpage
\begin{thebibliography}{99}
	\bibitem{1} The Treaty on Principles Governing the Activities of States in the Exploration and Use of Outer Space, including the Moon and other Celestial Bodies, of 27 January 1967, United Nations RES 2222 (XXI).
	\bibitem{2} Yan-Ming Liang $ \& $ Jia-Li Zhao. (2021). A comprehensive evaluation study of agricultural economy in southwest China - based on entropy weight TOPSIS method. Science, Technology and Industry (11), 227-232. doi:CNKI:SUN:CYYK.0.2021-11-040.
	\bibitem{3} Guo JY, Zhang ZB $ \& $ Sun QY. (2008). Research and application of hierarchical analysis method. Chinese Journal of Safety Science (05), 148-153. doi:10.16265/j.cnki.issn1003-3033.2008.05.018.
	\bibitem{4} Guo Q, Wang S, Li XH, Ma H, Wang S, Xu LW $ \& $ Li YY. (2006). Application of Lorenz curve and Gini Coefficient in the evaluation of fairness of community health service resource allocation. China Health Economics (01), 50-53. doi.
	\bibitem{5} https://baike.baidu.com/item/\%E6\%B4\%9B\%E4\%BC\%A6\%E5\%85\%B9\%E6\%9B\\\%B2\%E7\%BA\%BF/2903864?fr=aladdin
	\bibitem{6} Hu, Z. K.. (2004). A study on the theoretical optimal value of Gini Coefficient and its simple calculation formula. Economic Research (09), 60-69. doi:
	\bibitem{7} Wang Guoyu $ \& $ Tao Yangzi. (2015). Analysis of the U.S. Outer Space Resources Exploration and Utilization Act of 2015 and Suggestions for Response. China Space (12), 21-25. doi:
	\bibitem{8} https://en.wikipedia.org/wiki/Moon\_Treaty
	\bibitem{9} Jiang X. (2006). The origin and challenges of Outer Space Treaty. Theoretical World (03), 114-116. doi:
	\bibitem{10} Wang Guoyu. (2016). Pulling the plug on the outer space mining race? --A legal policy analysis of U.S. planetary mining legislation. Space International (05), 12-21. doi:
	\bibitem{11} Li, Shouping. (2013). The commercial use of outer space and the response in China. Journal of Beijing University of Technology (Social Science Edition) (01), 99-106. doi:10.15918/j.jbitss1009-3370.2013.01.015.
	\bibitem{12} Zhao, Yun. (2010). A review of the hot issues in Outer Space Treaty. Journal of Beijing University of Aeronautics and Astronautics (Social Science Edition) (01), 42-48. doi:10.13766/j.bhsk.1008-2204.2010.01.006.
	\bibitem{13} Bo Shoushou. (2009). The current status of the five major treaties on Outer Space Treaty and the challenges they face. Journal of Beijing University of Aeronautics and Astronautics (Social Science Edition) (01), 38-42. doi:10.13766/j.bhsk.1008-2204.2009.01.003.
	%\bibitem{14} Wang, Kongxiang. (2005). The challenge of space arms race to outer space law. Journal of Wuhan University (Philosophy and Social Science Edition) (03), 386-391. doi:
\end{thebibliography}

\end{document}  % 结束$
%http://www.360doc.com/content/14/0505/10/15883912_374732455.shtml
%Task5里的参考文献



%无序和有序排列
%\clearpage
%\begin{itemize}
%	\item \textbf{Impact of the World War II}
%\end{itemize}
%
%In the 1940s, World War II broke out, and because the war had a melancholy color, the works created in the 1940s were also more negative (less valence) and less suitable for dance, resulting in less danceability. At the same time, Europe was caught in the middle of the war, and a large number of European artists immigrated to the United States for refuge. After World War II, the music artists who immigrated from Europe and the native American music artists influenced each other, which made the music develop well. Coupled with the relatively peaceful international environment after World War II, people had more energy and leisure time to enjoy music, and the popularity of music had a great development after World War II\upcite{3}
%.
%
%\begin{itemize}
%	\item \textbf{The impact of information technology}
%\end{itemize}
%
%The emergence of the Internet in the 1970s and its subsequent development diversified the vehicles and media for disseminating musical works and made it cheaper and more convenient, and the popularity of music continued to increase. At the same time, however, offline musiac was more costly and therefore accounted for less than the ease and affordability of distributing music online, thus reducing the liveness characteristics of music.
%
%\begin{itemize}
%	\item \textbf{The impact of politics}
%\end{itemize}
%
%During the anti-war and civil rights movements in the United States in the 1960s, rock and roll made a huge impact on mainstream culture as a powerful weapon for youth subculture groups. During these movements, rock music was loved by many people. The danceability of the music increased due to the passionate rhythm of rock and roll, which made people dance to the music involuntarily\upcite{4}
%
%\begin{itemize}
%	\item \textbf{The impact of the economy}
%\end{itemize}
%
%In the 1930s, the economic crisis broke out across the United States, and the unemployment rate rose sharply in the face of closures and bankruptcies brought about by the Great Depression. The record industry at the time was hit hard. The music industry was hit hard and the popularity of music declined from the 1930s to the 1940s\upcite{5}
%
%
%\section{Strengths and Weaknesses}
%\begin{itemize}
%	\item Strengths:
%\end{itemize}
%
%1.It is innovative to construct influence models through networks.
%
%2.The selection of the network parameters of the influence network is scientific and reasonable.
%
%\begin{itemize}
%	\item Weaknessess:
%\end{itemize}
%
%1.Because of the large amount of data, our model cannot analyze each genre thoroughly.
%
%2.The indicators of the popularization model cannot be quantified, resulting in a low persuasibility of the model.
%
%3.As time changes, artists in one genre may create music in other genre styles, and we cannot consider this situation.
%
%4.We did not consider the impact that a collaboration between artists would have on the work.