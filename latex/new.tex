% ============================================================================
% MCM 2026 Problem C: Dancing with the Stars 计分系统分析
% ============================================================================
%
% 论文结构概览(共25页,无附录)
% ----------------------------------------------------------------------------
%
% 1. Summary (摘要) - 1页
%    - 问题概述:DWTS节目的计分系统与观众投票分析
%    - 核心方法:贝叶斯MCMC投票估计、敏感性分析、混合效应回归
%    - 主要发现:两种计分方法的差异、争议案例解释、新系统设计
%
% 2. Introduction (引言) - 2页
%    - 背景:Dancing with the Stars 节目机制介绍
%    - 问题动机:计分公平性、观众参与度、争议案例
%    - 方法概述
%
% 3. Assumptions and Notations (假设与符号) - 1页
%    - 投票行为假设(理性投票、无大规模刷票)
%    - 数据完整性假设
%    - 符号定义表
%
% 4. Data Preprocessing (数据预处理) - 1.5页
%    - 数据结构描述(34季、选手信息、周度分数)
%    - 缺失值处理策略
%    - 计分规则变化说明(排名法 vs 百分比法)
%
% 5. Task 1: Vote Estimation Model (投票预估模型) - 4页
%    - 问题定义:从淘汰结果逆向推断投票
%    - 方法:特征增强自适应贝叶斯投票估计 (FABVE)
%    - 验证:淘汰预测准确率、不确定性量化
%
% 6. Task 2: Scoring Method Comparison (计分方法比较) - 3页
%    - 排名法 vs 百分比法数学定义
%    - 指标:翻盘率、隐含权重、Kendall相关
%    - 结论:哪种方法更偏向评委/观众
%
% 7. Task 3: Controversy Analysis (争议案例分析) - 2.5页
%    - 案例:Jerry Rice, Billy Ray Cyrus, Bristol Palin, Bobby Bones
%    - 方法:反事实分析
%    - 结论:计分方法对争议的影响
%
% 8. Task 4: Method and Mechanism Discussion (方法与机制讨论) - 2页
%    - 两种计分方法的优劣对比
%    - Season 28+ 评委淘汰机制分析
%    - 综合建议
%
% 9. Task 5: Feature Impact Analysis (特征影响分析) - 3页
%    - 舞伴效应(固定效应模型)
%    - 选手特征(年龄、职业、地区)
%    - 方法:混合效应回归 / 随机森林
%
% 10. Task 6: New Scoring System Design (新计分系统设计) - 3页
%     - 设计目标:公平性、参与度、戏剧性、稳定性
%     - 提议:动态加权混合法 + 进步奖励
%     - 历史数据回测评估
%
% 11. Sensitivity Analysis (敏感性分析) - 1页
%     - 投票估计的不确定性传播
%     - 新系统参数敏感性
%
% 12. Strengths and Weaknesses (优缺点) - 0.5页
%
% 13. Conclusions (结论) - 0.5页
%
% 14. Memo to Producers (给制作人的备忘录) - 2页
%     - 执行摘要
%     - 关键发现
%     - 具体建议
%     - 实施考虑
%
% References (参考文献) - 1页
%
% 【页数汇总】
%   摘要1 + 引言2 + 假设1 + 数据1.5 + T1(4) + T2(3) + T3(2.5) + T4(2)
%   + T5(3) + T6(3) + 敏感性1 + 优缺点0.5 + 结论0.5 + 备忘录2 + 参考1 = 25页
%
% ============================================================================

\documentclass[12pt]{article}  % 官方要求字号不小于12号

% === 基础宏包 ===
\usepackage[table]{xcolor}
\usepackage[2622678]{easymcm}  % 队伍控制号,需要时修改
\problem{C}  % 题号

% === 字体设置 ===
\usepackage{palatino}  % COMAP官方杂志采用的Palatino字体

% === 表格与图片 ===
\usepackage{longtable}
\usepackage{tabu}
\usepackage{threeparttable}
\usepackage{float}
\usepackage{graphicx}
\usepackage{subfigure}

% === 代码展示 ===
\usepackage{listings}

% === 列表紧凑化 ===
\usepackage{paralist}
\let\itemize\compactitem
\let\enditemize\endcompactitem

% === 其他 ===
\usepackage{setspace}
\usepackage[normalem]{ulem}
\useunder{\uline}{\ul}{}

% === 引用命令 ===
\newcommand{\upcite}[1]{\textsuperscript{\textsuperscript{\cite{#1}}}}

% === 标题 ===
\title{Dancing with the Data: A Comprehensive Analysis of Scoring Systems}

% ============================================================================
% 文档开始
% ============================================================================
\begin{document}

% ----------------------------------------------------------------------------
% 摘要
% ----------------------------------------------------------------------------
\begin{abstract}

% TODO: 撰写摘要
% 结构建议:
% 1. 问题背景(1-2句)
% 2. Task 1 投票估计方法与结果
% 3. Task 2-3 计分方法比较与争议分析
% 4. Task 4-5 机制讨论与特征影响
% 5. Task 6 新系统设计
% 6. 主要结论

\vspace{5pt}
\textbf{Keywords}: Voting estimation; Scoring system; Bayesian inference; Sensitivity analysis; Mixed-effects model

\end{abstract}

\maketitle  % 生成 Summary Sheet
\tableofcontents
\newpage

% ----------------------------------------------------------------------------
% 第1章:引言
% ----------------------------------------------------------------------------
\section{Introduction}

\subsection{Background}
% TODO: 介绍DWTS节目背景
% - 节目历史(2005年至今,34季)
% - 计分机制(评委+观众投票)
% - 计分规则变化(排名法、百分比法、评委淘汰机制)

\subsection{Problem Restatement}
% TODO: 问题重述
% - 6个子任务概述
% - 核心挑战:投票数据不公开

\subsection{Our Approach}
% TODO: 方法概述
% - 整体建模思路
% - 各任务方法简述

% ----------------------------------------------------------------------------
% 第2章:假设与符号说明
% ----------------------------------------------------------------------------
\section{Assumptions and Notations}

\subsection{Assumptions}
% ============================================================================
% 假设清单(共7条)
% ============================================================================
%
% 【假设1】投票行为理性假设
%   - 观众投票反映其真实偏好,不存在大规模有组织的刷票行为
%   - 理由:节目有投票验证机制,且历史上未曝出大规模刷票丑闻
%
% 【假设2】评委评分独立性假设
%   - 各评委独立打分,不受其他评委或观众反应影响
%   - 理由:评委同时亮分,且为专业舞蹈从业者,有独立判断能力
%
% 【假设3】淘汰规则严格执行假设
%   - 每周淘汰结果严格按照公布的计分规则执行(综合得分最低者淘汰)
%   - 理由:作为公开电视节目,规则透明且受监督
%
% 【假设4】投票与评委分数相关性假设
%   - 观众投票与评委分数存在正相关(技术好的选手倾向于获得更多投票)
%   - 理由:观众能看到评委打分,会受其影响;同时好的表演也会吸引投票
%   - 用途:作为贝叶斯先验的基础
%
% 【假设5】选手知名度可由职业类别代理假设
%   - 选手的初始知名度可通过其职业背景(演员、运动员、歌手等)近似估计
%   - 理由:无法获取每位选手的社交媒体粉丝数等外部数据
%   - 用途:构建投票先验分布
%
% 【假设6】缺失评委分数可插补假设
%   - 部分周次仅有3位评委(无第4位),可用已有评委平均分代替
%   - 理由:第4位评委的加入是后期变化,不影响整体评分趋势
%
% 【假设7】季节间可比性假设
%   - 不同季节的评委评分标准大致一致,可以跨季比较
%   - 理由:核心评委(如Len Goodman)长期稳定,评分尺度相对一致
%   - 限制:仍需加入季节固定效应控制异质性
%
% ============================================================================

\subsection{Notations}
% ============================================================================
% 符号表
% ============================================================================
%
% 【基础符号】
%   n          - 当周参赛选手数量
%   i          - 选手索引,i = 1, 2, ..., n
%   t, w       - 周次(week number),t = 1, 2, ..., 11
%   s          - 季数(season),s = 1, 2, ..., 34
%
% 【评委相关】
%   S_i        - 选手i的评委总分(4位评委分数之和,满分40)
%   S_{i,j}    - 选手i第j位评委的分数(1-10分)
%   R_i^S      - 选手i按评委分数的排名(1为最高分)
%   q_i        - 选手i的评委分数百分比,q_i = S_i / ΣS_j
%
% 【投票相关】
%   V_i        - 选手i的观众投票数(未知,需估计)
%   R_i^V      - 选手i按观众投票的排名
%   p_i        - 选手i的投票百分比,p_i = V_i / ΣV_j
%
% 【综合得分】
%   C_i^{rank} - 排名法综合得分:C_i = R_i^S + R_i^V (越小越好)
%   C_i^{pct}  - 百分比法综合得分:C_i = q_i + p_i (越大越好)
%
% 【淘汰相关】
%   E          - 被淘汰选手的索引
%   I(·)       - 指示函数
%
% 【模型参数(通用)】
%   α, β, γ    - 回归模型系数
%   σ²         - 方差参数
%   θ          - 一般参数向量
%
% 【Task 1: 贝叶斯投票估计参数】
%   π          - 投票份额向量,π_i ∈ [0,1],Σπ_i = 1
%   α          - Dirichlet先验强度参数
%   β          - 先验关联系数(评委分数与投票的相关程度)
%   θ          - 先验权重向量
%   f_i^feature- 选手i的特征权重(职业、年龄等)
%   τ          - softmax温度参数
%   τ_0        - 基础温度
%   τ_eff      - 有效温度(动态调整后)
%   λ          - MCMC提议步长
%   λ_0        - 初始步长
%   N          - 后验样本数
%   B          - 预热(burn-in)迭代数
%   X          - 选手特征矩阵
%   a_i        - 选手i的年龄
%
% 【验证指标】
%   EPA        - 淘汰预测准确率 (Elimination Prediction Accuracy)
%   W          - 分析的总周数
%   P_elim     - 模型对淘汰结果的置信概率
%   τ_kendall  - Kendall秩相关系数
%   CV         - 变异系数 (Coefficient of Variation)
%
% ============================================================================

% 符号表格
Here we clarify some key symbols. Others will be explained in context.
\vspace{-0.4cm}
\begin{table}[H]
\centering
\caption{Notations}
\begin{tabular}{cl}
\toprule[1.5pt]
\textbf{Symbol} & \textbf{Description} \\
\midrule
$S_i$ & Total judges' score for contestant $i$ \\
$\boldsymbol{\pi}$ & Vote share vector, $\pi_i \in [0,1]$, $\sum_i \pi_i = 1$ \\
$C_i^{\text{rank}}$ & Combined score under ranking method \\
$C_i^{\text{pct}}$ & Combined score under percentage method \\
\bottomrule[1.5pt]
\end{tabular}
\end{table}
\vspace{-0.4cm}

% ----------------------------------------------------------------------------
% 第3章:数据预处理
% ----------------------------------------------------------------------------
\section{Data Preprocessing}

\subsection{Data Overview}
% ============================================================================
% 数据概览
% ============================================================================
%
% 【数据来源】
%   - 官方提供的 2026_MCM_Problem_C_Data.csv
%   - 涵盖 Dancing with the Stars 第1-34季全部选手数据
%
% 【数据规模】
%   - 总记录数:约400+条(每位参赛选手一条记录)
%   - 季数跨度:34季(2005-2024年)
%   - 每季选手数:6-15人不等
%
% 【字段说明】
%
%   选手信息类:
%   - celebrity_name          : 明星选手姓名
%   - celebrity_industry      : 职业背景(Actor/Actress, Athlete, Singer, Model, TV等)
%   - celebrity_homestate     : 家乡州(美国选手)
%   - celebrity_homecountry   : 国籍/地区
%   - celebrity_age_during_season : 参赛时年龄
%
%   舞伴信息类:
%   - ballroom_partner        : 专业舞伴姓名(如Derek Hough, Mark Ballas等)
%
%   比赛结果类:
%   - season                  : 季数(1-34)
%   - results                 : 结果描述("1st Place", "Eliminated Week 3"等)
%   - placement               : 最终排名(数字,1为冠军)
%
%   评委分数类:
%   - week{1-11}_judge{1-4}_score : 第X周第Y位评委的分数
%   - 分数范围:1-10分(可含0.5分小数)
%   - 特殊值:0表示该周已被淘汰,N/A表示该评委该周未参与
%
% 【基本统计量】(待数据分析后填入)
%   - 平均参赛周数:约X周
%   - 平均每周评委总分:约X分
%   - 职业分布:运动员X%、演员X%、歌手X%...
%
% ============================================================================

\subsection{Scoring Rule Evolution}
\label{sec:scoring-evolution}
% ============================================================================
% 计分规则演变(关键!三个阶段)
% ============================================================================
%
% 【阶段1:排名法(Ranking Method)】
%   适用季数:Season 1-2, Season 28-34
%
%   计算方式:
%     Combined Score = Judge_Rank + Audience_Rank
%     淘汰规则:综合排名分数最高者淘汰(排名数字越大越差)
%
%   特点:
%     - 评委和观众各占50%权重(均为排名)
%     - 对极端分数不敏感(只看相对排名)
%     - 有利于"虽然分数低但粉丝多"的选手
%
%   Season 28+ 附加机制:
%     - 综合排名最后两名进入"底部二人区"
%     - 由评委投票决定两人中谁被淘汰
%     - 这给了评委"最终决定权",可纠正观众投票的"不公"
%
% 【阶段2:百分比法(Percentage Method)】
%   适用季数:Season 3-27
%
%   计算方式:
%     Judge_Pct = S_i / Σ S_j  (评委分数占总分比例)
%     Vote_Pct  = V_i / Σ V_j  (投票占总投票比例)
%     Combined Score = Judge_Pct + Vote_Pct
%     淘汰规则:综合百分比最低者淘汰
%
%   特点:
%     - 评委和观众仍各占约50%(但受分数分布影响)
%     - 对极端分数敏感(高分选手的评委百分比优势大)
%     - 技术优秀者更有优势
%
% 【规则变化时间线】
%   Season 1-2   (2005-2006): 排名法
%   Season 3-27  (2006-2018): 百分比法
%   Season 28-34 (2019-2024): 排名法 + 评委淘汰机制
%
% 【为什么要变化?】
%   - 百分比法时期出现多次争议(评委最低分选手获胜)
%   - 如 Bobby Bones (S27冠军) 多次评委垫底但靠粉丝投票晋级
%   - 制作方希望增加评委话语权,引入评委淘汰机制
%
% ============================================================================

\subsection{Data Cleaning}
% ============================================================================
% 数据清洗策略
% ============================================================================
%
% 【问题1:缺失的第4位评委分数】
%   现象:早期季节(约S1-S17)只有3位评委,第4位评委列为N/A
%   处理:
%     方案A:仅使用前3位评委分数求和(不推荐,跨季不可比)
%     方案B:计算平均分后乘以4,统一为40分制(推荐)
%     方案C:分阶段建模,3评委和4评委季节分开处理
%
% 【问题2:淘汰后的0分记录】
%   现象:选手被淘汰后,后续周次分数全为0
%   处理:
%     - 这是正常的"结构性缺失",不是错误
%     - 在周度分析中:仅使用该选手被淘汰前的数据
%     - 在存活分析中:标记淘汰周次作为事件时间
%
% 【问题3:特殊值"N/A"】
%   现象:部分单元格为字符串"N/A"
%   处理:
%     - 转换为数值时识别并处理
%     - 区分"评委缺席的N/A"和"选手淘汰的0"
%
% 【问题4:分数格式不一致】
%   现象:部分分数为小数(如8.5),部分为整数
%   处理:统一转换为浮点数,保留一位小数
%
% 【问题5:选手信息缺失】
%   现象:部分选手的homestate为空(非美国选手)
%   处理:
%     - 创建 is_us 二值变量标识美国选手
%     - homestate缺失时填充为"Non-US"或单独类别
%
% 【问题6:职业类别归并】
%   现象:celebrity_industry有多种取值,部分类别样本量小
%   处理:
%     - 保留主要类别:Actor/Actress, Athlete, Singer, Model, TV Personality
%     - 小类别合并为"Other"
%     - 或使用职业大类(Entertainment, Sports, Other)
%
% 【数据清洗流程伪代码】
%   1. 读取CSV,处理编码(UTF-8 BOM)
%   2. 识别并转换N/A为NaN
%   3. 转换分数列为数值类型
%   4. 计算每周评委总分(处理3/4评委差异)
%   5. 标记有效比赛周(分数>0且非NaN)
%   6. 创建派生变量(is_us, industry_grouped等)
%   7. 验证数据完整性(每季应有placement从1到n的完整排名)
%
% ============================================================================
% - 缺失值处理
% - 异常值检测

% ----------------------------------------------------------------------------
% 第4章:Task 1 - 投票预估模型
% ----------------------------------------------------------------------------
%
% ============================================================================
% 模型名称: Feature-Enhanced Adaptive Bayesian Vote Estimation (FABVE)
%           特征增强自适应贝叶斯投票估计模型
% ============================================================================
%
% 核心挑战:观众投票数据从未公开,只能观察评委分数和淘汰结果
% 建模思路:贝叶斯推断 + 软约束 + 特征增强先验 + 自适应MCMC
%
% ============================================================================

\section{Task 1: Vote Estimation Model}

% ============================================================================
% 4.1 问题形式化
% ============================================================================

\subsection{Problem Formulation}

% 在《与星共舞》节目中,每周的淘汰由评委分数和观众投票共同决定。
In \textit{Dancing with the Stars}, weekly eliminations are determined by a combination
of judges' scores and audience votes.
% 然而,在节目34季的历史中,实际的观众投票数据从未被公开披露。
However, the show has never disclosed the actual
audience voting data throughout its 34-season history.
% 这构成了一个经典的逆问题:我们必须从可观测的淘汰结果反推未观测到的投票分布。
This constitutes a classic
\textbf{inverse problem}: we must infer the unobserved voting distribution from
the observable elimination outcomes.

% 形式化地,对于每季s的每周w,我们观察到:
Formally, for each week $w$ in season $s$, we observe:
\begin{itemize}
    % 评委分数向量S,其中每个分数在4到40分之间
    \item Judge scores $\mathbf{S} = (S_1, S_2, \ldots, S_n)$, where $S_i \in [4, 40]$
    % 淘汰结果E,即被淘汰选手的索引
    \item Elimination result $E \in \{1, 2, \ldots, n\}$ (the index of the eliminated contestant)
\end{itemize}

% 我们的目标是估计未观测到的投票份额分布π
Our goal is to estimate the unobserved vote share distribution
$\boldsymbol{\pi} = (\pi_1, \ldots, \pi_n)$,
% 其中π_i表示选手i获得的观众投票比例
where $\pi_i$ represents the fraction
of audience votes received by contestant $i$, satisfying:
% 约束条件:所有份额之和为1,且每个份额非负
\begin{equation}
\sum_{i=1}^{n} \pi_i = 1, \quad \pi_i \geq 0
\end{equation}

% 淘汰由节目的计分规则决定:
The elimination is governed by the show's scoring rules:

% 排名法(第1-2季,第28-34季):
\textbf{Ranking Method} (Seasons 1--2, 28--34):
% 综合排名 = 评委排名 + 投票排名,排名数最高者被淘汰
\begin{equation}
C_i^{\text{rank}} = R_i^{S} + R_i^{\pi}, \quad E = \arg\max_i C_i^{\text{rank}}
\end{equation}
% 其中R_i^S和R_i^π分别是选手i按评委分数和粉丝投票的排名
where $R_i^{S}$ and $R_i^{\pi}$ represent the contestant's ranks by judge scores and fan votes, respectively.

% 百分比法(第3-27季):
\textbf{Percentage Method} (Seasons 3--27):
% 综合得分 = 评委分数百分比 + 粉丝投票百分比,得分最低者被淘汰
\begin{equation}
C_i^{\text{pct}} = \frac{S_i}{\sum_j S_j} + \pi_i, \quad E = \arg\min_i C_i^{\text{pct}}
\end{equation}

% 从数学上看,给定淘汰结果E,存在无穷多个可能产生该结果的投票分布。

Mathematically, given an elimination outcome $E$, there exist infinitely many fan votes distributions that satisfy this result.
% 我们的任务是通过融入先验知识来识别最合理的分布,同时恰当地量化固有的不确定性。
Our task is to identify the most plausible distribution
by incorporating prior knowledge while properly quantifying the inherent uncertainty.

% ============================================================================
% 4.2 方法论
% ============================================================================

\subsection{Methodology: Bayesian Vote Estimation}

% 我们采用贝叶斯推断框架来解决这个逆问题。
We adopt a \textbf{Bayesian inference framework} to address this inverse problem.
% 该框架提供三个关键优势:
This framework offers three key advantages:
\begin{enumerate}
    % 1. 不确定性量化:提供完整的后验分布而非点估计
    \item \textbf{Uncertainty quantification}: Provides complete posterior distributions rather than point estimates
    % 2. 先验知识融合:融入关于投票行为的领域知识
    \item \textbf{Prior knowledge integration}: Incorporates domain knowledge about voting behavior
    % 3. 软约束:以概率方式使用淘汰信息,而非硬约束
    \item \textbf{Soft constraints}: Uses elimination information probabilistically rather than as hard constraints
\end{enumerate}

% 后验分布由贝叶斯定理给出:
The posterior distribution is given by Bayes' theorem:
% 后验 ∝ 似然 × 先验
\begin{equation}
P(\boldsymbol{\pi} | E, \mathbf{S}) \propto
  P(E | \boldsymbol{\pi}, \mathbf{S}) \cdot P(\boldsymbol{\pi} | \mathbf{S}, \mathbf{X})
\end{equation}
% 其中X表示选手特征(职业、年龄等)
where $\mathbf{X}$ denotes contestant features (profession, age, etc.).

% --- 先验分布 ---
\subsubsection{Prior Distribution}

% 我们使用Dirichlet分布作为先验:
We use a Dirichlet distribution as the prior:
\begin{equation}
\boldsymbol{\pi} \sim \text{Dirichlet}(\alpha \cdot \boldsymbol{\theta})
\end{equation}
% 其中α是控制集中度的先验强度参数,θ是先验权重向量
where $\alpha$ is the prior strength parameter controlling concentration,
and $\boldsymbol{\theta} = (\theta_1, \ldots, \theta_n)$ is the prior weight vector.

% 先验权重向量θ设计为加权混合:
The prior weight vector $\boldsymbol{\theta}$ is designed as a weighted mixture:
\begin{equation}
\theta_i = (1 - \beta) \cdot f_i^{\text{feature}} + \beta \cdot \tilde{S}_i
\end{equation}
% 其中β控制先验与评委分数的关联程度,f_i^feature是选手i的特征权重,S̃_i是归一化评委分数
where $\beta \in [0,1]$ controls the correlation between prior and judges' scores,
$f_i^{\text{feature}}$ is the feature-based weight for contestant $i$,
and $\tilde{S}_i = S_i / \sum_j S_j$ is the normalized judges' score.

% 基于特征的权重:f_i^feature项融入职业人气权重和年龄效应
\textbf{Feature-based weights}: The term $f_i^{\text{feature}}$ incorporates profession
popularity weights (e.g., singers and actors typically have larger fan bases) and age effects.
% 这些权重基于领域知识:某些职业天然拥有更大的粉丝群体
These weights are based on domain knowledge that certain professions inherently command
larger followings.
% 重要的是,敏感性分析表明模型结果对这些权重设置的变化具有鲁棒性
Importantly, our sensitivity analysis (Section~\ref{sec:sensitivity})
demonstrates that model results are robust to variations in these weight settings,
% 因为主要信息来自淘汰约束而非先验
as the primary information comes from elimination constraints rather than priors.

% 年龄效应:对25-35岁年龄段有轻微偏好(社交媒体参与度更高)
\textbf{Age effect}:
\begin{equation}
\text{age\_factor}_i = 1.0 - 0.01 \times |a_i - 30|
\end{equation}
% 其中a_i是选手i的年龄,结果限制在[0.7, 1.2]范围内
where $a_i$ is the age of contestant $i$, capped within $[0.7, 1.2]$.

% --- 似然函数 ---
\subsubsection{Likelihood Function}

% 我们采用softmax软约束而非硬约束:
We employ a softmax soft constraint instead of hard constraints:

% 百分比法的似然:
\textbf{For Percentage Method}:
% 综合得分低的选手被淘汰概率更高(指数形式)
\begin{equation}
P(E = k | \boldsymbol{\pi}, \mathbf{S}) =
  \frac{\exp(-\tau \cdot C_k^{\text{pct}})}{\sum_{j} \exp(-\tau \cdot C_j^{\text{pct}})}
\end{equation}

% 排名法的似然:
\textbf{For Ranking Method}:
% 综合排名高的选手被淘汰概率更高
\begin{equation}
P(E = k | \boldsymbol{\pi}, \mathbf{S}) =
  \frac{\exp(\tau \cdot C_k^{\text{rank}})}{\sum_{j} \exp(\tau \cdot C_j^{\text{rank}})}
\end{equation}

% 温度参数τ控制约束的严格程度:值越高,对淘汰结果的约束越严格
The temperature parameter $\tau$ controls constraint strictness:
higher values enforce stricter adherence to elimination outcomes.
% 我们使用动态温度:
We use dynamic temperature:
\begin{equation}
\tau_{\text{eff}} = \tau_0 \cdot (1 + 2 \cdot m)
\end{equation}
% 其中τ_0是基础温度,m衡量淘汰边界的清晰程度
where $\tau_0$ is the base temperature and $m$ measures the clarity of the elimination margin.


% ============================================================================
% 4.3 模型实现
% ============================================================================

\subsection{Model Implementation}

% 由于后验分布没有解析解,我们使用MCMC采样来近似它
Since the posterior distribution lacks an analytical solution, we approximate it
using Markov Chain Monte Carlo (MCMC) sampling.

% --- 单纯形上的提议分布 ---
\subsubsection{Proposal Distribution on the Simplex}

% 我们使用Dirichlet提议分布,它自然满足单纯形约束:
We use a Dirichlet proposal that naturally respects the simplex constraint:
\begin{equation}
\boldsymbol{\pi}' \sim \text{Dirichlet}\left(\frac{\boldsymbol{\pi}}{\lambda}\right)
\end{equation}
% 其中λ是控制探索幅度的提议步长
where $\lambda$ is the proposal step size controlling exploration magnitude.

% --- 自适应步长 ---
\subsubsection{Adaptive Step Size}

% 步长λ自适应调整以达到约30%的目标接受率
The step size $\lambda$ is adaptively adjusted to maintain an acceptance rate around 30\%.

% --- 超参数设置 ---
\subsubsection{Hyperparameter Settings}

% 默认超参数:τ_0=12, α=1.5, β=0.3, λ_0=0.06
The default hyperparameters are: $\tau_0=12$, $\alpha=1.5$, $\beta=0.3$, $\lambda_0=0.06$,
% 在3000次预热后采集10000个后验样本
with $N=10000$ posterior samples after $B=3000$ burn-in iterations.
% 这些值通过敏感性分析确定,以平衡模型性能和计算效率
These values were
determined through sensitivity analysis to balance model performance and computational
efficiency.
% 固定随机种子以确保可重复性
The random seed is fixed to ensure reproducibility.

% ============================================================================
% 4.4 结果验证
% ============================================================================

\subsection{Results and Validation}

% --- 一致性指标(Q1a)---
\subsubsection{Consistency Metrics (Q1a)}

% 淘汰预测准确率(EPA):预测正确的周数占总周数的比例
\textbf{Elimination Prediction Accuracy (EPA)}:
\begin{equation}
\text{EPA} = \frac{1}{W} \sum_{w=1}^{W} \mathbb{I}[\hat{E}_w = E_w]
\end{equation}

% 平均淘汰概率:模型对实际淘汰结果的平均置信度
\textbf{Mean Elimination Probability}:
\begin{equation}
\bar{P}_{\text{elim}} = \frac{1}{W} \sum_{w=1}^{W} P(E_w | \bar{\boldsymbol{\pi}}_w, \mathbf{S}_w)
\end{equation}

% 其他指标包括Top-2准确率和Kendall's τ秩相关系数
Additional metrics include Top-2 accuracy and Kendall's $\tau$ rank correlation.

% 表格总结了34季的一致性指标
Table~\ref{tab:consistency} summarizes the consistency metrics in all 34 seasons.

\begin{table}[htbp]
\centering
\caption{Consistency Metrics Summary (Q1a)}
% 一致性指标汇总表
\label{tab:consistency}
\begin{tabular}{lc}
\hline
\textbf{Metric} & \textbf{Value} \\
\hline
% 分析的总周数
Total weeks analyzed & 301 \\
% 淘汰预测准确率
Elimination Prediction Accuracy (EPA) & 93.7\% \\
% Top-2准确率(淘汰者在预测的后两名中)
Top-2 Accuracy & 97.7\% \\
% 平均淘汰概率
Mean Elimination Probability & 0.481 \\
% 平均Kendall's τ
Mean Kendall's $\tau$ & 0.717 \\
\hline
% 按计分方法分类:
\multicolumn{2}{l}{\textit{By Scoring Method:}} \\
% 排名法(74周)
\quad Ranking Method (n=74) & EPA = 95.9\% \\
% 百分比法(227周)
\quad Percentage Method (n=227) & EPA = 93.0\% \\
\hline
\end{tabular}
\end{table}

% --- 一致性结果分析 ---
% 结果表明模型与观测淘汰结果一致
The 93.7\% EPA indicates that our estimated vote distributions correctly predict
the actual elimination in most weeks.
% Top-2准确率97.7%:几乎所有情况下,实际被淘汰者都在模型预测的后两名中
The 97.7\% Top-2 accuracy shows that in nearly all cases, the eliminated contestant
was among the bottom two predicted by our model.
% 排名法EPA略高于百分比法,这与排名法的离散约束更明确有关
The Ranking Method achieves slightly higher EPA (95.9\%) than the Percentage Method (93.0\%),
likely due to its sharper discrete constraints.
% Kendall's τ为0.717表明预测排名与投票排名相关性较好
The Kendall's $\tau$ of 0.717 indicates reasonable rank correlation between predicted
and implied vote rankings.

% --- 确定性指标(Q1b)---
\subsubsection{Certainty Metrics (Q1b)}

% 变异系数(CV):标准差与均值的比值,衡量估计的相对不确定性
\textbf{Coefficient of Variation (CV)}:
\begin{equation}
\text{CV}_i = \frac{\sigma_{\pi_i}}{\mu_{\pi_i}}
\end{equation}

% 95%可信区间宽度:后验分布的2.5%和97.5%分位数之差
\textbf{95\% Credible Interval Width}:
\begin{equation}
\text{CI}_i = \pi_i^{97.5\%} - \pi_i^{2.5\%}
\end{equation}

% 我们还报告有效样本量(ESS)和MCMC接受率来评估采样质量
We also report effective sample size (ESS) and MCMC acceptance rates to assess
sampling quality.

% 表格总结了确定性指标
Table~\ref{tab:certainty} summarizes the certainty metrics.

\begin{table}[htbp]
\centering
\caption{Certainty Metrics Summary (Q1b)}
% 确定性指标汇总表
\label{tab:certainty}
\begin{tabular}{lc}
\hline
\textbf{Metric} & \textbf{Value} \\
\hline
% 所有选手的平均CV
Mean CV (all contestants) & 0.710 \\
% 平均95%可信区间宽度
Mean 95\% CI Width & 0.292 \\
\hline
% 按淘汰状态分类:
\multicolumn{2}{l}{\textit{By Elimination Status:}} \\
% 被淘汰选手的CV
\quad Eliminated contestants CV & 0.761 \\
% 未被淘汰选手的CV
\quad Non-eliminated contestants CV & 0.704 \\
% 被淘汰选手的CI宽度
\quad Eliminated contestants CI width & 0.167 \\
% 未被淘汰选手的CI宽度
\quad Non-eliminated contestants CI width & 0.308 \\
\hline
\end{tabular}
\end{table}

% --- 确定性结果分析 ---
% 平均CV为0.710,反映逆问题固有的不确定性
The mean CV of 0.710 reflects the inherent uncertainty of this inverse problem.
% 被淘汰者CI更窄,因为淘汰事件提供了强约束
Eliminated contestants have narrower CI widths (0.167 vs. 0.308) because
the elimination event constrains their vote share more tightly.


% ----------------------------------------------------------------------------
% 第5章:Task 2 - 计分方法比较
% ----------------------------------------------------------------------------
\section{Task 2: Scoring Method Comparison}

\subsection{Method Characteristics}

% 两种计分方法在如何结合评委分数和观众投票方面存在根本差异:
The two scoring methods differ fundamentally in how they combine judges' scores
and audience votes:

\begin{itemize}
    % 排名法:将分数和投票转换为名次后相加,消除数值大小信息
    \item \textbf{Ranking Method}: Converts scores and votes to ordinal ranks before
    combining, discarding numerical magnitude.
    % 百分比法:保留数值信息,直接按百分比相加
    \item \textbf{Percentage Method}: Preserves numerical information by combining
    score percentages directly.
\end{itemize}

% 这导致三个关键差异:
% (1) 排名法权重固定50%-50%,百分比法权重动态变化
% (2) 排名法仅在名次变化时响应,百分比法连续响应
% (3) 排名法对极端值具有鲁棒性
These lead to three key differences:
(1) The Ranking Method has fixed 50\%--50\% weight structure, while the Percentage
Method has dynamic effective weights;
(2) The Ranking Method responds only when ranks change, while the Percentage
Method responds continuously;
(3) The Ranking Method is robust to outliers.

\subsection{Metrics}

% 淘汰翻转率(EFR):切换计分方法时淘汰决定改变的周数比例
\textbf{Elimination Flip Rate (EFR)}: The fraction of weeks where switching methods
would change the elimination decision:
\begin{equation}
    \text{EFR} = \frac{1}{W}\sum_{w=1}^{W} \mathbf{1}[E^{\text{actual}}_w \neq E^{\text{cf}}_w]
\end{equation}

% 有效权重:各成分对综合得分的方差贡献比例
\textbf{Effective Weight}: The variance contribution to the combined score:
\begin{equation}
    \alpha^{\text{pct}} = \frac{\text{Var}(P^{(S)})}{\text{Var}(P^{(S)}) + \text{Var}(P^{(V)})}
\end{equation}
% 其中P^(S)和P^(V)分别是评委分数百分比和投票百分比
where $P^{(S)}$ and $P^{(V)}$ are the judge score and vote percentages.

\subsection{Results}

% 使用Task 1的投票估计,对34季301周数据进行反事实分析
Using the vote estimates from Task~1, we conduct counterfactual analysis on
301 weeks across 34 seasons.

% 反事实分析汇总表
\begin{table}[htbp]
\centering
\caption{Counterfactual Analysis Summary}
\label{tab:counterfactual-summary}
\begin{tabular}{lccc}
\hline
\textbf{Metric} & \textbf{Overall} & \textbf{Ranking Era} & \textbf{Percentage Era} \\
\hline
% 分析周数
Weeks analyzed & 301 & 74 & 227 \\
% 淘汰翻转次数
Elimination flips & 66 & 16 & 50 \\
% 翻转率
Flip Rate & 21.9\% & 21.6\% & 22.0\% \\
\hline
\end{tabular}
\end{table}

% 有效权重分析表
\begin{table}[htbp]
\centering
\caption{Effective Weight Analysis}
\label{tab:weight-decomposition}
\begin{tabular}{lcc}
\hline
\textbf{Metric} & \textbf{Ranking Method} & \textbf{Percentage Method} \\
\hline
% 评委有效权重
Judge effective weight & 48.2\% & 22.9\% \\
% 观众有效权重
Audience effective weight & 51.8\% & 77.1\% \\
\hline
\end{tabular}
\end{table}

% 有效权重分析表明:百分比法给予观众77.1%的有效权重,而排名法为51.8%
The effective weight analysis reveals that the Percentage Method assigns 77.1\%
effective weight to audience votes, compared to 51.8\% under the Ranking Method.
% 原因:投票分布通常比评委分数更分散,导致投票主导百分比综合得分
This occurs because vote distributions are typically more dispersed than judge
scores, causing votes to dominate the combined percentage score.

% 结论:百分比法比排名法更偏向粉丝投票
% 这解释了第3-27季的争议:低评委分数但高粉丝支持的选手能够晋级
% 节目在第28季回归排名法,旨在恢复评委与观众之间的平衡
\textbf{Conclusion}: The Percentage Method favors fan votes more than the Ranking
Method. This explains the controversies during Seasons 3--27, where contestants
with low judge scores but strong fan support could advance.
The show's return to the Ranking Method in Season 28 aimed to restore balance.

% ----------------------------------------------------------------------------
% 第6章:Task 3 - 争议案例分析
% ----------------------------------------------------------------------------
\section{Task 3: Controversy Case Analysis}

\subsection{Identifying Controversial Cases}
% TODO: 识别争议案例
% - Jerry Rice (Season 2)
% - Billy Ray Cyrus (Season 4)
% - Bristol Palin (Season 11)
% - Bobby Bones (Season 27)

\subsection{Counterfactual Analysis}
% TODO: 反事实分析
% - 如果使用另一种计分方法,结果如何变化

\subsection{Findings}
% TODO: 发现
% - 计分方法对争议的贡献

% ----------------------------------------------------------------------------
% 第7章:Task 4 - 方法与机制讨论
% ----------------------------------------------------------------------------
\section{Task 4: Discussion on Methods and Mechanisms}

\subsection{Pros and Cons of Each Method}
% TODO: 各方法优缺点

\subsection{Judge Elimination Mechanism}
% TODO: 评委淘汰机制分析(Season 28+)

\subsection{Recommendations}
% TODO: 综合建议

% ----------------------------------------------------------------------------
% 第8章:Task 5 - 特征影响分析
% ----------------------------------------------------------------------------
\section{Task 5: Feature Impact Analysis}

\subsection{Professional Partner Effect}
% TODO: 舞伴效应
% - 固定效应模型
% - 舞伴排名

\subsection{Celebrity Characteristics}
% TODO: 选手特征影响
% - 年龄
% - 职业背景
% - 地区

\subsection{Model Results}
% TODO: 模型结果
% - 回归系数
% - 特征重要性

% ----------------------------------------------------------------------------
% 第9章:Task 6 - 新计分系统设计
% ----------------------------------------------------------------------------
\section{Task 6: New Scoring System Design}

\subsection{Design Objectives}
% TODO: 设计目标
% - 公平性
% - 观众参与度
% - 戏剧性
% - 稳定性

\subsection{Proposed System: Dynamic Weighted Hybrid}
% TODO: 提议系统
% - 动态权重公式
% - 进步奖励机制

\subsection{Historical Backtesting}
% TODO: 历史回测
% - 评估指标
% - 与现有方法对比

\subsection{Recommendations}
% TODO: 建议

% ----------------------------------------------------------------------------
% 第10章:敏感性分析
% ----------------------------------------------------------------------------
\section{Sensitivity Analysis}
\label{sec:sensitivity}

To validate the robustness of our vote estimation model, we conduct comprehensive
sensitivity analysis on key hyperparameters.

\subsection{Hyperparameter Impact Analysis}

We systematically vary each hyperparameter while holding others at default values,
measuring the impact on Elimination Prediction Accuracy (EPA):

\begin{itemize}
    \item \textbf{Temperature $\tau$}: Controls the softness of elimination constraints.
          Tested range: $[3, 30]$. Default: $\tau_0 = 12$.
    \item \textbf{Prior strength $\alpha$}: Controls the concentration of the Dirichlet prior.
          Tested range: $[0.5, 10]$. Default: $\alpha = 1.5$.
    \item \textbf{Prior correlation $\beta$}: Controls the weight of judges' scores in the prior.
          Tested range: $[0, 1.5]$. Default: $\beta = 0.3$.
    \item \textbf{Initial scale $\lambda_0$}: Controls the initial MCMC proposal step size.
          Tested range: $[0.02, 0.15]$. Default: $\lambda_0 = 0.06$.
\end{itemize}

\subsection{Robustness of Feature Weights}

The profession-based popularity weights in our prior (e.g., singers having larger
fan bases than news anchors) are based on domain knowledge rather than empirical
calibration. We specifically analyze whether model results are sensitive to these
weight settings.

Our analysis shows that the EPA variation across different weight configurations
remains within acceptable bounds (EPA range $< 0.1$), demonstrating that:
\begin{enumerate}
    \item The primary information driving vote estimates comes from the \textbf{elimination constraints}
          (likelihood function), not the prior weights.
    \item The prior serves mainly to regularize the solution and provide initial direction,
          but does not dominate the posterior.
    \item Even with substantially different weight assumptions, the model reaches
          similar conclusions about vote distributions.
\end{enumerate}

This robustness is expected: Bayesian inference with sufficient data (elimination events)
naturally down-weights prior influence as likelihood information accumulates.

\subsection{Stability Summary}

For each hyperparameter, we compute the EPA range (max - min) across tested values:
\begin{itemize}
    \item EPA range $< 0.1$: High stability
    \item EPA range $\in [0.1, 0.2]$: Medium stability
    \item EPA range $> 0.2$: Low stability (requires careful tuning)
\end{itemize}

Our results indicate that all four hyperparameters exhibit high to medium stability,
confirming that the model is robust to reasonable parameter variations.

% ----------------------------------------------------------------------------
% 第11章:模型验证
% ----------------------------------------------------------------------------
\section{Model Validation}

% TODO: 模型验证
% - 交叉验证
% - 留一季验证

% ----------------------------------------------------------------------------
% 第12章:优缺点
% ----------------------------------------------------------------------------
\section{Strengths and Weaknesses}

\subsection{Strengths}
% TODO: 优点

\subsection{Weaknesses}
% TODO: 缺点

% ----------------------------------------------------------------------------
% 第13章:结论
% ----------------------------------------------------------------------------
\section{Conclusions}

% TODO: 结论

% ----------------------------------------------------------------------------
% 备忘录
% ----------------------------------------------------------------------------
\newpage
\section{Memo to Producers}

\begin{center}
\textbf{MEMORANDUM}
\end{center}

\noindent\textbf{TO:} Dancing with the Stars Production Team \\
\textbf{FROM:} Team 2622678 \\
\textbf{DATE:} January 2026 \\
\textbf{RE:} Recommendations for Scoring System Improvements

\vspace{1em}

\subsection*{Executive Summary}
% TODO: 执行摘要

\subsection*{Key Findings}
% TODO: 关键发现

\subsection*{Recommendations}
% TODO: 建议

\subsection*{Implementation Considerations}
% TODO: 实施考虑

% ----------------------------------------------------------------------------
% 参考文献
% ----------------------------------------------------------------------------
\begin{thebibliography}{99}

% TODO: 添加参考文献
% \bibitem{1} Author, Title, Journal, Year.

\end{thebibliography}

\end{document}
